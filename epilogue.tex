% 后记
\chapter[后记]{后记}
\section[地图绘制]{地图绘制}
刚入学时,我就始终因为不熟悉学校布局,且实验室与教室门牌号复杂,需要提前许多才能按时进入教室而苦恼;后来,在需要去各个办公室提交申请资料时,又发现找不到办公室的地点。

当时,网络上的主流校园整体地图是由“林弄人”在2015年绘制的\uline{\href{https://www.zcool.com.cn/work/ZMTgxMDQwMjg=.html}{《手绘潍医》}},但其在鄙人看来略显抽象,部分建筑的大小、空间关系因美观而妥协的情况使得地图无法真实反应现实,也缺少了许多重要的细节。此外,学校的校园情况、建筑也在按实际需要逐步扩建、更改,更加让本应承担“指路”功能的地图易于“迷路”。

因此,在2022年10月左右,本人四处寻找适宜用作地图的图纸原稿,最终决定以“UG潍医”微信小程序的开发者提供的《潍坊医学院新校区修建性详细规划(调整)》为初稿,通过使用不同颜色的色块标识建筑与路面,根据实际情况增加路名与建筑名称,以学校实际建设情况为据增添新建筑等形成了初稿并发布(Mika为地图的曾用署名)。

而后,随着使用量的增加,各类建议也纷至沓来,地图的细节亦不断完善。此后,在各位校领导以及学校宣传部张主任的指导与不断帮助中,地图的用语更加规范,格式更加正式;此外,我又在21级临床医学院学工办刘主任的帮助下重制了教学楼的内部详细地图。

因虞河校区缺少地图,因此,在2023年寒假期间开始着手绘制虞河校区以及周边人民医院见习点的整体轮廓地图。在先前绘制累的经验的加持下,虞河校区地图于2024年3月10日正式完成。鄙人谨在此再一次向所有指导、审核与提供帮助的各位领导、教师以及同学致以诚挚的感谢!

\section[指南写作]{指南写作}

在2022年末,山二医校园频道(曾用名:潍坊医学院表白墙QQ频道)希望我基于自己改制的地图与其提供的原始大纲,编撰一份入学指南,我也很愉快地答应了。但是,在整合的过程中,我发现其中的众多内容不符合当今实际、许多语句措辞不当、各文章病句繁多、各文档内排版混乱等问题,我也一并进行了修正。

但是由于各种错误层出不穷,逐个修正费时耗力且成效甚微,我当即决定以供稿内容为基础,自行抽象出共性内容与特性内容,仅将供稿视作文章目录并系统地重写一份,以便使文章逻辑通顺、文风统一,也就有了今天的《山东第二医科大学入学与生活指南》。

在此过程中,学校各部门的领导、老师给予了我很大的帮助,没用各位领导、老师的鼎力相助也不可能有本指南的诞生,因文章长度所限无法一一致谢敬请见谅。

\section[效果优化]{效果优化}
\subsection[文字显示]{文字显示}
自豪地使用梦源宋体(思源宋体的改版)进行排版,作为宋体糟糕显示效果的替代品。

\subsection[图片压缩]{图片压缩}
在地图的制作过程中,因地图清晰度过高、图像画布过大导致的图片文件过大\footnotemark 的问题始终困扰着我。在反复研究后,我通过非文字区域的马赛克化、复用色彩、更换MozJpeg图片算法、图片文字矢量化等多种方法尽力减小图像体积,力求在地图“保真”的前提下使得地图便于传播。最终,学校地图以pdf版本的形式嵌入到了本指南中,且所有地图文件总大小不超过1.8M,实现了质的飞跃。
\footnotetext{导出后浮烟山校区整体地图的原始png文件大小可达75M以上。}

\subsection[书签、区域与表格排版]{书签、区域与表格排版}
一开始,本文采用的是Microsoft® Word 2021进行的排版。但是,在文稿频繁交换意见、改进与审核的交换过程中出现了不同版本Word显示不同,高分屏与普通屏排版不同,Word、LibreOffice Writer与WPS重复打开并保存后格式混乱需要全面重新排版,书签与超链接难以进行管理等诸多问题。因此,早在2022年末,我便已开始计划用TeX语言对此进行重写,但因种种原因未能付诸行动。

在2023年暑假期间,我着手使用LaTeX进行全面重构,在“庚午版 2023.7.21”在新生群发布以后,经慎重考虑,鄙人谨决定彻底放弃维护Word版本(原Word版本由“山二医校园频道”接手维护),用全部精力维护LaTeX版本以保证文稿质量。最终,在2023年8月13日,LaTeX初版维护完毕,并不断跟随学校实际更新。

\subsection[编辑]{编辑}
为避免文章引用混乱与格式在不同Office版本上的不断改变,本文章全部使用\TeX 语言撰写,使任何电脑都能在同时使用Tex Live套件与VSCode编辑器\footnotemark 的情况下获得完全相同的编辑体验(\sout{虽然相较Word编辑的入门门槛也高了亿点点})。最终发布时选用\XeLaTeX 多次编译以获得 $pdf$ 文件,保证无论是纸质版还是电子版都能在相同的格式下进行阅读。
\footnotetext{详见\uline{\href{https://gitee.com/LinkChou/sdsmu_welcome_tex/blob/master/README.md}{README.md}}。}

\section[实际应用]{实际应用}
最后,特别感谢刘书记的支持,在他的支持下,大多数临床医学院2023级新生在入学前通览了本文,避免了众多常见错误的再犯,显著减少了因使用违规电器导致的停电次数,极大地降低了学生因不熟悉校园而产生的各类问题的发生率!
