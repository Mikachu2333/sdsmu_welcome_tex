%chapter7
\chapter[学习方面]{学习方面}
\section[学分]{学分\footnote{因不同学制、学院、年级要求各不相同,本图仅以2021级临床医学院临床医学系普通5年制本科为例,依照教务处网站2021级培养方案制作,详情可参考学生手册和\href{https://jwch.wfmc.edu.cn/2022/0916/c5343a107934/page.htm}{\uline{教务处官网}}说明,如有变动恕不另行通知。}\vspace{-1em}}
\begin{table}[ht]
    \centering
    \includegraphics[width=\textwidth]{学分.jpg}
    \vspace{-1em}
    \caption[学分组成示意图]{学分组成示意图}
    \label{score}
\end{table}

\section[学费]{学费}
\begin{enumerate}
    \item 学费收缴工作按照学校财务处通知进行,通常以班级为单位进行通知,可开具电子发票
    \item 如需申请助学贷款、生源地贷款等有特殊情况的同学可咨询学校财务处
    \item 选修课学费按照学分进行收费,所以同一个班的同学学费亦可不同,个人学费以\textbf{“潍坊医学院财务”公众号(潍坊医学院统一支付平台)}中的数据为准
    \item 选修课学分当前规定为1分/100元,多选课多交钱,少选课少交钱(希望大家如果看到了自己希望进一步学习的课程不要吝啬那几百块钱,选课机会只有一次,课程不会重开!)
    \item 此外,一定结合上面的说明修够学分,\textbf{修不够规定学分不能毕业}
\end{enumerate}

\section[早操与晚自习]{早操与晚自习}
\begin{enumerate}
    \item 早操时间一般为6:00-7:00,晚自习时间通常为18:30-21:00,教室22:00关闭
    \item 早操与晚自习贯穿整个大一\footnote{按照惯例,麻醉专业无早操,只需要大一、大二早晨七点签到。},并且跑操与晚自习均有院系学生会不定时进行抽查人数是否到齐等指标,最终结果计入班级综测(详见下文)的评分
\end{enumerate}

\section[关于选修课的补充说明]{关于选修课的补充说明}
\begin{enumerate}
    \item 选修课分专业选修和公共选修两大类(详见\uline{\ref{score}}),\textbf{推荐在大一全年、大二上学期的把各类选修学分全都修满},这样就不用在后面学业愈重的情况下兼顾选修课的学习了,可以专心针对专业课程进行深入学习
    \item 专业选修课有限选和非限选之分,限选的课程无需操心,教务系统会自动选课,只需要保证非限选的课程学分达标即可
    \item 公共选修课每一类都要选至少一门,且需要满足总分,其中部分类别还有额外要求(详见\uline{\ref{score}}),国防教育类有国家限选课程(到时候看具体通知,会说的很明白的)
    \item 公共选修课有一部分是在教室上的,还有一部分是线上课程(使用“知到”app进行学习,大多数有平时分,不能突击),可以根据自己的实际情况选择\footnote{一般来讲线下课好过,每星期去一次教室听听课就行,结课考试也很简单;但是线上课随时都能刷课,刷完课考完试就不用每周都去听课了,根据自己的需求选择。}
    \item \textbf{\uuline{公共选修课联盟的公共选修课程不算学分、不收学费}}
\end{enumerate}

\section[特殊规定]{特殊规定}
\begin{enumerate}
    \item \textbf{\uuline{大一不组织参加四六级,大二才能报名四六级考试}}
    \item \textbf{\uuline{转专业}:}\footnote{依据2022年6月20日发布的《潍坊医学院2022年普通全日制本科学生转专业工作方案》简化而来,具体规定详见\uline{\href{https://jwch.wfmc.edu.cn/2022/0620/c2593a105784/page.htm}{教务处网站官方通告}}。今后如有变动,以学校官网为准。}
          \begin{enumerate}
              \item 需满足以下条件:
                    \begin{enumerate}
                        \item 学校普通全日制一年级本科(不含专升本)在校学生
                        \item 未受处分,思想过硬
                        \item 符合体检要求
                        \item 第一学年必修课和专业选修课平均成绩在本专业排名前30\%
                    \end{enumerate}
              \item 流程:
                    \begin{enumerate}
                        \item 确认考试科目为思政、英语、数学,比例为1:1:1
                        \item 公示各类数据
                        \item 8月29日开始报名
                        \item 9月2日组织考试
                        \item 9月5日-7日公示录取名单\footnote{按照“分数优先,遵循志愿”的原则从上往下按照既定转专业名额录取。}
                        \item 9月8日-9日报到
                    \end{enumerate}
          \end{enumerate}
    \item 关于奖学金:本校有国家奖学金、校长奖学金、校级3等级奖学金等
    \item 新生开学考试\footnote{不公布具体成绩}的内容为高中英语、高中数学,旨在让各位同学收心
    \item \textbf{关于档案填写:}开学以后需要大量填写各类表格,如果你听到“入档”这两个字时需要格外注意,入档资料具有不能涂改、不能标记、不能修正,遵循“三个不能”的原则,因此严禁使用修正液或修正带对其涂改(涂改修正后立即作废)。填表时务必注意各类时间填写务必正确、与原始档案一致。此外,建议初次填写时使用铅笔轻轻填写,待负责人确认无误后方可擦除后使用黑色中性笔或中油笔填写(入档纸张为特殊A4纸,与正常A4纸相比厚、重、滑,难以自行复印请注意)
    \item 挂科与补考、重修:因各年级、院系要求不一,详见学校下发的学生手册
    \item \textbf{关于实验课与实验服:在实验课上课时务必按要求正确洗手并佩戴头套、口罩、手套,着实验服,\uuline{严禁任何人在任何地点(尤其是餐厅)着实验服}!因在实验室中需频繁接触实验动物、微生物,\uuline{极其推荐将实验课书包与普通课书包分开}!}
\end{enumerate}