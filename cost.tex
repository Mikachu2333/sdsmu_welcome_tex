%费用
\chapter[费用、ATM与银行卡]{费用、ATM与银行卡}

\section[生活费]{生活费}
一般生活费范围约1000~1500元(仅吃饭和购买水果、牛奶等常规生活情景)

\section[学费]{学费\footnotemark}
\footnotetext{学费缴费系统使用教程详见\uline{\ref{fee_pay}},财务处地址:行政楼1层西侧。}
\begin{enumerate}
    \item 依惯例,新生开学前需要进行注册学费的缴纳工作,按照录取通知书说明以及新生预报到系统\footnotemark 的提示进行
          \footnotetext{预报到系统使用教程见此\uline{\ref{freshman_query}}。}
    \item 学分学费的收缴工作在开学后按照学校财务处通知进行,通常以班级为单位进行通知,可开具电子发票
    \item 如需申请助学贷款、生源地贷款等有特殊情况的同学详询财务处收费管理科
    \item 选修课学费按照学分进行收费,所以同一个班的同学学费亦可不同,个人学费以\textbf{“潍坊医学院财务”公众号(潍坊医学院统一支付平台)}中的数据为准
    \item 选修课学分当前规定为1分/100元,多选课多交钱,少选课少交钱(希望大家如果看到了自己希望进一步学习的课程不要吝啬那几百块钱,选课机会只有一次,课程不会重开!)
    \item 此外,一定结合培养方案的要求(参见\uline{\ref{score}})修够学分,\textbf{修不够规定学分不能毕业}
\end{enumerate}

\section[ATM]{ATM\footnotemark}
\footnotetext{根据业务动态调整,详情见各银行官网。}
\begin{enumerate}
    \item 工商银行:教学楼E区,靠近杏林路侧
    \item 邮政银行:中和广场,A103对过
\end{enumerate}

\section[银行卡]{银行卡}
学校随录取通知书寄出未激活银行卡一张,请按照官方提示进行操作。该银行卡将在大学5年中用于奖助学金和助学贷款发放等用途
