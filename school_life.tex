%chapter9
\chapter[衣食住玩与生活]{衣食住玩与生活}
\noindent\textbf{特别声明:}
\begin{enumerate}
    \item 本文中所有\textbf{“大服”},均为\textbf{“大学生服务中心”}的习惯性缩略称呼;
    \item 本文中所有\textbf{“南街”},均为\textbf{“汇金街”}的习惯性缩略称呼。
\end{enumerate}
\section[衣]{衣}
\begin{enumerate}
    \item 大服的2、3层均有服饰商店可自行选购衣物
    \item 推荐网购,也可乘71路等公共汽车前往大型商超购置
    \item 部分院系提供自愿的系服购买服务,详见各院系通知
\end{enumerate}

\newcommand\FootnoteTipAboutFood{%
    \footnote{因文章篇幅原因,本指南仅罗列了同学们提及次数较多的食物或店铺,未能全部列出敬请谅解。}%
    \footnote{下列提及的食物(店铺)均按照空间顺序排列,与好吃程度无关,所用名称为同学习惯性称呼,括号内为特别提醒。}%
    \footnote{标注“\textsuperscript{〈早〉}”的店铺约6:00即开始供应。}%
    \footnote{标注“\textsuperscript{〈晚〉}”的店铺营业时间最晚可至22:30,其余均在18:30左右停业。}%
    \footnote{奶茶/咖啡店、超市、水果店等单独说明。}%
}
\section[美食与生活]{美食与生活\FootnoteTipAboutFood}

\subsection[大服]{大服}
大服有大量商家提供多种食物,大部分的价格较食堂稍高。
\begin{table}[ht]
    \centering
    \begin{tabular}{|c|c|c|c|c|c|}
        \Xhline{1.2pt}
        \multirow{3}{*}{1层}  & \multirow{2}{*}{内}            & 金小麵$^{〈早〉}$(锅贴)                  %
                             & 自选菜                           & 老陕面馆                & 馋嘴鱼        \\
        \cline{3-6}
                             &                               & 米粉                               %
                             & 肠粉                            & 肉夹馍$^{〈晚〉}$         & 冒菜         \\
        \Xcline{2-6}{0.8pt}
                             & 外                             & 砂锅$^{〈早〉}$(火烧|豆脑)                %
                             & 大饼卷一切$^{〈晚〉}$                 & 速食主义$^{〈早〉}$        & 烧烤$^{〈晚〉}$ \\
        \Xhline{1.2pt}
        \multirow{2}{*}{-1层} & \multirow{2}{*}{$\backslash$} & 福香面馆$^{〈早〉}$(豆脑|油条)              %
                             & 兰李于                           & 螺狮粉                 & 自选菜        \\
        \cline{3-6}
                             &                               & 蟹王堡                              %
                             & 烤鸡架                           & 老陕面馆                & 馋嘴鱼        \\
        \Xhline{1.2pt}
    \end{tabular}
\end{table}

\subsection[杏林餐厅]{杏林餐厅}
\vspace{-1em}
杏林餐厅全部三层均有大量食物,大多物美价廉。
\newpage
\begin{table}[ht]
    \centering
    \begin{tabular}{|c|c|c|c|c|}
        \Xhline{1.2pt}
        \multirow{3}{*}{1层} & 麦西麦乐                                                & 包子水饺$^{〈早〉}$ %
                            & 牛肉板面                                                & 兰州拉面         \\
        \cline{2-5}
                            & 永和豆浆$^{〈早〉}$(油条|麻花)                                 & 自选菜(稍贵)      %
                            & 豆浆油条                                                & 盒饭(便宜量大)     \\
        \cline{2-5}
                            & 粥$^{〈早〉}$(种类多)                                      & 馄饨           %
                            & 麻辣烫                                                 & 略            \\
        \Xhline{1.2pt}
        \multirow{3}{*}{2层} & 大骨饭                                                 & 麻汁馄饨         %
                            & 水饺                                                  & 东北玉米面        \\
        \cline{2-5}
                            & 烤鸭饭(瓦罐汤)                                            & 铁板炒饭(量大管饱)   %
                            & 清真窗口                                                & 茶拌饭          \\
        \cline{2-5}
                            & 馋嘴鱼                                                 & 自选水饺         %
                            & \multicolumn{2}{c|}{略}                                             \\
        \Xhline{1.2pt}
        3层\footnotemark     & \multicolumn{4}{c|}{自选菜(稍贵;小包间式,部门聚餐推荐,包间人数上限为15人)}                \\
        \Xhline{1.2pt}
    \end{tabular}
\end{table}
\footnotetext{仅餐厅东南侧层梯可前往,餐厅东北侧层梯通往原乒乓球场。}

\subsection[汇金街]{汇金街\footnotemark}
\footnotetext{按照拼音顺序排列}
出学校南门,往东一个路口,有大量的饭店,价格较市里相对高昂。
\begin{table}[ht]
    \centering
    \begin{tabular}{|c|c|c|c|c|}
        \Xhline{1.2pt}
        满江红 & 暖溢水饺 & 石锅鱼 & 小四川 & 幸福餐厅 \\
        \Xhline{1.2pt}
    \end{tabular}
\end{table}

\subsection[超市]{超市}\label{market}
\begin{table}[ht]
    \centering
    \begin{tabular}{|c|c|c|}
        \Xhline{1.2pt}
        习惯称呼         & 地点      & 物品                   \\
        \Xhline{1.2pt}
        大服超市$^{〈晚〉}$ & 在大服正中央  & 日用品,零食,饮料,手套,头套,作业本等 \\
        \hline
        中和超市         & 中和广场    & 日用品(少)、零食、饮料等        \\
        \hline
        餐厅超市         & 餐厅西北侧入口 & 零食、饮料等               \\
        \Xhline{1.2pt}
    \end{tabular}
\end{table}

\subsection[水果店]{水果店}
\begin{table}[!ht]
    \centering
    \begin{tabular}{|c|c|c|c|c|c|}
        \Xhline{1.2pt}
        习惯称呼    & 地点                     & 种类 & 新鲜   & 价格 \\
        \Xhline{1.2pt}
        餐厅南水果店  & 餐厅正南侧入口                & 较多 & 较好   & 略高 \\
        \hline
        餐厅西水果店  & 餐厅正西侧入口                & 较少 & 一般   & 一般 \\
        \hline
        大服水果店   & 大服西南侧                  & 最多 & 一般或好 & 最高 \\
        \hline
        中和/大服超市 & 见 \uline{\ref{market}} & 最少 & 一般   & 最低 \\
        \Xhline{1.2pt}
    \end{tabular}
\end{table}

\subsection[奶茶/咖啡店]{奶茶/咖啡店}
\begin{table}[!ht]
    \centering
    \begin{tabular}{|c|c|c|c|c|c|}
        \Xhline{1.2pt}
        \multirow{3}{*}{食堂} & \multirow{2}{*}{内部}               & 蜜雪冰城         & 沪上阿姨  %
                            & 阿水大杯茶                             & 麦克风                  \\
        \cline{3-6}
                            &                                   & 小度           & 冰雪岛   %
                            & 超级奶爸                              & $\backslash$         \\
        \cline{2-6}
                            & 外部                                & 潍医咖啡         & 臻茶    %
                            & \multicolumn{2}{c|}{$\backslash$}                        \\
        \Xhline{1.2pt}
        \multirow{2}{*}{大服} & 1层                                & 茶百道          & 益禾堂   %
                            & \multicolumn{2}{c|}{$\backslash$}                        \\
        \cline{2-6}
                            & 2层                                & 库迪咖啡         & 遇觅烧仙草 %
                            & \multicolumn{2}{c|}{$\backslash$}                        \\
        \Xhline{1.2pt}
    \end{tabular}
\end{table}

\newpage
\subsection[其他常用地点]{其他常用地点}
\label{common_locations}
\begin{table}[!ht]
    \vspace{-1em}
    \centering
    \begin{tabular}{|c|c|c|c|}
        \Xhline{1.2pt}
        地点                    & 习惯称呼                         & 位置     & 功能           \\
        \Xhline{1.2pt}
        \multirow{9}{*}{大服}   & 厕所                           & -1层东北  & 略            \\
        \cline{2-4}
                              & 联通营业厅                        & 1层超市旁                 %
                              & 联通业务办理                                               \\
        \cline{2-4}
                              & 药店、牙科                        & 2层西北                  %
                              & 买药、看牙、\textbf{冷藏药品}                                  \\
        \cline{2-4}
                              & 理发店(两家)                      & 2层     & 烫染剪发         \\
        \cline{2-4}
                              & 复印店(两家)                      & 2层                    %
                              & \textbf{打印复印扫描、证件照}、复习资料购买                           \\
        \cline{2-4}
                              & 移动营业厅                        & 2层     & 移动业务办理       \\
        \cline{2-4}
                              & 电信营业厅                        & 2层     & 电信业务办理       \\
        \cline{2-4}
                              & 广电营业厅                        & 2层     & 广电业务办理       \\
        \cline{2-4}
                              & 干洗店                          & 2层东    & 干洗、实验服购买、配钥匙 \\
        \cline{2-4}
                              & 裁缝店                          & 2层东南   & 改衣           \\
        \cline{2-4}
                              & 维修店                          & 2层东南                  %
                              & 手机电脑维修、手机配件购买                                        \\
        \cline{2-4}
                              & \textbf{办公室}                 & 2层东北   & 水卡的办卡充值退卡    \\
        \cline{2-4}
                              & 大服健身房\footnotemark           & 3层                    %
                              & 运动健身、办理健身会员卡                                         \\
        \cline{2-4}
                              & 台球厅\                         & 3层     & 打台球          \\
        \Xhline{1.2pt}
        \multirow{3}{*}{中和广场} & 学生印务                         & A106对过                %
                              & 打印复印扫描、\textbf{复习资料购买、二手书}                           \\
        \Xhline{1.2pt}
        \multirow{4}{*}{其他}   & 证件照                          & B207旁                 %
                              & \textbf{证件照}、特殊复印(80g/120g纸)                         \\
        \cline{2-4}
                              & \textbf{证明打印}                & D105旁                 %
                              & \textbf{学籍证明、成绩证明}等                                  \\
        \cline{2-4}
                              & 自助打印                         & 餐厅北侧   & 打印           \\
        \cline{2-4}
                              & 二手书买卖                        & 大服西北角                 %
                              & \textbf{二手书}(大量)                                     \\
        \Xhline{1.2pt}
    \end{tabular}
\end{table}
\footnotetext{仅大服西北侧楼梯可前往,健身卡收费详情咨询工作人员,与文体中心健身房不同。}

\section[住]{住}
\begin{enumerate}
    \item 宾馆:南街提供大量宾馆、客房等
    \item 自习室:南街部分宾馆提供通宵自习服务
    \item 出租房:附近小区由较多房屋出租\footnotemark
\end{enumerate}
\footnotetext{须在学校办理走读手续后才可在外居住。}

\section[玩]{玩}
\begin{enumerate}
    \item 多数同学常通过步行前往南街,有KTV、电影院等娱乐场所
    \item 可通过69、71路等公交车(北门乘坐)或13路等公交车(南门乘坐)前往市区(如泰华、万达、谷德茂等)游玩
    \item 文体中心(位置参见\uline{\ref{map_a}})内有羽毛球馆、篮球馆(两者互斥)、健身房,还有\textbf{游泳馆}等\footnotemark
          \footnotetext{具体收费标准及预约方式见下文\uline{\ref{sports_center}}。}
\end{enumerate}