%chapter9
\chapter[衣食住玩与生活]{衣食住玩与生活}
\section*{特别声明}
\begin{enumerate}
    \item 本文中所有\textbf{“大服”},均为\textbf{“大学生服务中心”}的习惯性缩略称呼;
    \item 本文中所有\textbf{“南街”},均为\textbf{“汇金街”}的习惯性缩略称呼。
\end{enumerate}
\section[衣]{衣}
\begin{enumerate}
    \item 大服的2、3层均有服饰商店可自行选购衣物
    \item 推荐网购,也可乘71路等公共汽车前往大型商超购置
    \item 部分院系提供自愿的系服购买服务,详见各院系通知
\end{enumerate}

\section[美食与生活]{美食与生活}

\subsection*{注意}
\begin{enumerate}
    \item 因文章篇幅原因,本指南仅罗列了同学们提及次数较多的食物或店铺,未能全部列出敬请谅解;
    \item 下列提及的店铺(食物)均按照空间顺序排列,与好吃程度无关;
    \item 所用名称为同学习惯性称呼,括号内为特别提醒。
    \item 上标“㊐”的店铺夏季约6:00开始供应(冬季约6:30);
    \item 上标“㊰”的店铺营业时间最晚可至22:30,其余均在18:30左右停业。
    \item 奶茶/咖啡店、超市、水果店等单独说明。
\end{enumerate}

\subsection[大服]{大服}
大服有大量商家提供多种食物,大部分的价格较食堂稍高。
\begin{table}[H]
    \centering
    \begin{tabular}{|c|c|c|c|c|c|}
        \Xhline{1.2pt}
        \multirow{3}{*}{1层}  & \multirow{2}{*}{内}                              %
                             & 金小麵$^{㊐}$(锅贴)                 & 自选菜             %
                             & 陕西面馆                          & 馋嘴鱼             \\
        \cline{3-6}
                             &                                                 %
                             & 新疆炒米粉                         & 肠粉              %
                             & 肉夹馍$^{㊰}$                     & 冒菜              \\
        \Xcline{2-6}{0.8pt}
                             & 外                                               %
                             & 烧烤$^{㊰}$                      & 砂锅$^{㊐}$(火烧|豆脑) %
                             & 大饼卷一切$^{㊰}$                   & 速食主义$^{㊐}$      \\
        \Xhline{1.2pt}
        \multirow{2}{*}{-1层} & \multirow{2}{*}{$\backslash$}                   %
                             & 兰李于                           & 自选菜             %
                             & 酸菜鱼                           & 螺狮粉             \\
        \cline{3-6}
                             &                                                 %
                             & 烤鸡架                           & 宽巷面馆            %
                             & 馋嘴鱼                           & 略               \\
        \Xhline{1.2pt}
    \end{tabular}
\end{table}

\subsection[杏林餐厅]{杏林餐厅}

杏林餐厅全部三层均有大量食物,大多物美价廉。
\begin{table}[H]
    \centering
    \begin{tabular}{|c|c|c|c|c|}
        \Xhline{1.2pt}
        \multirow{3}{*}{1层} & 麦西麦乐                & 包子水饺$^{㊐}$ %
                            & 牛肉板面                & 兰州拉面       \\
        \cline{2-5}
                            & 永和豆浆$^{㊐}$(油条|麻花)   & 自选菜(稍贵)    %
                            & 豆腐脑                 & 盒饭(便宜量大)   \\
        \cline{2-5}
                            & 粥$^{㊐}$(种类多)        & 馄饨$^{㊐}$   %
                            & 麻辣烫                 & 烤夫王        \\
        \Xhline{1.2pt}
        \multirow{3}{*}{2层} & 大骨饭                 & 麻汁馄饨       %
                            & 水饺                  & 东北玉米面      \\
        \cline{2-5}
                            & 烤鸭饭(瓦罐汤)            & 铁板炒饭(量大管饱) %
                            & 米线                  & 清真窗口       \\
        \cline{2-5}
                            & 馋嘴鱼                 & 自选水饺       %
                            & 茶拌饭                 & 略          \\
        \Xhline{1.2pt}
        3层\footnotemark     & \multicolumn{4}{c|}              %
        {略(较贵;有包间,部门聚餐可选,包间人数上限为12人)}                          \\
        \Xhline{1.2pt}
    \end{tabular}
\end{table}
\footnotetext{除餐厅东南侧楼梯外均可前往。}

\subsection[汇金街]{汇金街\footnotemark}
\footnotetext{按照拼音顺序排列}
出学校南门,往东一个路口。有大量的饭店,价格大多较市里相对高昂,部分味道一般。
\begin{table}[ht]
    \centering
    \begin{tabular}{|c|c|c|c|}
        \Xhline{1.2pt}
        满江红  & 暖溢水饺(相对平价) & 石锅鱼  & 小四川        \\
        \hline
        志科全驴 & 炖大鹅        & 生炖羊茬 & 幸福餐厅(平价量大) \\
        \Xhline{1.2pt}
    \end{tabular}
\end{table}

\subsection[超市]{超市}
\label{market}
\begin{table}[ht]
    \centering
    \begin{tabular}{|c|c|c|}
        \Xhline{1.2pt}
        习惯称呼       & 地点      & 物品                   \\
        \Xhline{1.2pt}
        大服超市$^{㊰}$ & 在大服正中央  & 日用品,零食,饮料,手套,头套,作业本等 \\
        \hline
        中和超市       & 中和广场    & 日用品(少)、零食、饮料等        \\
        \hline
        餐厅超市       & 餐厅西北侧入口 & 零食、饮料等               \\
        \Xhline{1.2pt}
    \end{tabular}
\end{table}

\subsection[水果店]{水果店}
\begin{table}[H]
    \centering
    \begin{tabular}{|c|c|c|c|c|c|}
        \Xhline{1.2pt}
        习惯称呼    & 地点                     & 种类 & 新鲜   & 价格 \\
        \Xhline{1.2pt}
        餐厅南水果店  & 餐厅正南侧入口                & 较多 & 较好   & 略高 \\
        \hline
        餐厅西水果店  & 餐厅正西侧入口                & 较少 & 一般   & 一般 \\
        \hline
        大服水果店   & 大服西南侧                  & 最多 & 一般或好 & 最高 \\
        \hline
        中和/大服超市 & 见 \uline{\ref{market}} & 最少 & 一般   & 最低 \\
        \Xhline{1.2pt}
    \end{tabular}
\end{table}

\subsection[奶茶/咖啡店]{奶茶/咖啡店}
\begin{table}[H]
    \centering
    \begin{tabular}{|c|c|c|c|c|c|}
        \Xhline{1.2pt}
        \multirow{2}{*}{食堂} & \textbf{蜜雪冰城} & 臻茶            & 沪上阿姨  %
                            & 阿水大杯茶         & 麦克风                   \\
        \cline{2-6}
                            & 超级奶爸          & 小度            & 冰雪岛   %
                            & \textbf{瑞幸咖啡} & $\backslash$          \\
        \Xhline{1.2pt}
        \multirow{2}{*}{大服} & 1层/-1层        & \textbf{茶百道}  & 益禾堂   %
                            & 幸运咖           & 归臻咖啡                  \\
        \cline{2-6}
                            & 2层            & \textbf{库迪咖啡} & 遇觅烧仙草 %
                            & 手打冰沙          & $\backslash$          \\
        \Xhline{1.2pt}
    \end{tabular}
\end{table}

\subsection[其他常用地点]{其他常用地点}
\label{common_locations}
\begin{table}[H]
    \centering
    \begin{tabular}{|c|c|c|c|}
        \Xhline{1.2pt}
        地点                    & 习惯称呼          & 位置         & 功能                           \\
        \Xhline{1.2pt}
        \multirow{16}{*}{大服}  & 厕所            & -1层东北      & 略                            \\
        \cline{2-4}
                              & 联通营业厅         & 1层超市旁      & 联通业务办理                       \\
        \cline{2-4}
                              & 药店、牙科         & 2层西北       & 买药、看牙、\textbf{冷藏药品}          \\
        \cline{2-4}
                              & 理发店(三家)       & 2层         & 烫染剪发                         \\
        \cline{2-4}
                              & 复印店(两家)       & 2层         & \textbf{打印复印扫描、证件照}、 复习资料    \\
        \cline{2-4}
                              & 电信营业厅         & 2层         & 电信业务办理                       \\
        \cline{2-4}
                              & 移动业务咨询处       & 2层东        & 移动业务咨询                       \\
        \cline{2-4}
                              & 广电营业厅         & 2层北        & 广电业务办理                       \\
        \cline{2-4}
                              & 干洗店           & 2层东        & 干洗、实验服购买、配钥匙                 \\
        \cline{2-4}
                              & 裁缝店           & 2层东南       & 改衣                           \\
        \cline{2-4}
                              & 维修店           & 2层东南       & 手机电脑维修、配件购买                  \\
        \cline{2-4}
                              & 联想服务中心        & 2层西        & 维修检测、配件购买                    \\
        \cline{2-4}
                              & \textbf{办公室}  & 2层东北       & 办水卡、充值退卡                     \\
        \cline{2-4}
                              & 大服健身房                                                     %
        \footnotemark %
                              & 3层            & 运动健身、办理会员卡                                \\
        \cline{2-4}
                              & 台球厅           & 3层         & 打台球                          \\
        \cline{2-4}
                              & 彩购师           & 3层         & 衣物与饰品购买                      \\
        \Xhline{1.2pt}
        \multirow{2}{*}{中和广场} & 学生印务          & A106对过     & 打印复印扫描、\textbf{复习资料、二手书}     \\
        \cline{2-4}
                              & 移动营业厅         & A104对过     & 移动业务办理                       \\
        \Xhline{1.2pt}
        \multirow{5}{*}{其他}   & 证件照           & B207旁      & \textbf{证件照}、特殊复印(80g/120g纸) \\
        \cline{2-4}
                              & \textbf{证明打印} & D105旁      & \textbf{学籍证明、成绩证明}等          \\
        \cline{2-4}
                              & 自助打印          & 餐厅北侧       & 打印                           \\
        \cline{2-4}
                              & 二手书买卖         & 大服西北角      & \textbf{二手书}(大量)             \\
        \cline{2-4}
                              & 校医院           & 仁和山        & \textbf{药品购买与冷藏,证明开具}        \\
        \Xhline{1.2pt}
    \end{tabular}
\end{table}
\footnotetext{仅大服西北侧楼梯可前往,健身卡收费详情咨询工作人员,与文体中心健身房不同。}

\section[住]{住}
\begin{enumerate}
    \item 宾馆:南街提供大量宾馆、客房等
    \item 自习室:南街部分宾馆提供通宵自习服务
    \item 出租房:附近小区由较多房屋出租\footnotemark
\end{enumerate}
\footnotetext{须在学校办理走读手续后才可在外居住。}

\section[玩]{玩}
\begin{enumerate}
    \item 多数同学常通过步行前往南街,有KTV、电影院等娱乐场所
    \item 也可通过69、71、101路等公交车(北门乘坐)或13、109路等公交车(南门乘坐)前往市区(如泰华、万达、谷德茂等)游玩
    \item 文体中心(位置参见\uline{\ref{map_a}})内有羽毛球馆、篮球馆(两者互斥)、健身房,还有\textbf{游泳馆}等\footnotemark
          \footnotetext{具体收费标准及预约方式见下文\uline{\ref{sports_center}}。}
\end{enumerate}