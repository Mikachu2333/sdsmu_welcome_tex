%chapter11
\chapter[常用教程]{常用教程}

(一)校园网
1.到校报到完成后,学校将按照个人身份证号在 http://210.44.80.65/ 为大家开通校园网账号,并告知初始密码(每年不同,详见具体规定);
2.充值:点击 https://slzfw.wfmc.edu.cn ,或在校园网的登陆页面点击“自助服务”按钮,登陆后即可使用支付宝充值(微信支付无法使用);
3.校园网流量当前为每月免费60G。

(二)校园手机卡
1.绝大部分流量仅限山东省内使用;通常套餐内含全国流量10G,校区流量130G+;
2.随录取通知书一并寄出,是否开通按需进行。

(三)空调使用教程
1.空调右下方有二维码,微信关注“海享租”公众号,点击“在线租赁”模块,注册账号后点击“扫一扫”功能,扫描空调右下角二维码进行租赁;
2.若租赁失败,请根据小程序提示,联系上一级同宿舍的学长或学姐,让其退租,成功后即可重新租赁;
3.租赁完成后,点击“设备”-“空调图标”-“时长”,进行充值即可;
4.点击“设备”-“空调图标”-“成员管理”,在此功能下设置宿舍全部成员均可管理空调开关;
5.空调按小时收费(0.55元/小时);通常在选出舍长后,由舍长统一集中收费并据实结算。
(四)浴室预约教程
1.在手机应用市场下载“大白U帮”app,按照实际住宿情况注册后授予其“定位”、“蓝牙”权限并充值;
2.各宿舍楼一楼的大浴室需要在软件开始页面中预约使用;
3.其他宿舍楼层使用淋浴时请点击页面最上方的按钮(标志如右),
选择“蓝牙设备”,开启手机定位功能,按软件指示进行即可;洗澡结束后务必按流程退出软件,否则可能会误扣费;
4.具体收费标准详见软件规定。

(五)洗衣机使用教程
1.前往宿舍一层公共厕所隔壁的洗衣间,微信扫码注册后使用即可;
2.需要自带洗衣液(极个别同学可能会使用洗衣机洗鞋);
3.收费标准详见软件提示。

(六)吹风机使用教程
1.每层公共浴室旁边有两个公用吹风机,需扫码租赁使用;
2.手机发出“滴”-“滴”的声音后租赁成功,将手机扬声器对准吹风机租赁器方可正常使用;
3.收费标准:详见软件提示,1分钱起步。

(七)设施报修教程
1.加入各宿舍楼的QQ报修群,在群内反映具体故障;或前往一层宿管处填表报修;
2.也可在宿舍一层宿管旁边的公告栏处查看负责该楼层的物业电话,直接拨打即可。

(八)学工系统(微信小程序)
1.学工系统主要用于晚点名、体温填报、返校信息填报、外出审批等日常工作;
2.在微信搜索“智慧学工”小程序,授予“定位”权限后根据学校下发的账号密码进行登录(推荐立即与微信绑定以免忘记密码)。

(九)教务系统
教务系统 https://jwgl.wfmc.edu.cn/ 主要负责选课,课程表查看,缓考申请,成绩查询等。
(十)资源访问控制系统
1.本系统用于在校外访问校内网络信息资源,如:知网、教务系统等;
2.登录 https://webvpn.wfmc.edu.cn/login ,点击“统一身份认证登录”按钮,按照学校下发的专用账号、密码登录即可;
3.如需在校外访问教务系统,请完成上述操作后选择“教务系统-非单点登录”;
4.如需校外查阅文献推荐使用CARSI系统,账密同校园网账密;
5.因本系统响应速度较慢,推荐在校内直接登录教务系统、知网等的官网进行操作。

(十一)校务行(微信小程序)
1.用于快速成绩查询、成绩报告单打印、学籍证明打印;
2.打印的收费标准详见小程序内部说明。

(十二)常用课程表程序
1.wakeup课程表:各手机应用商店直接下载,按教程进行即可,无广告;
2.超级课程表:各手机应用商店直接下载,有广告,使用人数较多。

(十三)学生票购买流程(具体规定详见12306官网)
1.入学后,根据学校统一通知,网上填报“家乡⇄学校”的最近的火车站点信息;
2.学生证注册章盖章后(每学期均需要),拿学生证、身份证前往任一火车站自动售票机处,选择“学生优惠资质绑定”按钮,按提示进行操作;或直接在人工售票处完成学生优惠资质认证(注意:每年均需重新认证学生资质一次);
3.全年有4次优惠机会,高铁75折,普通火车5折。

(十四)文体中心预约教程
1.关注“潍医文体中心”公众号;
2.使用cas认证系统的账号密码登录,并录入面部信息,预约并缴费后即可前往;
3.收费情况详见公众号文章,校外人员可在有余票时直接千万前往前台购票入场。