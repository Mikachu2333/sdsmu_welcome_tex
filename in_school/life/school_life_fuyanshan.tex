% 生活
\section[衣食购住玩与生活]{衣食购住玩与生活}

\subsection*{特别声明}
\begin{enumerate}
    \item 本文中所有\textbf{“大服”},均为\textbf{“大学生服务中心”}的习惯性缩略称呼;
    \item 本文中所有\textbf{“南街”},均为\textbf{“汇金街”}的习惯性缩略称呼。
\end{enumerate}
\subsection[衣]{衣}
\begin{enumerate}
    \item 大服的2、3层均有服饰商店可自行选购衣物
    \item 推荐网购,也可根据后文\uline{\ref{free_bus}}公交车信息前往大型商超购置
    \item 部分院系提供自愿的系服购买服务,详见各院系通知
\end{enumerate}

\subsection[食]{食}
\subsubsection*{注意}
\begin{enumerate}
    \item 因文章篇幅原因,本指南仅列举了同学们提及次数较多的部分食物或店铺,敬请谅解;
    \item 下列提及的店铺(食物)均按照空间顺序排列,与好吃程度无关;
    \item 所用名称为同学习惯性称呼,括号内为特别提醒或补充说明。
    \item 上标“㊐”的店铺夏季约6:00开始供应(冬季约6:30);
    \item 上标“㊰”的店铺营业时间最晚可至22:30,其余均在18:30~20:30左右停业;
    \item 上标“㊒”的店铺因装修等原因尚未开业或长期停业(超过一周);
    \item 奶茶/咖啡店、水果店等单独说明。
\end{enumerate}

\subsubsection[杏林餐厅]{杏林餐厅}
杏林餐厅全部三层均有大量食物,大多物美价廉。
\begin{table}[H]
    \centering
    \begin{tblr}[
            theme = {no-caption},
            note{1} = {除餐厅东南侧楼梯外均可到达。},
        ]{
            cells = {c,m},
            cell{1}{1} = {r=3}{},
            cell{4}{1} = {r=3}{},
            cell{7}{2} = {c=4}{},
            vlines,
            hlines,
            hline{1,4,7-8} = {-}{1pt},
            column{1} = {cmd=\bfseries},
        }
        1层             & 麦西麦乐                                     & 包子水饺$^㊐$        & 牛肉板面 & 兰州拉面         \\
                        & 大米粒儿$^㊐$(油条)                        & 自选菜(稍贵)       & 豆腐脑   & 盒饭(便宜量大) \\
                        & 粥$^㊐$(种类多)                            & 馄饨$^㊐$            & 麻辣烫   & 烤夫王           \\
        2层             & 大骨饭                                       & 麻汁馄饨             & 水饺     & 东北玉米面       \\
                        & 烤鸭饭(瓦罐汤)                             & 铁板炒饭(量大管饱) & 米线     & 清真窗口         \\
                        & 馋嘴鱼                                       & 自选水饺             & 茶拌饭   & 略               \\
        3层\TblrNote{1} & 略(较贵。有包间、舞台、音响等,需提前预约) &                      &          &
    \end{tblr}
\end{table}

\subsubsection[大服]{大服}
大服有大量商家提供多种食物,大部分的价格较食堂稍高。
\begin{tblr}[
        long,
        theme = {no-caption},
    ]{
        cells = {c,m},
        cell{1}{1} = {r=3}{},
        cell{1}{2} = {r=2}{},
        cell{4}{1} = {r=2}{},
        cell{4}{2} = {r=2}{},
        vlines,
        hlines,
        hline{1,4,6} = {-}{1pt},
        hline{3} = {2-6}{0.8pt},
        column{1} = {cmd=\bfseries},
    }
    1层   & 内           & 金小麵$^㊐$(锅贴) & 自选菜                          & 陕西面馆        & 馋嘴鱼        \\*
          &              & 新疆炒米粉          & 肠粉                            & 肉夹馍$^㊰$     & 冒菜          \\*
          & 外           & 烧烤$^㊰$           & 砂锅$^㊐$(火烧\textbar{}豆脑) & 大饼卷一切$^㊰$ & 速食主义$^㊐$ \\
    --1层 & $\backslash$ & 兰李于              & 自选菜                          & 酸菜鱼          & 螺狮粉        \\*
          &              & 烤鸡架              & 宽巷面馆                        & 馋嘴鱼          & 略
\end{tblr}

\newpage
\subsubsection[汇金街]{汇金街}
出学校南门,往东一个路口。有大量的饭店,价格大多较市里相对高昂,部分味道一般。
\begin{table}[H]
    \centering
    \begin{tblr}[
            theme = {no-caption},
        ]{
            cells = {c,m},
            hline{1,3}= {-}{1pt},
            vlines,
            hlines,
        }
        满江红   & 暖溢水饺(相对平价) & 木南王府 & 小四川烧烤           \\
        志科全驴 & 炖大鹅               & 生炖羊茬 & 幸福餐厅(平价量大)
    \end{tblr}
\end{table}

\subsubsection[水果店]{水果店}
\begin{table}[H]
    \centering
    \begin{tblr}[
            theme = {no-caption},
        ]{
            cells = {c,m},
            hline{1-2,6}= {-}{1pt},
            hlines,
            vlines,
            row{1} = {cmd=\bfseries},
        }
        习惯称呼      & 地点                              & 种类 & 新鲜 & 价格 \\
        餐厅南水果店  & 餐厅正南侧入口                    & 较多 & 较好 & 略高 \\
        餐厅西水果店  & 餐厅正西侧入口                    & 较少 & 一般 & 一般 \\
        大服水果店    & 大服西南侧                        & 最多 & 一般 & 最高 \\
        中和/大服超市 & 见 \uline{\ref{market_fuyanshan}} & 最少 & 一般 & 一般
    \end{tblr}
\end{table}


\subsubsection[奶茶/咖啡店]{奶茶/咖啡店}
\begin{table}[H]
    \centering
    \begin{tblr}[
            theme = {no-caption},
        ]{
            cells = {c,m},
            cell{1}{1}={r=2}{},
            cell{3}{1}={r=2}{},
            hlines,
            vlines,
            hline{1,3,5}= {-}{1pt},
            hline{4} = {2-6}{0.8pt},
            column{1} = {cmd=\bfseries},
        }
        食堂 & \textbf{蜜雪冰城} & 臻茶              & 沪上阿姨   & 阿水大杯茶        & 麦克风       \\
             & 超级奶爸          & 小度              & 冰雪岛     & \textbf{瑞幸咖啡} & $\backslash$ \\
        大服 & 1层/--1层         & \textbf{茶百道}   & 益禾堂     & \textbf{幸运咖}   & 归臻咖啡     \\
             & 2层               & \textbf{库迪咖啡} & 遇觅烧仙草 & 手打冰沙          & $\backslash$ \\
    \end{tblr}
\end{table}

\subsection[购]{购}
\begin{table}[H]
    \centering
    \label{market_fuyanshan}
    \begin{tblr}[
            theme = {no-caption},
        ]{
            cells = {c,m},
            hline{1-2,5}= {-}{1pt},
            hlines,
            vlines,
            row{1} = {cmd=\bfseries},
        }
        习惯称呼      & 地点           & 物品                                     \\
        大服超市$^㊰$ & 在大服正中央   & 日用品,零食,饮料,手套,头套,作业本等 \\
        中和超市      & 中和广场       & 日用品(少),零食,饮料,作业本等       \\
        餐厅超市      & 餐厅西北侧入口 & 餐巾纸、零食、饮料等
    \end{tblr}
\end{table}

\subsection[玩]{玩}
\begin{enumerate}
    \item 多数同学常通过步行前往南街,有KTV、电影院等娱乐场所
    \item 汇金街每逢农历初三、初八、十三、十八、廿三、廿八有集
    \item 可通过13、169、路等公交车(北门乘坐)或13、109路等公交车(南门乘坐)前往市区(如泰华、万达、谷德茂等)游玩\footnotemark
          \footnotetext{详情公交信息与免费乘车指南参见\uline{\ref{free_bus}}。}
    \item 文体中心(位置参见\uline{\ref{map_fuyanshan_holistic}})内有羽毛球馆、篮球馆(两者互斥)、健身房,还有\textbf{游泳馆}等\footnotemark
          \footnotetext{具体收费标准及预约方式见下文\uline{\ref{sports_center_book}},开放时间见此\uline{\ref{sports_center_operating_hours}}。}
\end{enumerate}

\begin{table}[H]
    \centering
    \label{sports_center_operating_hours}
    \caption[文体中心开放时间]{文体中心开放时间}
    \begin{tblr}[
        note{1} = {仅限校内,校外政策详见公众号或咨询工作人员;具体政策请以学校通知为准。},
    ]{
        cells = {c,m},
        row{1} = {font=\bfseries},
        column{1} = {font=\bfseries},
        cell{1,3,12}{1} = {r=2}{},
        cell{1}{2} = {c=3}{},
        cell{3}{3} = {c=2,r=5}{},
        cell{5}{1} = {r=3}{},
        cell{8}{1} = {r=4}{},
        cell{8,10}{2} = {r=2}{},
        cell{8-11}{3} = {c=2}{},
        cell{12}{2} = {c=2,r=2}{},
        vlines,
        hlines,
        hline{1,3,8,12,14} = {-}{1pt},
        hline{2,10} = {2-4}{0.8pt},
        hline{5} = {1-2}{1pt},
    }
    开放项目 & 春夏季营业时间                                   %
    \TblrNote{1}                                                \\
             & 周一至周四     & 周五         & 周末及法定节假日 \\
    健体中心 & 11:45~13:45   & 08:00~21:00                    \\
             & 18:00~21:00   &                                 \\
    羽毛球馆 & 08:00~09:30   &                                 \\
             & 12:00~13:30   &                                 \\
             & 18:00~21:00   &                                 \\
    游泳馆   & 12:00~14:00   & 09:00~11:00                    \\
             &                & 12:00~14:00                    \\
             & 18:00~20:00   & 15:00~17:00                    \\
             &                & 18:00~20:00                    \\
    乒乓球馆 & 18:00~20:00   &              & 08:00~12:00     \\
             &                &              & 14:00~20:00
\end{tblr}
\end{table}

\subsection[住]{住}
\begin{enumerate}
    \item 宾馆:南街提供大量宾馆、客房等
    \item 自习室:南街部分宾馆提供通宵自习服务
    \item 出租房:附近小区有较多房屋出租\footnotemark
          \footnotetext{须在学校办理走读手续后才可在外居住。}
\end{enumerate}

\subsection[自提点]{自提点}
\pagebreak
\begin{tblr}[
        long,
        theme = {no-caption},
    ]{
        cells = {c,m},
        cell{2}{1} = {r=2}{},
        cell{4}{1} = {r=2}{},
        row{1} = {font=\bfseries},
        rowhead = {1},
        vlines,
        hlines,
        hline{1-2,4,6} = {-}{1pt},
    }
    类别     & 地点           & 显示名称                 \\*
    美团优选 & 中和移动营业厅 & 山二医美团优选移动       \\*
             & 9号宿舍楼      & 山二医九号楼群内免费送货 \\
    多多买菜 & 中和移动营业厅 & 潍医多多买菜移动营业厅   \\*
             & 9号宿舍楼      & 山二医九号楼自提
\end{tblr}

\subsection[其他生活常用地点]{其他常用地点}
\begin{tblr}[
        long,
        caption = {其他常用生活地点详表},
        label = {common_locations_fuyanshan},
        note{1} = {清晰度较“学生印务”略高,少量打印时价格略高。},
        note{2} = {仅大服北侧楼梯可前往,健身卡收费详情咨询工作人员,与文体中心健身房不同。},
        note{3} = {注意,该邮局无信件投递及接收业务。},
    ]{
        cells = {c,m},
        row{1} = {font=\bfseries},
        column{1} = {font=\bfseries},
        cell{2}{1} = {r=17}{},
        cell{19}{1} = {r=3}{},
        cell{22}{1} = {r=5}{},
        vlines,
        hlines,
        hline{1-2,19,22,27} = {-}{1pt},
    }
    地点     & 习惯称呼                & 位置           & 功能                                     \\
    大服     & 厕所                    & --1层东北      & 略                                       \\
             & 联通营业厅              & 1层超市旁      & 联通业务办理                             \\
             & 药店、牙科诊所          & 2层西北        & 买药、看牙、\textbf{冷藏药品}            \\
             & 智慧驾校                & 2层西北        & 驾考咨询、驾校报名                       \\
             & 理发店(三家)          & 2层            & 烫染剪发                                 \\
             & 复印店(两家)          & 2层            & \textbf{打印复印扫描、证件照}、 复习资料 \\
             & 电信营业厅              & 2层            & 电信业务办理                             \\
             & 移动业务咨询处          & 2层东          & 移动业务咨询                             \\
             & 广电营业厅              & 2层北          & 广电业务办理                             \\
             & 干洗店                  & 2层东          & 干洗、实验服购买、配钥匙                 \\
             & 裁缝店                  & 2层东南        & 改衣                                     \\
             & 维修店                  & 2层东南        & 手机电脑维修、配件购买                   \\
             & 联想服务中心            & 2层西          & 维修检测、配件购买                       \\
             & \textbf{办公室}         & 2层东北        & 办水卡、充值退卡                         \\
             & 大服健身房 \TblrNote{2} & 3层            & 运动健身、办理会员卡                     \\
             & 台球厅                  & 3层            & 打台球                                   \\
             & 彩购师                  & 3层            & 衣物与饰品购买                           \\
    中和广场 & 学生印务                & A106对过       & 打印复印扫描、\textbf{复习资料、二手书}  \\
             & 移动营业厅              & A104对过       & 移动业务办理                             \\
             & 酷跑文印社\TblrNote{1}  & A103对过       & 打印复印扫描                             \\%
    \pagebreak
    其他     & 证件照                  & B207旁         & \textbf{证件照}、特殊复印(80g/120g纸)  \\
             & \textbf{证明打印}       & D105旁         & \textbf{学籍证明、成绩证明}等            \\
             & 自助打印                & 餐厅北侧       & 打印                                     \\
             & 二手书买卖              & 大服西北角     & \textbf{二手书}(大量)                  \\
             & 邮局                    & 餐厅西北侧入口 & 学校纪念品购买 \TblrNote{3}
\end{tblr}

