% 安全
\chapter[安全]{安全}

\section[用电安全]{用电安全}
\begin{enumerate}
    \item 使用安全合规的充电器、充电线、插排
    \item 如遇空调、电灯、电风扇等器具出现短路、跳闸、停电等现象时切勿自行维修,请按照\hyperref[repair_report]{此处}教程处理
    \item 切勿在宿舍使用冰箱(断电且宿管检查),\textbf{\uuline{如果各位同学需要冷藏保存药物(如胰岛素等),\linebreak[3]请前往校医院或大服2楼的药店处}}(地点参见\hyperref[common_locations_fuyanshan]{此处}),具体收费情况请咨询相关人员
    \item 切勿私改电路,如确有需要,应提前向宿管及公寓管理委员会\footnotemark 报备并获得相应许可
          \footnotetext{位于2号公寓东南侧,需提前预约。}
    \item 宿舍人走灭灯,无人则关电、切断插座电源
\end{enumerate}

\section[防火安全]{防火安全}
\begin{enumerate}
    \item 切勿使用蚊香以免发生火灾
    \item 宿舍内禁止烹饪,各类燃气灶、固体酒精便携灶、电磁炉等易燃易爆危险品均不允许使用
    \item \textbf{宿舍内及宿舍楼道均禁止吸烟},否则将触发烟雾报警器
    \item 请妥善保管打火机、打火石、镁条、火柴等易燃易爆物品
\end{enumerate}

\section[出行安全]{出行安全}
\begin{enumerate}
    \item 浮烟山校区周边基础设施尚在完善建设过程中(\sout{属实是兔葵燕麦、雨井烟垣}),且频繁有货车高速通过。如无特殊情况,\textbf{尽量不要骑公共自行车或者电动车去市里}(泰华城)等,推荐根据\hyperref[free_bus]{此处}的说明乘公交车前往游玩(预计行程50分钟左右)或打车前往
    \item 货车转弯盲区大,极其容易发生安全事故,在等红绿灯时请务必远离“禁止站立区域”
    \item 乘坐出租车(尤其是拼车时)请妥善保管自身财物;如遇失窃请尽快报警,避免正面冲突
    \item \textbf{\uuline{严格遵守交通规则,仔细观察周围情况,切忌边看手机边前进,切忌闯红灯}!}
\end{enumerate}

\section[食品安全]{食品安全}
如遇食物中毒请尽快前往校医院就医,其他食品安全问题可在小程序上投诉

\section[防诈骗及其他注意事项]{防诈骗及其他注意事项}
\begin{table}[H]
    \centering
    \large
    \textbf{\textcolor{red}{校园贷毁一生,远离高利贷!}}\\
    \textbf{\textcolor{red}{杜绝黄赌毒!不要高估自己的意志力!}}\\
    \textbf{\textcolor{red}{刷单就是诈骗!}}
\end{table}

\begin{enumerate}
    \item 贪小便宜乱扫码,信息泄露吃大亏!
    \item \textbf{各类\uuline{“通知群”“官方群”“学生公告群”“兼职事务群”等均不可信}!请在学校指引下加群或询问学长、学校教师,切勿轻信各类仿冒的通知群!}
    \item 如果碰到一些人自称是市场营销专业、经商专业的,需要卖笔卖本子\footnotemark (总之是找人要钱)才能完成期末考试的,千万不要相信!可以直接联系保卫处
          \footnotetext{一支批发0.1$¥$的笔卖10$¥$呢,比百乐斑马这种外国牌子都贵,利润高达10000\%……}
    \item 在浮烟山校区的女生晚上尽量不要独自前往人烟稀少的地方,尤其是西门附近的桃李路等
    \item 谨防诈骗,\textbf{学校永远不会以任何名义通过邮件表格或短信链接的形式通知填写敏感信息!}绝对不要相信以“更新银行卡信息”、“填表申请助学金”为由窃取密码、验证码的骗局!如果不确定消息是否属实请及时致电本班班长、班主任或学工办老师确认
    \item 各位家长在向学生转账时应\textbf{确认钱款具体用途、打款账户是否正确},同时应当\textbf{电话询问是否属实},谨防盗号诈骗
    \item 如在校内遇到物品丢失或失窃等情况需要调取监控录像的,请查看 \hyperref[sdsmu_app]{此处}的相关教程
\end{enumerate}
