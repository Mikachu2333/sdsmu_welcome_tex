% 宿舍
\section[宿舍条件]{宿舍条件}

\subsection[整体情况介绍]{整体情况介绍}
\begin{enumerate}
    \item 绝大多数的宿舍为6人间、上下床\footnotemark(各床铺靠墙侧均有1插座,可接插排),配备桌子(桌洞8个)一张、书桌(理论可坐3人但空间紧张,常2人,含6个小书架)一张、有柜子(常为8个)、垃圾桶、洗漱池、空调、电风扇、暖气片、烟雾报警器等宿舍基本设施
          \footnotetext{床铺尺寸为:200×90㎝;被子尺寸为:210×148㎝。}
    \item 柜子\footnotemark 一般为8个,6人各选一个柜子,剩下两个公用
          \footnotetext{空间尺寸参数大致如下:宽62㎝,高59㎝,深97㎝(各有1~2㎝误差)。}
    \item 具有独立卫生间
    \item 部分女生宿舍的卫生间内有浴室,可直接在卫生间内淋浴;其他同学可以在本楼层小公共澡堂(有2个毛玻璃隔间,有门帘,无需预约)或者一楼的大公共澡堂(不透明隔间,有门帘,需软件预约)洗澡(教程\uref{shower_software_f}{见此})
    \item 洗衣机位于每层楼的公共洗漱间内(教程\uref{washing_machine_f}{见此})
    \item 烘干机位于每栋楼1层公共洗漱间内(教程\uref{dry_machine}{见此})
    \item 吹风机位于每层楼的公共厕所旁(教程\uref{hair_drier}{见此})
    \item 大一期间,院学生会每周不定时抽查是否存在夜不归宿情况
    \item 空调采用单独线路供电,不受断电熄灯控制(全楼断电等特殊情况除外),需另外缴费(使用教程\uref{air_control}{见此})
    \item 使用门禁系统刷脸进出宿舍,宿舍楼06:00开门,23:00关门
    \item 每天23:00熄灯断电(\textbf{每周六、考试月和每年9月的第一个星期除外})
    \item 宿舍楼提供免费的100℃开水\footnotemark,若对饮水水质要求不高可直接饮用;否则请前往大学生服务中心〔简称“大服”〕打水处(位置参见\uref{map_fuyanshan_holistic}{此处})付费接取纯净水
          \footnotetext{饮水机每层一个,位于公共洗漱间内;为自来水烧开;熄灯后停止供应。推荐用于泡面、洗衣、泡脚等用途。}
    \item \textbf{\uuline{宿舍单个插座限电400W,超限将导致全楼停电}}
    \item 除跳闸断电等特殊情况外,其余时间段均有4G及5G信号覆盖,延迟波动较大(15~999$^+$㎳),平均网速1.5~5㎆/s
\end{enumerate}

\subsection[住宿注意事项]{住宿注意事项}
\begin{enumerate}
    \item 是否自带被褥等可按照个人需求决定\footnotemark
          \footnotetext{新生军训期间对于床单的颜色等具有特别要求,详见录取通知书相关说明。}
    \item 宿舍门禁系统将在开学军训期间进行人脸录入,\textbf{\uuline{切忌美颜过度,否则无法识别}}
    \item 无特殊情况\textbf{禁止自挂床帘},特殊需求请找导员开具证明
    \item 宿舍内备有烟雾传感器,\textbf{吸烟将引发报警\footnotemark}
          \footnotetext{若烟雾报警器报警,请立即联系宿舍管理人员或保卫处核实。}
    \item 床上桌等类似的东西建议到学校实际生活1个月以后再决定是否购买(大部分人都用不到,少数同学用来打游戏,\sout{然而身体在床上缩着打游戏超级难受};更少一部分同学用其学习,\sout{该现象\linebreak[3]比彩色大熊猫更加罕见})
    \item 宿舍单个插座的功率限制400W\footnotemark,\textbf{\uuline{吹风机、锅、电暖宝、电水壶、热得快等高功率电器均严\linebreak 禁使用},否则将引起宿舍全楼停电};宿管不定期来查,若被发现将被没收并通报批评、检讨
          \footnotetext{如果接了一个插排,插排的总功率不能超过400W,例如一个67W的手机充电器和2个175W的游戏本电脑就立马跳闸了。}
    \item 如需美化宿舍环境,可通过参加学校统一开展的\textbf{“宿舍文化月”}\footnotemark 以对宿舍进行小幅调整
          \footnotetext{禁止外来装修人员入校、禁止私改电路,详情内容见开学后下发的相关要求。}
    \item 背阴面宿舍的阳台作用极小,\textbf{推荐在楼下晾晒衣物},若只靠阴干,很容易发霉发臭;向阳面无此困扰
    \item \textbf{一旦离开宿舍必须关灯关电关水,若插排未拔将被没收并扣分}
    \item 禁止将正在持续充电的手机、充电宝直接置于被子等密闭、空气无流通的环境中,极不推荐直接将笔记本电脑置于被子上使用,严重影响散热且有着火风险
    \item \textbf{\uuline{为保证他人睡眠,熄灯后严禁使用台灯、手电筒在宿舍内继续学习;更不要制造噪音!}}\footnotemark
          \footnotetext{很多同学难以入睡且睡眠浅,小动静或急速的明暗变化就能被惊醒,请大家务必相互尊重、相互理解。}
    \item 如遇宿舍公用物品(如门锁、门轴、玻璃窗、灯管、水龙头、下水道等)损坏,请直接报修(教程\uref{repair_report}{见此})
    \item 宿舍一层有便民驿站(含针线包、打气筒、简易医药箱),即用即还
    \item 雨伞不得长时间放置于宿舍门外,避免影响正常通行
    \item \textbf{严禁在23:30以后使用洗衣机、烘干机、吹风机或洗澡}
\end{enumerate}

