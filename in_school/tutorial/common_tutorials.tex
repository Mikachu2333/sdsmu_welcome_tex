% 教程
\section[常用教程]{常用教程}

\subsection*{特别说明}
\begin{enumerate}
    \item 本节中所有上标“㊕”的网址仅在连接校园网时可成功访问相关服务。
    \item \textbf{严禁使用QQ、微信直接打开本文提及的一切网址,必须使用正常更新的主流浏览器打开!}
\end{enumerate}

\subsection[新生信息查询]{\uuline{新生信息查询}}
\label{freshman_query}
\begin{enumerate}
    \item 关注“山东第二医科大学学生之家”公众号或登录\uhref{https://zhxg.sdsmu.edu.cn}{智慧学工系统}
    \item 点击菜单栏“新生报到”并登录系统(注:账号为身份证号,初始密码为Sddeykdx+身份证号后六位)
    \item →根据其中的相关指引完成预报到流程
    \item →查看宿舍号、学号、班级、院系等相关信息,并根据个人需求选择是否订购军训套装、被褥套装\footnotemark 等
          \footnotetext{如自带则必须按学校有关规定购买尺寸、颜色一致的样式;详请参见\uref{bedding_set}{此处}与录取通知相关材料等说明。}
    \item 智慧学工系统的操作办法同上
\end{enumerate}

\subsection[校园网]{校园网}
\subsubsection[无线连接]{无线连接}
\label{wifi_register}
\begin{enumerate}
    \item 新生账号激活:
          \begin{enumerate}
              \item 到校报到完成后,连接名称为“sdsmu-net”的网络
              \item 在浏览器中打开\uhref{https://slzfw.sdsmu.edu.cn:8800}{https://slzfw.sdsmu.edu.cn:8800$^㊕$}
              \item 点击“用户激活”,账号为身份证号
              \item 按照流程完成设置密码\footnotemark 等一系列步骤即可激活账号
                    \footnotetext{密码需10位以上17位以下,同时含有含数字、大小写字母和特殊字符。}
          \end{enumerate}
    \item 登陆:在浏览器中打开\uhref{https://slrz.sdsmu.edu.cn}{https://slrz.sdsmu.edu.cn$^㊕$}或\uhref{http://210.44.80.65}{http://210.44.80.65$^㊕$}并登陆即可
    \item 忘记密码:在浏览器中打开\uhref{https://slzfw.sdsmu.edu.cn:8800}{https://slzfw.sdsmu.edu.cn:8800$^㊕$},点击右下角“忘记密码”并按提示进行重置,如果因更换手机号或手机号停机等原因导致无法重置,请联系班长或本班网络信息员。
    \item \textbf{充值\footnotemark}:点击\uhref{https://slzfw.sdsmu.edu.cn:8800}{https://slzfw.sdsmu.edu.cn:8800$^㊕$},并登录;或在校园网的登陆页面点击“自助服务”按钮,登陆后即可使用支付宝充值(微信支付暂时无法使用)
          \footnotetext{若点击“自助服务”按钮后网页显示“403 Error”,请将网址前缀“http”改为“https”,按回车即可。如发现无法正确弹出支付宝支付二维码(通常因校园网额度耗尽导致),请换用热点,并刷新该付款二维码页面。}
    \item 当前政策为每月免费60G,超出部分按0.5¥/G收费,最多额外购买60G流量
    \item \textbf{警告:}一个账号最多允许3台设备同时在线
\end{enumerate}

\subsubsection[有线连接]{有线连接}
\begin{enumerate}
    \item 按照\uref{wifi_register}{此处}的教程激活校园网
    \item 将网线插入宿舍内的“H3C”盒子的底部右侧的接口内并正确连接到电脑\footnotemark
          \footnotetext{若始终无法连接,应检查网线的内部线排列顺序,从左到右应为“白橙橙,白绿蓝,白蓝绿,白棕棕”。}
          \begin{enumerate}
              \item 首次连接:
                    \begin{enumerate}
                        \item 打开“设置”→“网络和Internet”→“拨号”→“设置新连接”
                        \item →“连接到Internet”→点击“否,创建新连接(C)”→选择“宽带(PPPoE)(R)”
                        \item →输入自己的校园网帐号以及密码→勾选“记住密码”
                    \end{enumerate}
              \item 再次连接:
                    \begin{enumerate}
                        \item 打开“设置”→“网络和Internet”→“拨号”
                        \item →选中之前设置的网络→连接即可
                    \end{enumerate}
          \end{enumerate}
    \item  注:上述均为Win10/Win11教程,Mac教程暂无。
\end{enumerate}

\subsection[校园手机卡(校园卡)]{校园手机卡(校园卡)}
\begin{enumerate}
    \item \textbf{开通与否全凭自愿,是否开通校园卡不影响校园网的使用。}随录取通知书一并寄出。如遇强制开通可告知带班学长或自行反馈
    \item 套餐\footnotemark 通常内含至少100分钟全国通话、80G校园流量(仅山东省内所有高校可用)
          \footnotetext{详情优惠政策可咨询营业厅,如追求更多流量建议学校各账号不绑定校园手机卡,每年根据新优惠政策调整手机卡(需及时注意注销旧手机卡以防欠费导致的信用记录问题)。}
    \item 开学报到当天可前往大服,免费领取礼品
    \item 欠费销户提醒:大部分均需要主动销户,公告明确支持欠费销户的仅有移动的部分套餐。不需要继续使用手机卡请及时销户,如长期欠费可能影响以后办理手机卡号;欠费极其严重的甚至可能被列入征信黑名单中!
    \item \textbf{\uuline{出租手机卡可能造成违法犯罪!!!}}
\end{enumerate}

\subsection[高校edu教育邮箱]{高校edu教育邮箱}
\label{email}
\begin{enumerate}
    \item 学校可免费开通以“@sdsmu.edu.cn”结尾的教育邮箱
    \item 具体开通方式与使用说明详见学校官网公告栏
    \item 附2023学年下学期通知以供参考:\uhref{https://www.sdsmu.edu.cn/2024/0308/c14a128552/page.htm}{《关于为在校生开通校内邮箱的通知》}
\end{enumerate}

\subsection[空调使用教程]{空调使用教程}
\label{air_control}
\begin{enumerate}
    \item 微信关注“海享租”公众号,点击公众号菜单“在线租赁”,并注册、登录
    \item 点击“扫一扫”→扫描空调右下角二维码进行租赁\footnotemark
          \footnotetext{若提示租赁失败,请按照软件提示联系同宿舍的学长/学姐退租,也可咨询学长学姐或向宿管反馈。}
    \item →租赁完成后,点击“设备”→“空调图标”→“时长”,进行充值
    \item →点击“设备”→“空调图标”→“成员管理”,在此页面下将宿舍全部成员权限设置为均可管理空调开关即可
    \item \textbf{注意:}空调使用时长收费(0.55元/小时),具体收费及租赁政策详见“海享租”公众号
    \item \textbf{警告:}请各位同学在搬离校区或毕业前\textbf{退租空调},否则将导致下一届学弟学妹无法租赁!
\end{enumerate}

\subsection[浴室预约与使用]{浴室预约与使用}
\subsubsection[浮烟山校区]{浮烟山校区}
\label{shower_software_f}
\begin{enumerate}
    \item 软件基础设置:
          \begin{enumerate}
              \item 在手机应用市场下载“大白U帮”app
              \item 按照实际住宿情况注册
              \item 授予并开启“定位”与“蓝牙”权限
          \end{enumerate}
    \item 本楼层小浴室使用:
          \begin{enumerate}
              \item 带好洗浴物品前往公共厕所旁边的浴室排队
              \item 进入浴室,点击如右图所示的按钮(\noindent\mbox{\includegraphics[height=2.4ex]{resources/sundry/bath.pdf}})→选择“蓝牙设备”→“点击进行时”
              \item →“洗澡”→“搜索洗澡”\footnotemark
                    \footnotetext{搜索不到设备请务必开启“蓝牙”功能,学校的设备无法扫码连接。}
              \item →选择设备\footnotemark →“开始洗澡”
                    \footnotetext{距离厕所入口最近的是1号,远的是2号;不确定可以询问学长。}
              \item 结束后点击“结束洗澡”按钮,并结算
          \end{enumerate}
    \item 一层公共大浴室预约:
          \begin{enumerate}
              \item 在软件初始界面根据实际情况选择“X号楼1层”的浴室
              \item →点击一个浴位,并点击“预约”按钮(若已满请选择“排队”)
              \item →在8分钟内前往浴室,并点击“开始洗浴”
              \item →结束后点击“结束洗澡”按钮,并结算
          \end{enumerate}
    \item 费用:以程序显示为准,详情收费标准略
    \item 申诉:如果在洗澡时突然停电导致无法结束洗澡而被扣费,请按照软件打开时弹出的公告,联系相关工作人员处理
\end{enumerate}
\subsubsection[虞河校区]{虞河校区}
\label{shower_software_y}
\begin{enumerate}
    \item 支付宝搜索“住理生活”小程序,并按照提示开通账户(建议支付宝绑定银行卡)
    \item →搜索“潍坊医学院虞河校区”,并绑定账号到自己性别的浴室
    \item →扫描设备二维码即可使用淋浴
    \item 也可在小程序主页面点击“洗浴”→“切换设备”→手动选择设备进行洗浴
    \item \textbf{注意:浴室营业时间为10:00~21:45,22:00停水,浴室位置较少,请错峰洗澡!}
\end{enumerate}
\subsubsection*{注意事项}
\begin{enumerate}
    \item 如果未点击“结束洗澡”按钮便直接离开可能会被多扣费
    \item 严令禁止在浴室内大便!!!
\end{enumerate}

\subsection[洗衣机/洗鞋机使用教程]{洗衣机/洗鞋机使用教程}
\subsubsection[浮烟山校区]{浮烟山校区}
\label{washing_machine_f}
\begin{enumerate}
    \item 在微信小程序搜索“海乐生活”,并注册、开启相机与定位权限
    \item 预约方法(也可直接使用):
          \begin{enumerate}
              \item 在小程序内点击“附近营业点”→找到“潍坊医学院X号楼”
              \item →选择相应的楼层→选中洗衣机并下单→等待上次洗衣结束
              \item →前往洗衣机→输入验证码→放入衣物与洗衣粉/洗衣液并缴费
          \end{enumerate}
    \item 扫码直接使用(不可预约):
          \begin{enumerate}
              \item 前往洗衣机→在小程序内点击“扫码使用”按钮
              \item →扫描洗衣机上的二维码→选择并下单
              \item →放入衣物与洗衣粉/洗衣液,输入验证码后缴费即可
          \end{enumerate}
    \item 收费标准详见软件说明
    \item 洗衣机错误处理办法(若无相关经验切忌自行操作):
          \begin{enumerate}
              \item 拨打洗衣机旁边的报修电话;
              \item E1:洗衣机断电后开门,打开洗衣机右下角小门,旋开阀门,使用镊子等工具伸入并清除其中堵塞管道的杂物,恢复原样即可;
              \item E4:旋开洗衣机后方的水管阀门即可。
          \end{enumerate}
\end{enumerate}
\subsubsection[虞河校区]{虞河校区}
\label{washing_machine_y}
\begin{enumerate}
    \item 微信小程序搜索“智慧笑联”→按照提示注册并授予定位权限
    \item →绑定“潍坊医学院(虞河校区)”→绑定至自己所在的宿舍楼
    \item →再次扫描洗衣机上的二维码,按提示操作并付款即可
    \item 收费标准详见软件说明
\end{enumerate}

\subsubsection[禁止事项]{禁止事项}
\begin{enumerate}
    \item 禁止向第二格内倾倒洗衣粉、洗衣液,\textbf{第二格是放柔顺剂的}!
    \item \textbf{禁止使用洗衣机洗鞋},请用旁边的洗鞋机!
    \item 禁止在洗衣机上堆放杂物
    \item \textbf{\uuline{禁止将袜子、内衣内裤等贴身衣物机洗!}}
\end{enumerate}

\subsection[烘干机使用教程]{烘干机使用教程}
\label{dry_machine}
\begin{enumerate}
    \item 注册等步骤详见\uref{washing_machine_f}{此处}
    \item 按照预约的方法,选择宿舍楼后,选择“烘干机”即可,其他步骤与洗衣相似
    \item 推荐烘干配置
          \begin{enumerate}
              \item 高温60分钟:大部分轻薄的衣物(例如T恤、卫衣、浴巾等)
              \item 高温120分钟:薄被(如夏凉被)
          \end{enumerate}
    \item \textbf{注意:}\textbf{使用前后务必控干水箱并清理滤网。}棉被、羽绒服等禁止使用烘干机烘干以免损坏及不必要的危险情况发生。
    \item 收费标准详见软件提示
\end{enumerate}

\subsection[吹风机使用教程]{吹风机使用教程}
\label{hair_drier}
\begin{enumerate}
    \item 每层公共浴室旁边有两个公用吹风机,需扫码\footnotemark 租赁使用
          \footnotetext{部分吹风机屏幕二维码可能有缺损,不易扫描成功,多次尝试即可。(推荐使用浏览器扫码并复制到微信内收藏该网址,下次直接在微信内点击即可使用。)}
    \item →待手机发出“滴--滴”的声音后租赁成功
    \item →将手机扬声器对准吹风机租赁器方可正常使用
    \item 收费标准:详见软件提示,1分钱起步
\end{enumerate}

\subsection[空闲教室查询]{空闲教室查询}
\label{spare_classroom}
\begin{enumerate}
    \item 根据\uref{cas_system}{下文}的CAS认证系统教程,登录山二医app(下载链接见\uref{sdsmu_app}{此处})
    \item 点击下方菜单栏“应用”→空闲教室查询
    \item 点击右上角“\ 〉”按钮→根据自己的需求进行筛选
    \item \textbf{注意:}在临近期末考试时,因考试教室占用等原因查询结果可能不准确。
\end{enumerate}

\subsection[图书馆座位预约教程]{图书馆座位预约教程}
\label{library_book}
\begin{enumerate}
    \item 微信小程序搜索“青栀校园”→微信注册登录并绑定学号→允许小程序通知
    \item →在小程序内点击“座位预约”或扫描图书馆座位上的二维码即可
    \item \textbf{座位暂离的注意事项:}
          \begin{enumerate}
              \item 如因各种原因需要长时间离开的,请在小程序上选择“暂时离开”,否则按违规处理
              \item 离馆时也需要在小程序内确认
              \item 如发现已预约的座位被他人占据,请扫描桌面上的二维码并在小程序内举报,工作人员将尽快处理
          \end{enumerate}
    \item \textbf{违规说明:}
          \begin{enumerate}
              \item 已预约而未按时到位的记一次违规
              \item 未选择暂离而离开座位被举报的记一次违规
              \item 停止使用后未选择退馆的记一次违规
              \item \textbf{三次违规后将取消座位预约资格三天}
          \end{enumerate}
    \item 其他禁止事项
          \begin{enumerate}
              \item 禁止占用其他人已经预约的座位
              \item 禁止在图书馆内喧哗、吸烟
              \item 禁止在图书馆的非背诵区域内背诵、朗诵、频繁交流
              \item 禁止在图书馆内谈恋爱、亲嘴
          \end{enumerate}
\end{enumerate}

\subsection[设施报修方式枚举]{设施报修方式枚举}
\label{repair_report}
\begin{enumerate}
    \item 宿舍维修(浮烟山校区):
          \begin{enumerate}
              \item 加入各宿舍楼的QQ报修群,在群内反映具体故障
              \item 前往一层宿管处填表报修
              \item 在宿舍一层宿管旁边的公告栏处查看相关负责人的电话,直接拨打即可
              \item 拨打学生公寓管理中心电话反馈
              \item 询问带班学长、学姐
          \end{enumerate}
    \item 教室维修(浮烟山校区):
          \begin{enumerate}
              \item 拨打后勤管理处的电话报修
              \item 拨打教室管理中心的电话报修(仅限多媒体及饮水机)
              \item 拨打物业电话报修
              \item 在“诉求留言”微信小程序内反馈
          \end{enumerate}
    \item 宿舍维修(虞河校区):在宿管处填写报修单据并送往物业维修中心(位置见\uref{common_locations_yuhe}{此处})
    \item 教室维修(虞河校区):拨打教室内张贴的报修电话
\end{enumerate}

\subsection[缓考申请教程]{缓考申请教程\footnotemark}
\footnotetext{各学院要求不一:临床医学院大部分课程可以只填写钉钉、教务系统,仅少数要求同时提交纸质表格;其他学院以教师要求为准。}
\begin{enumerate}
    \item 填写钉钉\footnotemark(详细步骤如下)
          \footnotetext{需在学校统一将大家拉入钉钉的“山东第二医科大学”企业后方可使用。}
          \begin{enumerate}
              \item 打开钉钉→点击左上角选择主企业为“山东第二医科大学”
              \item →点击页面最下方菜单栏“工作台”→点击“OA审批”
              \item →在“学风建设”类选择“学生缓考审批表”(或直接搜索“缓考”)
          \end{enumerate}
    \item 在教务处下载《\uhref{https://jwch.sdsmu.edu.cn/_upload/article/files/f7/d0/c172c4f74eecba307f700cde1a21/99599310-0254-48be-adc5-fcafa99e7341.doc}{山东第二医科大学学生缓考审批表}》,打印3份并按照要求填写完毕
    \item 填写教务系统(详细步骤如下)
          \begin{enumerate}
              \item 进入教务系统(仅校园网,详情参见\uref{academic_affairs_system}{此处})
              \item 点击左侧菜单“考试报名”→“我的申请”→“缓考申请”
              \item →选择“学年学期”和“活动名称”后,直接点击“搜索”(不要填写科目名称)
              \item →在弹出的菜单中选择缓考科目并填写申请\footnotemark
                    \footnotetext{若因病缓考体测,需要上传病历本等相关材料,并前往校医院开具证明,再前往学工办向教师当面说明情况。}
          \end{enumerate}
    \item 前往学工办交表并等待审批
\end{enumerate}

\subsection[教学楼多媒体教室/乐道济世书院第二课堂申请流程]{教学楼多媒体教室/乐道济世书院第二课堂申请流程}
\begin{enumerate}
    \item 提前做好《活动策划案》、《活动安全应急预案》
    \item 请指导老师(通常为班主任或学工办老师)审核上述文件并确认钉钉具体审批负责人
    \item 打开钉钉→点击左上角选择主企业为“山东第二医科大学”
    \item →点击页面最下方菜单栏“工作台”→点击“OA审批”
    \item →在“校园文化活动与社会实践”类选择对应申请表并填写(或直接搜索“教室”/“书院”)
    \item 前往教室E区2层(见\uref{map_fuyanshan_teach_building}{此处})靠近A区处的“教室管理中心”签字盖章
    \item 前往预约的教室,拨打讲台上教室管理员的电话提前沟通说明
\end{enumerate}

\subsection[学生会校级格式]{学生会校级格式}
\begin{enumerate}
    \item 用途:校级格式是用于校内各类\textbf{正式文稿}(如活动通知、综测条例、学生会文件等)的标准
    \item 具体要求:
          \begin{enumerate}
              \item 纸张大小:A4纸
              \item 装订:页面左侧上下各 $\frac{1}{4}$ 处,距左边界0.3~0.5㎝处,钉与纸张左边界平行
              \item 页边距:上下左右均为2.5㎝
              \item 页码:两页及以上的材料,在页面底端居中插入
              \item 行间距:固定值,26磅
              \item 标点:均为中文标点,除特殊情况\footnotemark 外不应使用英文标点
                    \footnotetext{特殊情况举例:时间(22:05),英文活动(So, let us practice English!)等。}
              \item 文章标题格式:方正小标宋简体,小二号,居中
              \item 标题与正文之间空一行
              \item 正文格式:仿宋-GB2312,三号,两端对齐,首行缩进2字符
              \item 正文一级标题格式:“\textbf{一、}”;黑体,三号,居左,首行缩进2字符
              \item 正文二级标题格式:“\textbf{(一)、}”;楷体-GB2312,三号,居左,首行缩进2字符
              \item 正文三级标题格式:“\textbf{1、}”;其他要求与正文相同
              \item 正文四级标题格式:“\textbf{(1)}”;其他要求与正文相同
              \item 正文与落款之间空2行
              \item 落款机构(个人姓名)格式:仿宋-GB2312,三号,右对齐
              \item 落款时间格式:仿宋-GB2312,三号,右对齐,时间格式范例:2024年02月10日
          \end{enumerate}
\end{enumerate}

\subsection[小组汇报PPT制作指南(初级)]{小组汇报PPT制作指南(初级)}
\begin{enumerate}
    \item 整体要求
          \begin{enumerate}
              \item 选择软件:请使用大多数人使用的微软\ Office办公软件,或者金山WPS最新版;并在电脑上进行编辑,手机当且仅当用于ppt的简单查看。如无特殊情况不要使用LibreOffice、OpenOffice、腾讯文档等办公套件以免不兼容。
              \item \textbf{明确比例}:PPT(也称Slide)有4:3与16:9两种主流比例\footnotemark
                    \footnotetext{注:4:3比例像正方形,16:9是明显的长方形;下述的各类参数(如:24/32号字体)均以4:3、16:9的顺序进行。}
              \item 首页规范:首页应含有所有的必要信息(详情见下文)
              \item 目录规范:PPT应当在第二页含有一个简洁明了的目录
              \item 内容规范:\textbf{PPT内仅应含有所讲内容的关键部分}而非一昧照抄原文
              \item 字体与段落规范:\textbf{PPT字体不应过小,间距不应过密}(详情建议见下文);禁止使用文字阴影;\textbf{特殊字体必须内嵌于PPT}(例如艺术字、书法体等),禁止现场在演示的电脑上下载补全需要的字体(操作步骤可搜索“在PPT内嵌入字体”)
              \item 图片规范:每张图片不应大于10M\footnotemark,如发现图片模糊,需手动检查PPT设置,将PPT更改为“禁止自动压缩图片”,操作步骤请自行搜索
                    \footnotetext{如果图片过大,可以使用\uhref{https://gitee.com/LinkChou/rimage_gui/releases/latest}{Rimage\_GUI}适当缩小图片体积(软件开源,如被报毒请自行分辨)。}
              \item 动画规范:\textbf{PPT不应有过多动画以及元素堆叠}(例如,绝对禁止PPT中的一页内含15张大图片,依靠动画一张张切换,以免软件突然崩溃)
              \item 配色规范:请\textbf{使用经典的“文字—背景”配色}(\sout{虽然确实难看}),例如“黑—白”,“红—白”,\linebreak[3]“白—黑”,“蓝—白”等,切勿使用“橙—白”、“红—蓝”等投影后效果一塌糊涂的配色\footnotemark
                    \footnotetext{请注意ppt配色以使色觉感知局限的同学能顺畅接受信息。}
              \item 文件命名规范:要求PPT文件名\textbf{简单易懂,包含所有必须信息}
              \item 文件保存规范\footnotemark:\textbf{必须同时以“.pptx”后缀与“.ppt”后缀各保存一份}以免部分电脑无法正常打开,\textbf{禁止保存为“.dps”、“.odp”等特殊格式},详情见下
                    \footnotetext{\textbf{若无法看到文件名称后缀,请搜索“电脑设置显示文件扩展名”。}}
              \item 学校校徽及图标等标识使用规范:详情见\uhref{https://www.sdsmu.edu.cn/4229/list.htm}{《山东第二医科大学VIS视觉识别系统手册》}(由校宣传部印发)
          \end{enumerate}
    \item 详细要求:
          \begin{enumerate}
              \item 首页:需注明小组成员姓名及学号,日期,课程名称,授课教师等必要信息并\textbf{仔细检查是否有误},标题的字体不应小于80/75号,小组成员及其他信息字体不应小于25/25号
              \item 目录页:简短凝练,总字数不应超过35个字;字体不应小于80/55号
              \item 内容页:字体\footnotemark 不应小于55/40号,上下左右应各留出 $\frac{1}{10}$ 左右的间距;如内容过多应自行分页,\textbf{严禁为节省页数而缩小字号};PPT显示的内容与口述补充的内容在6:4或7:3左右最佳,各类关键数据的引用(例如学术数据、课标外的公式定义等)应当按照\linebreak[3]《\textbf{GB/T}\ 7714—2015 信息与文献 参考文献著录规则》的相关标准\textbf{标明出处}
                    \footnotetext{下面的字体皆以微软雅黑为标准,楷体、仿宋等纤细字体请自行增加字号。}
              \item 文件命名\footnotemark:推荐使用“\textbf{20XX级临床X班X组关于XXX的汇报(终稿).pptx}”此类命名,\textbf{严禁使用默认名称以防混淆}(例如“新建 Microsoft PowerPoint 演示文稿.pptx”)。此外,在ppt未定稿时,推荐使用一些默认的规范进行命名(例如“\textbf{关于XXX的汇报草稿-4.pptx}”),以方便小组成员确定PPT版本,而非使用默认的“新建 Microsoft PowerPoint 演示文稿(1)(2)(5).pptx”这种高血压命名。
                    \footnotetext{文件名称后缀的“.pptx”、“.ppt”等各种类型并非手动添加,而是文件自带,\textbf{严禁手动修改后缀名!}}
              \item 文件保存\footnotemark:如需在“.ppt”“.pptx”两种格式间相互转换请使用PowerPoint或WPS等办公软件,“.dps”格式必须使用WPS才能转换为“.ppt”或“.pptx”格式。
                    \footnotetext{\textbf{转换方法}:在左上角的“文件”菜单选择“另存为”,在下拉框中选择“.pptx”并保存。}
          \end{enumerate}

\end{enumerate}

\subsection[钉钉请假流程]{钉钉请假\footnotemark 流程}
\label{leave_dingtalk}
\footnotetext{需在学校统一将大家拉入钉钉的“山东第二医科大学”企业后方可使用。\textbf{各学院要求不一,仅以临床医学院为例。}}
\begin{enumerate}
    \item 线下请假步骤(正常情况):
          \begin{enumerate}
              \item 前往学工办或班主任办公室,当面请假并获得假条
              \item →根据老师要求扫描相关二维码→钉钉填表
              \item →刷脸进出校门,并将请假条之一交给保卫处
              \item →返校后,在钉钉的电子假条处,以评论的方式销假
          \end{enumerate}
    \item 线上请假步骤:
          \begin{enumerate}
              \item 打开钉钉→点击左上角选择主企业为“山东第二医科大学”
              \item →点击页面最下方菜单栏“工作台”→点击“OA审批”
              \item →在“学生日常事务管理”类选择“浮烟山校区本科学生请假单”→填表
              \item →电话联系班主任老师或学工办老师,说明请假事由并等待审批
              \item →审批通过后刷脸进出校门
              \item →返校后,在钉钉的电子假条处,自觉以评论的方式销假
          \end{enumerate}
          \textbf{注意:}请大家自觉销假,切忌一再拖延。
\end{enumerate}

\subsection[家长、校友进校参观指南]{家长、校友进校参观指南\footnotemark}
\footnotetext{注:本节仅适用浮烟山校区;虞河校区为开放校区,除宿舍、教学楼外,均可参观。}
\begin{enumerate}
    \item 学校在每年8月30日前后允许家长参观校园\footnotemark,具体政策以学校官方说明为准
          \footnotetext{允许社会车辆在参观时段在校园内划定的区域内停放,(通常)允许家长在规定时间内参观宿舍环境,具体政策每年不同,具体情况以学校官方说明为准。}
    \item 其他校友若有返校需求可凭本人毕业证、学生证等有效证件(电子版也可),或通过当年所在院系的老师联系保卫处,在校门口登记后即可入校参观
\end{enumerate}

\subsection[学费缴费教程]{学费缴费教程}
\label{fee_pay}
\begin{enumerate}
    \item 官方:微信公众号“山东第二医科大学财务处”或“\uhref{https://tyzfpt.sdsmu.edu.cn/xysf/login.aspx}{山东第二医科大学校园统一缴费平台}”
    \item 用途:学费缴纳、卡号绑定等
    \item 注意:缴费系统仅在部分时间段开放,请按学校通知按时缴费
    \item 学费缴纳教程:
          \begin{enumerate}
              \item 前往公众号菜单,点击右下角“缴费管理”→“支付平台”
              \item →登录系统(新生的账号为高考考生号,密码格式同老生;老生的帐号为学号,初始密码为姓首字母大写加身份证后六位)
              \item →按照提示修改初始密码(请务必牢记)→进行缴费
          \end{enumerate}
    \item 银行卡绑定教程:
          \begin{enumerate}
              \item 目的:学校仅在初次使用时收集一次卡号并存储数据,以便下次直接使用\footnotemark
                    \footnotetext{详情见学校官方说明。}
              \item 打开“山东第二医科大学财务处”公众号,点击“财务中心”
              \item →使用帐号密码登录并绑定微信号(帐号为学号,初始密码为000000)
              \item →点击“卡号维护”→“管理”
              \item →按照提示填写相关信息,确认信息无误后提交即可
          \end{enumerate}
\end{enumerate}

\subsection[学工系统(微信小程序)]{学工系统(微信小程序)}
\begin{enumerate}
    \item 用途:学工系统主要用于晚点名、返校信息填报等日常工作\footnotemark
          \footnotetext{原“请假审批”、“外出审批”工作已基本转移至“钉钉”,教程参见\uref{leave_dingtalk}{此处}。}
    \item 使用方式:
          \begin{enumerate}
              \item 打开“定位权限”并允许微信使用→在微信搜索“智慧学工”小程序
              \item 根据学校下发的账号密码进行登录(推荐立即与微信绑定以免忘记密码)
          \end{enumerate}
\end{enumerate}

\subsection[教务系统]{\textbf{\uuline{教务系统}}}
\label{academic_affairs_system}
\begin{enumerate}
    \item 官网:\uhref{https://jwgl.sdsmu.edu.cn}{https://jwgl.sdsmu.edu.cn$^㊕$}
    \item 用途:\textbf{选课,缓考申请,成绩查询},查看(导出)课程表,空闲教室查询
    \item \textbf{注意:}仅限校内访问,如需在外使用教务系统,参见\uref{cas_system}{此处}条目
\end{enumerate}

\subsection[CAS资源访问控制系统(校内VPN)]{\textbf{\uuline{CAS资源访问控制系统(校内VPN)}}\footnotemark}
\footnotetext{在校内时可通过校园网直接访问相应系统,无需使用本系统中转。}
\label{cas_system}
\begin{enumerate}
    \item 官网:\uhref{https://webvpn.sdsmu.edu.cn}{https://webvpn.sdsmu.edu.cn}
    \item 说明:(也称CAS认证系统,智慧校园系统)本系统主要用于\textbf{在校外访问校内网络信息资源},如:教务系统(查成绩、课表)、知网、临床医学虚拟仿真实验中心\footnotemark 等
          \footnotetext{查阅文献推荐使用CARSI系统,速度快效果好,教程详见\uref{carsi_system}{此处}。}
    \item 异地登录教务系统教程(其它系统同理):
          \begin{enumerate}
              \item 打开网站,点击“统一身份认证登录”,使用学号+密码登录或手机号验证登陆、扫描登陆等方式均可(推荐绑定微信,初始密码为sdsmu@身份证后六位)
              \item →找到应用中心→“教务系统—非单点登录”\footnotemark →使用教务系统账号密码登录即可
                    \footnotetext{请注意,在校外时点击“教务系统”无法登录,只有“非单点”能校外登录!}
          \end{enumerate}
\end{enumerate}

\subsection[CARSI系统]{\textbf{\uuline{CARSI系统}}}
\label{carsi_system}
\begin{enumerate}
    \item 官网:\uhref{https://ds.carsi.edu.cn}{https://ds.carsi.edu.cn}
    \item 说明:(与CAS认证系统作用不同)用于快速访问学校订阅的各类数据库,如百度文库、知网、万方、维普等
    \item 使用教程:
          \begin{enumerate}
              \item 进入官网→搜索“山东第二医科大学”并勾选“记住我的选择”→点击后进入登陆界面\footnotemark
                    \footnotetext{请注意,不要收藏登录界面的网址!每次登录网址都不一样,只能从官网重新进入!}
              \item →使用\textbf{CAS认证系统}的账号密码登录系统,出现各类弹窗一律选择“Accept”即可\footnotemark
                    \footnotetext{仅推荐在自己的电脑上如此设置,如必须在网吧等公共场所的电脑上使用,请审慎阅读相关提示,并谨慎进行登录,如因账号泄露造成损失,一切责任自负。}
              \item →登录完成后,点击任意资源链接即可进入相应网站并获取论文
          \end{enumerate}
\end{enumerate}

\subsection[校园一卡通系统]{校园一卡通系统}
\label{union_card}
\begin{enumerate}
    \item 用途:校园支付(如:浮烟山校区杏林餐厅、虞河校区乐道餐厅\footnotemark)、图书馆(借阅)
          \footnotetext{餐厅完全支持支付宝支付与微信支付,普通学生使用此付款码频率较低。}
    \item 另:付款码负责餐厅消费,身份码负责图书馆借阅
    \item 注:初始密码为身份证后六位,X替换为0
\end{enumerate}

\subsection[校务行(微信小程序)]{校务行(微信小程序)}
\label{cert_prover}
\begin{enumerate}
    \item 官网:微信小程序
    \item 用途:查成绩,下载学籍证明、成绩证明的pdf版本
    \item 费用:以程序显示为准
    \item 教程:
          \begin{enumerate}
              \item 搜索小程序“校务行”
              \item →点击右上角“点击登录”(帐号为学号,密码为身份证后六位)\footnotemark
                    \footnotetext{新生可能无法在开学后即刻使用本程序,需等待学校将信息录入完毕。}
              \item →按需选择“电子成绩单”或“电子证明”→按照小程序提示进行即可
          \end{enumerate}
\end{enumerate}

\subsection[档案查询]{档案查询}
\subsubsection[档案远程服务利用系统(学校)]{档案远程服务利用系统(学校)}
\begin{enumerate}
    \item 官网:\uhref{https://dangan.sdsmu.edu.cn/service-utilization/web/management/index}{https://dangan.sdsmu.edu.cn/service-utilization/web/management/index}
    \item 微信公众号入口(与官网作用相同):“山东第二医科大学”公众号→微服务→档案查询
    \item 业务范围:
          \begin{enumerate}
              \item 招生录取名册(学生登记表)
              \item 学生成绩
              \item 其他学籍档案
              \item 查档预约、查档咨询
          \end{enumerate}
    \item 其他注意事项详见官网
\end{enumerate}
\subsubsection[高校档案查询利用平台]{高校档案查询利用平台}
\begin{enumerate}
    \item 官网:\uhref{http://gxda.dag.shandong.gov.cn:81/index}{http://gxda.dag.shandong.gov.cn:81/index}
    \item 业务范围:同“档案远程服务利用系统”
\end{enumerate}

\subsection[心理健康教育中心预约教程]{心理健康教育中心预约教程}
\begin{enumerate}
    \item 作用:纾解心理压力、提供心理咨询等
    \item 工作时间(周一至周五):08:00~11:30、14:00~17:30
    \item 提示:预约后1个工作日左右,咨询中心将回电联系并协商心理咨询时间,请注意接听电话
    \item 线下预约:至浮烟山校区E206填表后等待咨询中心回电
    \item 电话预约:拨打0536-8462130
    \item 线上预约:填写\uhref{https://www.wjx.cn/vm/YOHd59S.aspx}{心理咨询信息登记表}后等待回电

\end{enumerate}

\subsection[常见证明申请]{常见证明申请}
\begin{enumerate}
    \item 说明:适用于学籍证明\footnotemark、成绩证明、个人档案
          \footnotetext{在考试时,学生证与学籍证明拥有相同效力,丢失学生证可使用学籍证明代替。}
    \item 线下打印
          \begin{enumerate}
              \item 携带身份证、手机前往D区的打印机,详情位置\uref{common_locations_fuyanshan}{见此}
              \item 按照机器的说明填写相关信息并打印(无纸时请联系工作人员)
          \end{enumerate}
    \item 线上下载
          \begin{enumerate}
              \item 学籍证明:请按\uref{cert_prover}{校务行教程}申请下载至邮箱中
              \item 个人档案:按照\uref{cert_prover}{此处}的教程申请并下载,自行前往打印店打印即可(请注意及时删除打印完毕的文件以免泄漏隐私)
          \end{enumerate}
\end{enumerate}

\subsection[监控录像调取等申请]{监控录像调取等申请}
\label{sdsmu_app}
\begin{enumerate}
    \item 用途:调取监控录像、申请课程录制等
    \item 按照 \uref{cas_system}{此处}的说明登录“山东第二医科大学App”
    \item 点击下方菜单栏“应用”→“业务申请”
    \item 按需选择“保卫处调阅监控录像审批流程”等申请
\end{enumerate}

%\newpage % 必须使用newpage而非pagebreak,因为pagebreakv会使下方的表格尝试排版到上一页导致underfull vbox 警告
%尽量手动控制此节的排版……
\subsection[公交信息与免费乘车指南]{公交信息与免费乘车指南}
\label{free_bus}
运营时间通常为6:30~19:00

免费线路\footnotemark:9、13、19、29、63、69、71、101、109、166、167
\footnotetext{据《滨海及浮烟山两地大学城在校学生免费乘坐校区至中心城区公交车实施方案》(潍交城〔2024〕5号),免费公交政策暂行时间为:2024.04.29~2025.04.28。在学校提交相关信息后,可通过\textbf{与提交信息相符}的手机号注册认证“潍坊公交大学生乘车码”小程序免费乘车,新生可持学生证或录取通知书临时代替。}

\begin{tblr}[
        long,
        caption = {常用站点名称对应关系一览表},
    ]{
        cells = {c,m},
        rowhead = 1,
        row{1} = {cmd=\bfseries},
        vlines,
        hlines,
        cell{2,8,30,32}{1} = {r=2}{},
        cell{10,20,23}{1} = {r=3}{},
        cell{4,13}{1} = {r=4}{},
        cell{28,31}{3} = {r=2}{},
        hline{1-2,8,13,17,Z} = {-}{1pt},
    }
    %尽量手动控制,尽量避免因自动断页导致的underfull vbox警告
    常见目的地       & 具体方位     & 站点名                   \\
    浮烟山校区       & 南门         & 山东第二医科大学南门     \\*
                     & 北门         & 山东第二医科大学北门     \\\nopagebreak[2]
    虞河校区         & 北门         & 胜利街虞河路路口西       \\*
                     & 东北侧       & 虞河路胜利街路口北       \\*
                     & 东门         & 虞河路胜利街路口南       \\*
                     & 西侧         & 鸢飞路胜利街路口北       \\
    附属医院         & 西门         & 附属医院                 \\*
                     & 南侧         & 福寿街虞河路路口东       \\
    人民医院         & 本部东门     & 人民医院东门             \\*
                     & 本部西侧     & 人民医院                 \\*
                     & 北辰院区南门 & 北辰院区                 \\
    火车站           & 北广场       & 火车站                   \\*
                     & 东侧         & 青年路铁路桥北           \\*
                     & 东北侧       & 火车站                   \\*
                     & 潍坊北站     & 高铁北站                 \\
    人民公园         & 西门         & 人民公园西门             \\
    市中医院         & 东门         & 市中医院                 \\
    妇幼保健院       & 东门         & 潍坊市妇幼保健院潍城院区 \\
    泰华城           & 北侧         & 泰华城                   \\*
                     & 西南侧       & 青年路胜利街路口南       \\*
                     & 南侧         & 风筝广场                 \\
    万达广场         & 西侧         & 鸢飞路福寿街路口南       \\*
                     & 南侧         & 潍坊中学                 \\*
                     & 北侧         & 福寿街鸢飞路路口东       \\
    谷德广场         & 南侧         & 谷德广场                 \\
    奎文防疫站       & 无           & 奎文防疫站               \\
    鲁台会展中心     & 北门         & 鲁台会展中心             \\*
    蓝海大饭店       & 南门         &                          \\*
    谷德锦           & 南侧         & 北宫街清平路路口西       \\*
                     & 东侧         & 潍坊交通职业中等专业学校 \\*
    奥林匹克体育公园 & 东侧         &                          \\*
                     & 南侧         & 奥体中心                 \\
    十笏园           & 西侧         & 十笏园
\end{tblr}

\begin{tblr}[
        long,
        caption = {常用路线汇总表},
        note{1} = {加粗线路为上述免费乘车线路。},
    ]{
        rowhead = 1,
        cells = {c,m},
        row{1} = {cmd=\bfseries},
        cell{1}{2} = {c=3}{},
        cell{2,7,9,11,15,20,26}{1} = {r=2}{},
        cell{4}{1} = {r=3}{},
        cell{4,18,19,22,24,28}{1} = {}{cmd=\bfseries},
        vlines,
        hlines,
        hline{1-2,Z} = {-}{1pt},
    }
    路线\TblrNote{1} & 常见目的地         &                          &                  \\*
    5                & 火车站(北广场)   & 市中医院(东门)         & 风筝广场         \\*
                     & 万达广场(西侧)   &                          &                  \\
    13               & 浮烟山校区(南门) & 虞河校区(北门/东北侧)  & 火车站(东侧)   \\*
                     & 附院(西门)       & 泰华城(南侧)           & 妇幼保健院       \\*
                     & 奎文防疫站         & 人民公园                 & 风筝广场         \\
    22               & 泰华城(北侧)     & 十笏园                   & 谷德锦           \\*
                     & 奥林匹克公园       &                          &                  \\
    23               & 火车站(北广场)   & 附院(南侧)             & 十笏园           \\*
                     & 万达广场(北侧)   &                          &                  \\
    25               & 虞河校区(东北侧) & 附院(西门)             & 火车站(东北侧) \\*
                     & 火车站(潍坊北站) & 奎文防疫站               & 妇幼保健院       \\
    D28              & 人民医院(东门)   & 人民医院(北辰院区南门) &                  \\
    56               & 火车站(北广场)   & 泰华城(北侧)           & 谷德广场         \\
    59               & 虞河校区(北门)   & 虞河校区(东门)         & 人民医院(东门) \\*
                     & 谷德锦(南侧)     & 奥林匹克公园(东南侧)   &                  \\
    66               & 虞河校区(北门)   & 市中医院                 & 万达广场(南侧) \\
    69               & 浮烟山校区(北门) & 火车站(东侧)           & 火车站(北广场) \\
    71               & 浮烟山校区(北门) & 火车站(东侧)           & 火车站(北广场) \\
    75               & 虞河校区(西侧)   & 万达广场(西侧)         & 风筝广场         \\*
                     & 火车站(北广场)   &                          &                  \\
    101              & 浮烟山校区(北门) & 火车站(北广场)         &                  \\
    106              & 火车站(东北侧)   & 火车站(潍坊北站)       &                  \\
    109              & 浮烟山校区(南门) & 泰华城(东北侧)         &                  \\
    132              & 人民医院(西侧)   & 万达广场(西侧)         &                  \\
    162              & 虞河校区(东门)   & 人民医院(东门)         & 附院(西门)     \\*
                     & 奎文防疫站         &                          &                  \\
    167              & 浮烟山校区(南门) & 鲁台会展中心(北门)     & 蓝海大饭店
\end{tblr}

%\newpage %手动控制断页,原因不再赘述
\subsection[高铁学生票购买流程]{高铁学生票\footnotemark 购买流程}
\footnotetext{仍保留线下优惠资质核验、学生票购买渠道,若操作遇到问题可直接前往线下售票处,通过工作人员核验与购买;详细的注意事项等其他相关内容参见12306官网或其官方app。}
\begin{enumerate}
    \item 关注班级群内信息,及时填写“火车学生优惠申请”类的表格\footnotemark
          \footnotetext{仅入学时填写一次,如因搬迁等原因变更优惠区间请咨询班长。}
    \item 待学校完成学籍注册并下发学生证后,确认学生证背面贴有“火车票学生优惠卡”\footnotemark
          \footnotetext{若学生证丢失请以学籍证明替代,并前往人工窗口办理相关业务;学生证每学期以班级为单位统计一次并统一补办。}
    \item 首次使用请先在\uhref{https://www.chsi.com.cn}{学信网}确认学籍无误且已更新
    \item 打开12306 app→进入“我的”页面→点击“学生优惠资质核验”旁的“点击查看”按钮
    \item →填写学生资质信息→等待审核结果(3个工作日内反馈)
    \item 价格:高铁75折,普通火车5折(仅二等座可使用优惠,详见相关规定)
    \item \textbf{注意:}
          \begin{enumerate}
              \item 每学年\footnotemark 有4次优惠机会
                    \footnotetext{每年10月1日至下一年的9月30日为一个学年。}
              \item 每学年需重新核验\footnotemark 一次优惠资质
                    \footnotetext{现均已升级为线上自动核验,通常情况下无需操心。}
          \end{enumerate}
\end{enumerate}

\subsection[文体中心预约教程]{文体中心预约教程}
\label{sports_center_book}
\begin{enumerate}
    \item 关注“山二医文体中心”公众号→点击“场地预约”→“预约入口”
    \item →点击“我的”→“校内登录”→使用CAS认证系统的账号密码登录
    \item 进入个人中心页面,点击“人脸录入”,录入信息→
    \item 按需预约游泳馆、羽毛球馆等,并付费→
    \item 在预定时间段\footnotemark 前往场馆,向工作人员出示二维码即可
          \footnotetext{场馆开放时间见\uref{sports_center_operating_hours}{此处}。}
    \item 费用(校内学生):
          \begin{enumerate}
              \item 健体中心:3元/2小时
              \item 羽毛球馆\footnotemark:6元/时/片 3元/人/2小时
                    \footnotetext{羽毛球等多人运动项目预约方式为“单人预约,多人共享”,详情可咨询负责相关场馆的教师。}
              \item 篮球馆:20元/1小时(半场)或40元/1小时(全场)
              \item 游泳馆\footnotemark:8元/场/2小时
                    \footnotetext{注意:游泳馆仅限本人购票,代购无效,每人每场限购一张。}
              \item 室内乒乓球场:2元/台/2小时
              \item 室内网球场:6元/片/1小时
          \end{enumerate}
    \item 特殊说明:因羽毛球馆和篮球馆共用同一场地,故此二者互斥;场馆具体开放时间以及价格变动以公众号通知为准
\end{enumerate}
