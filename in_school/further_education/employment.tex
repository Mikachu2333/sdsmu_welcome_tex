% 就业指引
\chapter[就业指引]{就业指引\footnotemark}
\footnotetext{特别感谢:本章节内容由临床医学院党委提供。}

\section[总述]{总述}
\subsection[就业形式]{就业形式}
\begin{enumerate}
    \item 毕业生趋势:数量逐年增多
    \item 考研趋势:考研人数保持高位报考,“逆向考研”增多,再“考研”现象多见
    \item 就业情况:高质量充分就业难度增大,考研、考公、考编是学生心目中的“高质量就业”
\end{enumerate}

\subsection[就业政策]{就业政策}
以山东省潍坊市为例,每个地市都有相应的政策,如就业补贴、创业补贴、人才补贴、安家补贴等人才政策(可从人社部网站查找)
\bigbreak
部分常用政策及相关网站链接见下:
\begin{enumerate}
    \item 山东省人力资源与社会保障厅\uhref{http://hrss.shandong.gov.cn/articles/ch00300/202404/9dfeed90-6eb1-41dc-9ab3-a0fcedef8491.shtml}{《山东省级重点人才政策清单》}
    \item 潍坊市人力资源与社会保障局\uhref{http://rsj.weifang.gov.cn/zcfg/rszc/rczc}{《潍坊市人才政策》}(含人才引进政策、人才创业政策、人才就业政策、职称评审政策、人才激励政策等)
    \item 山东省人力资源与社会保障厅\uhref{http://103.239.153.109/sdjyweb/index.action}{山东公共就业人才服务网上服务大厅}
    \item \uhref{https://m12333.cn/weifang.aspx}{潍坊人社通}
\end{enumerate}

\section[常见就业形式]{常见就业形式}
\subsection[升学录取(含出国境)]{升学录取(含出国境)}
\subsubsection[考研]{考研}
考研是应届生的首选,如已经考过,要积极地从其他渠道进行就业

以近年考验情况为参考,每届毕业生的过线率在70\%左右,当年最终录取50\%;“二战”、“三战”后的总体录取比例达70\%左右,与当年的过线率基本一致。即,有30\%左右的同学,最终是需要就业的

请同学们根据自己的实际情况做好定位、统筹谋划、积极就业(可在备考同时考规培、助研助管等,一用一备,二战上岸或是二战失利仍有工作)

\subsubsection[第二学士学位]{第二学士学位}
适合于想成为复合型人才,需要有一个学校学习氛围的同学。可以以在校生的身份享有学校的学习资源和学习氛围;本校目前支持药学专业,其他学校还有很多

\subsection[住院医师规范化培训社会化招生(社会化规培)]{住院医师规范化培训社会化招生(社会化规培)}
是医学生就业升学的第二渠道,有以下一些优点:
\begin{enumerate}
    \item “两个同等对待”,相当于研究生毕业,入院招聘、职称晋升均同等对待
    \item “同工同酬”,规培期间社会化规培医生基本相当于医院的住院医师
    \item 计算工龄,且在医院积累三年工作经历,为后续就业积攒了优质工作经历和行业人脉资源,利于职业发展
    \item 规培是医学继续教育的一部分,职称晋升的必须,早规培早受益
    \item 规培期间考研不受影响,距离导师也近,可以“近水楼台先得月”
    \item 规培期间还可以申请在职硕士,三年后也有专硕学位
    \item 规培的政策、待遇还将继续优化,值得期待
    \item 研究生还有专陪
\end{enumerate}

\subsection[考公考编]{考公考编}
要积极获取信息、积极参与

有以下几种类型可供参考:
\begin{enumerate}
    \item 公务员如卫健委等
    \item 公立医院岗位
    \item 医学院校及职业院校教师、高校辅导员
    \item 国有企业(医药企业)等相关岗位
    \item 西部计划、三支一扶、乡村医生专项等政策性岗位
    \item 社区医院、乡镇医院等岗位
\end{enumerate}

\subsection[科研助理、管理助理]{科研助理、管理助理}
可以考取外校、科研院所、科研团队的科研助理、管理助理岗位,有不少可以“近水楼台先得月”,就近就便考研;还可以聘任医院、学校等的助研、助管岗位

\subsection[合同制、劳务派遣、第三方等到医院就业]{合同制、劳务派遣、第三方等到医院就业}
是曲线达成目标的方式,先选定就业单位、就业科室,然后在工作同时再考编制等,“先到先得”占住岗位。民营医院也是一个选择

\subsection[校医院等]{校医院等}
校医院,如大、中、小学的校医院、机场医生等

\subsection[医药企业、保险企业、健康查体企业等]{医药企业、保险企业、健康查体企业等}
临床也有很多毕业生进入企业,从事医药学术推广、医药代表、医学沟通顾问、医学信息专员、临床QC、医疗保险专员、健康专员等

\subsection[自由职业]{自由职业}
根据个人实际及兴趣爱好等实现其他形式就业,如自由职业、灵活就业等;直播带货、网文作家、网店店主等

\section[部分就业岗位信息链接]{部分就业岗位信息链接\footnotemark}
\footnotetext{部分链接为外部网站所整理,本处仅作汇总整理。敬请自行甄别筛选,谨防诈骗。}
\begin{enumerate}
    \item \uhref{https://www.ncss.cn}{国家大学生就业服务平台(24365校园招聘服务)}
    \item \uhref{https://job.ncss.cn}{教育部大学生就业网}
    \item \uhref{https://www.sdgxbys.cn}{山东高校毕业生就业信息网}(公众号“山东就业创业导航”)
    \item \uhref{https://www.sdbys.com}{就选山东・青年人才全链条服务平台}
    \item \uhref{https://wfmc.sdbys.com}{山东第二医科大学就业信息网}(公众号“山二医就业服务”)
    \item \uhref{https://www.med66.com/weishengdanweizhaopin/gonggao/wa2406033156.shtml}{正保医学教育网-2024年6月全国各级医疗卫生单位招聘公告汇总}
    \item 住院医师规范化培训(公众号“规培”、“规培基地”,小程序“规培宝”)
    \item 第二学士学位(公众号“第二学士学位”、“第二学士学位咨询台”,小程序“第二学士学位帮帮帮”)
    \item \uhref{http://job.mohrss.gov.cn}{中国公共招聘网}
    \item \uhref{https://www.newjobs.com.cn}{中国国家人才网}
    \item \uhref{https://www.yimaitongdao.com}{医脉同道}
    \item \uhref{https://www.jobonline.cn/probation/index}{就业在线-百万就业见习岗位募集}
\end{enumerate}