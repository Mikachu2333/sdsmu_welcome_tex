%chapter5textbf
\chapter[宿舍]{宿舍}
\section[整体情况介绍]{整体情况介绍}
\begin{enumerate}
    \item 绝大多数的宿舍为6人间,上下床(各床铺靠墙侧均有1插座,可解插排),配备桌子(桌洞8个)一张,书桌(理论可坐3人但空间紧张,常2人,含6个小书架)一张,有柜子(常为8个),垃圾桶,洗漱池,空调,电风扇,暖气片,烟雾报警器等宿舍基本设施
    \item 柜子一般为8个,6人各选一个柜子,剩下两个公用(柜子长度极长,入口较小,如需购置隔板等建议在学校期间先测量尺寸再购置)
    \item 具有独立卫生间
    \item 部分女生宿舍的卫生间内有浴室,可直接在卫生间内淋浴;其他同学可以在本楼层小公共澡堂(有2个毛玻璃隔间,有门帘,无需预约)或者一楼的大公共澡堂(不透明隔间,有门帘,需软件预约)洗澡。教程见此
    %\ref{wash}
    \item 吹风机使用教程见此
    %\ref{dry}
    \item 宿舍楼06:00开门,22:30关门,23:00熄灯(每周六、每年9月和考试月除外);大一期间,院系学生会每周不定时抽查是否存在夜不归宿情况
    \item 使用门禁系统刷脸进出宿舍
    \item 宿舍楼提供免费的100℃开水\footnote{饮水机每层一个,位于公共洗漱间内;为自来水烧开;熄灯后停止供应。推荐用于泡面、洗衣、泡脚等用途},若对饮水水质要求不高也可直接饮用;否则请前往大学生服务中心〔大服〕打水处(地图见\ref{map_a})付费接取纯净水
    \item 宿舍单个插座限电300W,超限将导致全楼停电
    \item 除跳闸断电等特殊情况外,其余时间段均有4G及5G信号覆盖,延迟波动较大(20-200ms),平均网速1.5-5M/s
\end{enumerate}

\section[住宿注意事项]{住宿注意事项}
\begin{enumerate}
    \item \textbf{\uuline{宿舍以专业为单位进行随机分配宿舍楼,班内宿舍多按姓氏顺序分配,与录取分数无关}}
    \item 是否自带被褥等可按照个人需求决定,录取通知书里有相关的尺寸以及样式要求\footnote{新生军训期间对于床单的颜色等具有特别要求}
    \item 宿舍门禁系统将在开学军训期间进行人脸录入,\textbf{\uuline{切忌美颜过度,否则无法识别}}
    \item 因存在消防隐患,学校统一\textbf{禁止自挂床帘}
    \item 宿舍内备有烟雾传感器,\textbf{吸烟将引发报警}
    \item 床上桌等类似的东西建议到学校实际生活1个月以后再决定是否购买(大部分人都用不到,少数同学用来打游戏,\sout{然而身体在床上缩着打游戏超级难受};更少一部分同学用其学习,\sout{该现象\\比彩色大熊猫更加罕见})
    \item 宿舍单个插座的功率限制300W\footnote{如果接了一个插排,插排的总功率不能超过300W,例如一个67W的手机充电器和一个200W的游戏本电脑,再加一个小台灯就很容易跳闸了},\uuline{\textbf{吹风机,锅,电暖宝,电水壶,热得快等高功率电器均严}}\\\uuline{\textbf{禁使用},否则将引起宿舍全楼停电};宿管不定期来查,若被发现将被没收并通报批评、写检讨
    \item 如需装修宿舍,可通过参加学校统一开展的\textbf{“宿舍装扮大赛”}\footnote{禁止外来装修人员入校、禁止私改电路,详情内容见开学后下发的参赛要求}以对宿舍进行小幅调整
    \item 背阴面宿舍的阳台相当于摆设,\textbf{推荐在楼下晾晒衣物},若只靠阴干,很容易发霉发臭;向阳面无此困扰
    \item \textbf{一旦离开宿舍必须关灯关电,若插排未拔将被没收并扣分}
    \item 禁止将正在持续充电的手机、充电宝直接置于被子等密闭、空气无流通的环境中,极不推荐直接将笔记本电脑置于被子上使用,严重影响散热且有着火风险
    \item \uuline{\textbf{为保证他人睡眠,熄灯后严禁使用台灯、手电筒在宿舍内继续学习;更不要在制造噪音!}}\\\uuline{(很多同学难以入睡且睡眠浅,小动静或急速的明暗变化就能被惊醒,请大家务必相互尊\\重、相互理解)}
    \item 如遇宿舍公用物品损坏如门锁,门轴,玻璃,灯管,水龙头,下水道等,可前往宿舍一楼楼管处报修或咨询带班学长学姐处理方法,也可拨打物业电话。为保障安全,学校严禁自行修理
    \item 新建的13、14号公寓具体使用情况未明,以学校通知为准
\end{enumerate}

