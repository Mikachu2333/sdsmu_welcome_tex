%chapter9
\chapter[衣食住玩与生活]{衣食住玩与生活}
\noindent 特别声明:本文中所有\textbf{“大服”},均为\textbf{“大学生服务中心”}的习惯性缩略称呼。
\section[衣]{衣}
\begin{enumerate}
    \item 大服的2、3层均有服饰商店可自行选购衣物
    \item 推荐网购,也可乘公共汽车前往大型商超购置
    \item 部分院系提供自愿的系服购买服务,详见各院系通知
\end{enumerate}

\section[食与生活]{食与生活\footnotemark[1]\footnotemark[2]\footnotemark[3]\footnotemark[4]\footnotemark[5]}
\footnotetext[1]{因文章篇幅原因,本指南仅罗列了同学们提及次数较多的食物或店铺,未能全部列出敬请谅解。}
\footnotetext[2]{下列提及的食物(店铺)均按照空间顺序排列,与好吃程度无关,所用名称为同学习惯性称呼,括号内为特别提醒。}
\footnotetext[3]{标注“$^{〈早〉}$”的店铺约6:00即开始供应。}
\footnotetext[4]{标注“$^{〈晚〉}$”的店铺营业时间最晚可至22:30,其余均在18:30左右停业。}
\footnotetext[5]{奶茶/咖啡店、超市、水果店等单独说明。}

\subsection[大服]{大服}
大服有大量商家提供多种食物,大部分的价格较食堂稍高。
\begin{table*}[ht]
    \centering
    \begin{tabular}{c|c|c|c|c|c|}
        \Xhline{1.2pt}
        \multirow{3}{*}{1层}  & \multirow{2}{*}{内部} & 金小麵$^{〈早〉}$(锅贴)     & 自选菜           & 老陕面馆                   & 馋嘴鱼        \\
        \cline{3-6}
                             &                     & 米粉                  & 肠粉            & 肉夹馍$^{〈晚〉}$            & 冒菜         \\
        \Xcline{2-6}{0.8pt}
                             & 外部                  & 砂锅$^{〈早〉}$(火烧、豆腐脑)  & 大饼卷一切$^{〈晚〉}$ & 速食主义$^{〈早〉}$           & 烧烤$^{〈晚〉}$ \\
        \Xhline{1.2pt}
        \multirow{2}{*}{-1层} & 兰李于                 & 福香面馆$^{〈早〉}$(豆腐脑油条) & 螺狮粉           & 自选菜                    & 蟹王堡        \\
        \cline{2-6}
                             & 烤鸡架                 & 老陕面馆                & 馋嘴鱼           & \multicolumn{2}{c|}{略}              \\
        \Xhline{1.2pt}
    \end{tabular}
\end{table*}

\subsection[杏林餐厅]{杏林餐厅}
杏林餐厅全部三层均有大量食物,大多物美价廉。

\newpage
\begin{table*}[ht]
    \centering
    \begin{tabular}{c|c|c|c|c|}
        \Xhline{1.2pt}
        \multirow{3}{*}{1层} & 麦西麦乐                                                & 包子水饺$^{〈早〉}$ & 牛肉板面                   & 兰州拉面     \\
        \cline{2-5}
                            & 永和豆浆$^{〈早〉}$(油条麻花)                                  & 自选菜(稍贵)      & 豆浆油条                   & 盒饭(便宜量大) \\
        \cline{2-5}
                            & 粥$^{〈早〉}$(种类多)                                      & 馄饨           & 麦西麦乐面包                 & 略        \\
        \Xhline{1.2pt}
        \multirow{3}{*}{2层} & 大骨饭                                                 & 麻汁馄饨         & 水饺                     & 东北玉米面    \\
        \cline{2-5}
                            & 烤鸭饭(瓦罐汤)                                            & 铁板炒饭(量大管饱)   & 清真窗口                   & 茶拌饭      \\
        \cline{2-5}
                            & 馋嘴鱼                                                 & 自选水饺         & \multicolumn{2}{c|}{略}            \\
        \Xhline{1.2pt}
        3层\footnotemark     & \multicolumn{4}{c|}{自选菜(稍贵;小包间式,部门聚餐推荐,包间人数上限为15人)}                                                    \\
        \Xhline{1.2pt}
    \end{tabular}
\end{table*}
\footnotetext{仅餐厅东南侧楼梯可前往,餐厅东北侧楼梯通往原乒乓球场。}

\subsection[汇金街]{汇金街}
学校南门顺福源路往东一个路口,有大量的饭店,价格较市里相对高昂。
\begin{table*}[ht]
    \centering
    \begin{tabular}{|c|c|c|c|}
        \Xhline{1.2pt}
        满江红 & 石锅鱼 & 暖溢水饺 & 幸福餐厅 \\
        \Xhline{1.2pt}
    \end{tabular}
\end{table*}

\subsection[超市]{超市}
\begin{table*}[ht]
    \centering
    \begin{tabular}{|c|c|c|}
        \Xhline{1.2pt}
        习惯称呼         & 地点      & 物品                   \\
        \Xhline{1.2pt}
        大服超市$^{〈晚〉}$ & 在大服正中央  & 日用品,零食,饮料,手套,头套,作业本等 \\
        \hline
        中和超市         & 中和广场    & 日用品(少)、零食、饮料等        \\
        \hline
        餐厅超市         & 餐厅西北侧入口 & 零食、饮料等               \\
        \Xhline{1.2pt}
    \end{tabular}
\end{table*}

\subsection[水果店]{水果店}
\begin{table*}[ht]
    \centering
    \begin{tabular}{|c|c|c|c|c|c|}
        \Xhline{1.2pt}
        习惯称呼    & 地点      & 种类 & 新鲜   & 价格 \\
        \Xhline{1.2pt}
        餐厅西水果店  & 餐厅正西侧入口 & 较少 & 一般   & 一般 \\
        \hline
        餐厅南水果店  & 餐厅正南侧入口 & 多  & 好    & 略高 \\
        \hline
        大服水果店   & 大服      & 最多 & 一般或差 & 最高 \\
        \hline
        中和/大服超市 & 略       & 最少 & 一般或差 & 最低 \\
        \Xhline{1.2pt}
    \end{tabular}
\end{table*}

\section[奶茶/咖啡店]{奶茶/咖啡店}

沪上阿姨(食堂一楼)、蜜雪冰城(食堂一楼)、茶百道(大服一楼外部)、益禾堂(大服一楼外部)、阿水大杯茶(食堂一楼)、臻茶(山东特有,食堂一楼)、库迪咖啡(大服二楼,[正在装修],估计九月开业)、遇见烧仙草(大服二楼)、麦克风(食堂一楼)、小度(食堂一楼)、冰雪岛(食堂一楼)、超级奶爸(食堂一楼),潍医咖啡(食堂一楼外部);
7.其他常用地点:
大服办公室(水卡办卡、充值与退卡)在大服二楼东北侧(仅在非法定节假日正常上班时间受理业务);
联通营业厅位于大服一楼北侧;
移动营业厅位于中和广场,在A104教室正对面;
理发店(两家),均在大服二楼;
干洗店(也配钥匙)、裁缝店均位于大服二层东侧;
手机、电脑维修店位于大服二楼最东南侧;
复印店$^{〈晚〉}$(两家)均位于大服二楼;
小药店、牙科门诊位于大服二楼最西北侧;
大服的厕所在负一层东北侧;
中和打印店(卖二手书)位于中和广场,在A105教室正对面;
大服健身房位于大服三层(需从西北侧楼梯进入);
台球厅位于大服三层。

(三)住
1.南街(路名:汇金街,位于校园东南侧)提供宾馆、住宿型自习室等可供住宿的地方;
2.附近小区有较多房屋出租(须在学校办理走读手续后才可在外居住)。

(四)玩
1.多数同学常通过步行前往南街(汇金街),有KTV、电影院等娱乐场所;
2.可通过69、71路公交车(北门乘坐)或13、109路公交车(南门乘坐)前往市区(如泰华、万达、谷德茂等)进行游玩;
3.校内的文体中心(详见地图)内有羽毛球馆、篮球馆(两者互斥)、健身房,还有游泳馆,具体收费标准及预约方式见下文教程;