%chapter11
\chapter[常用教程]{常用教程}

\section*{特别说明}
本节中所有上标“㊕”的网址仅在连接校园网时可成功访问相关服务。

\section[新生信息查询]{\uuline{新生信息查询}}
\label{freshman_query}
\begin{enumerate}
    \item 关注“潍坊医学院学生之家”公众号
    \item 点击菜单栏中的“新生报到”→使用身份证号、手机号、验证码登录
    \item →根据其中的相关指引完成预报到流程
    \item →查看宿舍号、学号、班级、院系等相关信息,并根据个人需求选择是否订购军训套装、被褥套装等\footnotemark
          \footnotetext{也可来校当场订购,详见系统及录取通知书说明。}
\end{enumerate}

\section[校园网]{校园网}
\begin{enumerate}
    \item 激活:到校报到完成后,学校将按照个人身份证号在\uline{\href{http://210.44.80.65/}{http://210.44.80.65/}$^㊕$}为大家开通校园网账号,并告知初始密码\footnotemark
          \footnotetext{每年不同,详见班级群内具体通知。}
    \item \textbf{\uuline{充值}}:点击\uline{\href{https://slzfw.wfmc.edu.cn:8800/home/}{https://slzfw.wfmc.edu.cn:8800/home}$^㊕$},并登录;或在校园网的登陆页面点击“自助服务”按钮\footnotemark,登陆后即可使用支付宝充值(微信支付暂时无法使用)
          \footnotetext{若点击按钮后网页显示“403 Error”,请将网址最前面的“http”改为“https”,按回车即可。}
    \item 校园网流量当前政策为每月免费30G,超出部分按0.5\textsf{¥}/G收费
\end{enumerate}

\section[校园手机卡]{校园手机卡}
\begin{enumerate}
    \item \textbf{开通与否全凭自愿,是否开通校园卡不影响校园网的使用。}随录取通知书一并寄出。如遇强制开通可告知带班学长或自行反馈
    \item 通常内含200分钟全国通话、10G全国流量、80G校园流量(仅山东省内所有高校可用)\footnotemark
          \footnotetext{详情优惠政策可咨询营业厅,如追求更多流量建议学校各账号不绑定校园手机卡,每年根据新优惠政策调整手机卡(需及时注意注销旧手机卡以防欠费导致的信用记录问题)。}
    \item 开学报到当天可前往大服,免费领取礼品
\end{enumerate}

\section[空调使用教程]{空调使用教程}
\label{air_control}
\begin{enumerate}
    \item 微信关注“海享租”公众号,点击公众号菜单“在线租赁”,并注册、登录
    \item 点击“扫一扫”→扫描空调右下角二维码进行租赁\footnotemark
          \footnotetext{若提示租赁失败,请按照软件提示联系同宿舍的学长/学姐退租,也可咨询学长学姐或向宿管反馈。}
    \item →租赁完成后,点击“设备”→“空调图标”→“时长”,进行充值
    \item →点击“设备”→“空调图标”→“成员管理”,在此页面下将宿舍全部成员权限设置为均可管理空调开关即可
    \item \textbf{注意:}空调使用时长收费(0.55元/小时),具体收费及租赁政策详见“海享租”公众号
\end{enumerate}

\section[浴室预约与使用]{浴室预约与使用}
\label{wash_software}
\begin{enumerate}
    \item 软件基础设置:
          \begin{enumerate}
              \item 在手机应用市场下载“大白U帮”app
              \item 按照实际住宿情况注册
              \item 授予并开启“定位”与“蓝牙”权限
          \end{enumerate}
    \item 本楼层小浴室使用:
          \begin{enumerate}
              \item 带好洗浴物品前往公共厕所旁边的浴室排队
              \item 进入浴室,点击如右图所示的按钮(\mbox{\includegraphics[height=2.4ex]{bath.pdf}})→选择“蓝牙设备”→“点击进行时”
              \item →“洗澡”→“搜索洗澡”\footnotemark
                    \footnotetext{搜索不到设备请务必开启“蓝牙”功能,学校的设备无法扫码连接。}
              \item →选择设备\footnotemark →“开始洗澡”
                    \footnotetext{距离厕所入口最近的是1号,远的是2号;不确定可以询问学长。}
              \item 结束后点击“结束洗澡”按钮,并结算
          \end{enumerate}
    \item 一层公共大浴室预约:
          \begin{enumerate}
              \item 在软件初始界面根据实际情况选择“X号楼1层”的浴室
              \item →点击一个浴位,并点击“预约”按钮(若已满请选择“排队”)
              \item →在8分钟内前往浴室,并点击“开始洗浴”
              \item →结束后点击“结束洗澡”按钮,并结算
          \end{enumerate}
    \item 费用:以程序显示为准,详情收费标准略
    \item 申诉:如果在洗澡时突然停电导致无法结束洗澡而被扣费,请按照软件打开时弹出的公告,联系相关工作人员处理
    \item \textbf{注意:如果未点击“结束洗澡”按钮便直接离开可能会被多扣费}
\end{enumerate}

\section[洗衣机/洗鞋机使用教程]{洗衣机/洗鞋机使用教程}
\label{washing_machine}
\begin{enumerate}
    \item 在手机应用市场下载“海狸洗衣”app,并注册、开启相关权限
    \item 在软件内点击“立即下单”→搜索宿舍楼号(单个数字即可,例如:11),选择楼层与洗衣机,预约即可
    \item →放入衣物与洗衣粉,输入验证码然后缴费即可使用
    \item \textbf{注意:}洗衣机有两个,旁边就是洗鞋机,\textbf{请勿使用洗衣机洗鞋!!!}
    \item \textbf{洗衣粉和洗衣液都在第一格,禁止向第二格内倾倒洗衣粉、洗衣液,那是柔顺剂的格子!}
    \item 收费标准详见软件说明
    \item 洗衣机错误处理办法(若无相关经验切勿自行动手操作):
          \begin{enumerate}
              \item 拨打洗衣机旁边的报修电话;
              \item E1:洗衣机断电后开门,打开洗衣机右下角小门,旋开阀门,使用镊子等工具伸入并清除其中堵塞管道的杂物,恢复原样即可;
              \item E4:旋开洗衣机后方的水管阀门即可。
          \end{enumerate}
\end{enumerate}

\section[烘干机使用教程]{烘干机使用教程}
\label{dry_machine}
\begin{enumerate}
    \item 在手机应用市场下载“海狸洗衣”app,并注册、开启相关权限
    \item 查看烘干机是否空闲,放入已事先脱水过的衣物烘干即可
    \item 推荐烘干配置
          \begin{enumerate}
              \item 高温60分钟:大部分轻薄的衣物(例如T恤、卫衣、浴巾等)
              \item 高温120分钟:薄被(如夏凉被)
          \end{enumerate}
    \item \textbf{注意:}\textbf{使用前后务必控干水箱并清理滤网。}棉被、羽绒服等禁止使用烘干机烘干以免损坏及不必要的危险情况发生。
    \item 收费标准详见软件提示
\end{enumerate}

\section[吹风机使用教程]{吹风机使用教程}
\label{hair_drier}
\begin{enumerate}
    \item 每层公共浴室旁边有两个公用吹风机,需扫码\footnotemark 租赁使用
          \footnotetext{部分吹风机屏幕二维码可能有缺损,不易扫描成功,多次尝试即可。}
    \item →待手机发出“滴”-“滴”的声音后租赁成功
    \item →将手机扬声器对准吹风机租赁器方可正常使用
    \item 收费标准:详见软件提示,1分钱起步
\end{enumerate}

\section[设施报修方式枚举]{设施报修方式枚举}
\label{repair_report}
\begin{enumerate}
    \item 加入各宿舍楼的QQ报修群,在群内反映具体故障
    \item 前往一层宿管处填表报修
    \item 在宿舍一层宿管旁边的公告栏处查看相关负责人的电话,直接拨打即可
    \item 询问带班学长、学姐
\end{enumerate}

\section[多媒体教室申请流程]{多媒体教室申请流程}
\begin{enumerate}
    \item 打印《多媒体教室使用审批表》并填写
    \item 前往本年级学工办签字、盖章
    \item 前往教室E区2层(见\uline{\ref{map_t}})靠近A区处的“教室管理中心”签字盖章
    \item 前往预约的教室,拨打讲台上教室管理员的电话沟通说明
\end{enumerate}

\section[乐道济世书院教室申请流程]{乐道济世书院教室申请流程}
\begin{enumerate}
    \item 打印《入驻乐道济世书院申请表》并填写
    \item 前往本年级学工办签字、盖章
    \item 前往“书院管理办公室”(见\uline{\ref{map_a}})签字盖章
\end{enumerate}

\section[钉钉请假流程]{钉钉请假流程\footnotemark}
\label{leave_dingtalk}
\footnotetext{待学校统一将大家拉入钉钉的“潍坊医学院”企业后方可使用。}
\begin{enumerate}
    \item \textbf{注意:}各学院要求不一,仅以临床医学院为例
    \item 线下请假步骤(正常情况):
          \begin{enumerate}
              \item 前往学工办或班主任办公室,当面请假并获得假条
              \item →根据老师要求扫描相关二维码
              \item →钉钉填表
              \item →刷脸进出校门,并将请假条之一交给保卫处
              \item →返校后,在钉钉的电子假条处,以评论的方式销假
          \end{enumerate}
    \item 线上请假步骤:
          \begin{enumerate}
              \item 打开钉钉→点击左上角选择主企业为“潍坊医学院”
              \item →点击页面最下方菜单栏“工作台”→点击“OA审批”
              \item →在“学生日常事务管理”类选择“浮烟山校区本科学生请假单”→填表
              \item →电话联系班主任老师或学工办老师,说明请假事由并等待审批
              \item →审批通过后刷脸进出校门
              \item →返校后,在钉钉的电子假条处,以评论的方式销假
          \end{enumerate}
\end{enumerate}


\section[统一支付平台教程]{统一支付平台教程}
\label{fee_pay}
\begin{enumerate}
    \item 官网:微信小程序“潍坊医学院财务”或\uline{\href{http://tyzfpt.wfmc.edu.cn/xysf/login.aspx}{http://tyzfpt.wfmc.edu.cn/xysf/login.aspx}}
    \item 用途:学费缴纳、卡号绑定等
    \item 学费缴纳教程:
          \begin{enumerate}
              \item 前往网站或公众号菜单,点击右下角“缴费管理”→“支付平台”
              \item →使用学号+姓首字母大写加身份证后六位登录系统
              \item →按照提示修改初始密码(请务必牢记)→进行缴费
          \end{enumerate}
    \item 银行卡绑定教程:
          \begin{enumerate}
              \item 目的:学校仅在初次使用时收集一次卡号并存储数据,以便下次直接使用\footnotemark
                    \footnotetext{详情见学校官方说明。}
              \item 打开“潍坊医学院财务”公众号,点击“财务中心”
              \item →使用帐号密码登录\footnotemark 并绑定微信号
              \item →点击“卡号维护”→“管理”
              \item →按照提示填写相关信息
              \item 确认信息无误后提交即可
                    \footnotetext{帐号为学号,原始密码为000000。}
          \end{enumerate}
\end{enumerate}

\section[学工系统(微信小程序)]{学工系统(微信小程序)}
\begin{enumerate}
    \item 用途:学工系统主要用于晚点名、返校信息填报等日常工作\footnotemark
          \footnotetext{原“请假审批”、“外出审批”工作已基本转移至“钉钉”,教程参见\uline{\ref{leave_dingtalk}}。}
    \item 使用方式:
          \begin{enumerate}
              \item 在微信搜索“智慧学工”小程序
              \item 授予“定位”权限
              \item 根据学校下发的账号密码进行登录(推荐立即与微信绑定以免忘记密码)
          \end{enumerate}
\end{enumerate}

\section[教务系统]{\textbf{\uuline{教务系统}}}
\begin{enumerate}
    \item 官网:\uline{\href{https://jwgl.wfmc.edu.cn/}{https://jwgl.wfmc.edu.cn/}$^㊕$}
    \item 用途:\textbf{选课,缓考申请,成绩查询},查看(导出)课程表,空闲教室查询
    \item 缓考申请教程:
          \begin{enumerate}
              \item 进入教务系统
              \item 点击左侧菜单“考试报名”→“我的申请”→“缓考申请”
              \item →选择“学年学期”和“活动名称”后,直接点击“搜索”(不要填写科目名称)
              \item →在弹出的菜单中选择缓考科目并填写申请\footnotemark
                    \footnotetext{若因病缓考体测,需要上传病历本等相关材料,并前往校医院开具证明,再前往学工办向教师当面说明情况。}
          \end{enumerate}
    \item \textbf{注意:}仅限校内访问,如需在家使用教务系统,参见\uline{\ref{cas_system}}条目
\end{enumerate}

\section[资源访问控制系统(校内VPN)]{\textbf{\uuline{资源访问控制系统(校内VPN)}}\footnotemark}
\footnotetext{在校内请直接访问相应系统,无需使用本系统中转。}
\label{cas_system}
\begin{enumerate}
    \item 官网:\uline{\href{https://webvpn.wfmc.edu.cn/}{https://webvpn.wfmc.edu.cn/}}
    \item 用途:本系统用于在校外访问校内网络信息资源,如:教务系统、知网、临床医学虚拟仿真实验中心\footnotemark 等
          \footnotetext{查阅文献推荐使用CARSI系统,详情见\uline{\ref{carsi_system}}。}
    \item 异地登录教务系统教程(其它系统同理):
          \begin{enumerate}
              \item 打开网站,点击“统一身份认证登录”,按照学校下发的专用账号、密码登录即可(推荐绑定微信)
              \item →找到应用中心→“教务系统-非单点登录”\footnotemark →使用教务系统账号密码登录即可
                    \footnotetext{请注意,在校外时点击“教务系统”无法登录,只有“非单点”能校外登录!}
          \end{enumerate}
\end{enumerate}

\section[CARSI系统]{\textbf{\uuline{CARSI系统}}}
\label{carsi_system}
\begin{enumerate}
    \item 官网:\uline{\href{https://ds.carsi.edu.cn}{https://ds.carsi.edu.cn}}
    \item 用途:快速访问学校订阅的各类数据库,如百度文库、知网、万方、维普等
    \item 使用教程:
          \begin{enumerate}
              \item 进入官网→搜索“潍坊医学院”并勾选“记住我的选择”→点击后进入登陆界面\footnotemark
                    \footnotetext{请注意,不要收藏登录界面的网址!每次登录网址都不一样,只能从官网重新进入!}
              \item →使用\textbf{校园网账密}登录系统,出现各类弹窗一律选择“Accept”即可\footnotemark
                    \footnotetext{仅推荐在自己的电脑上如此设置,如必须在网吧等公共场所的电脑上使用,请审慎阅读相关提示,并谨慎进行登录,如因账号泄露造成损失,一切责任自负。}
              \item →登录完成后,点击任意资源链接即可进入相应网站并获取论文
          \end{enumerate}
\end{enumerate}

\section[校务行(微信小程序)]{校务行(微信小程序)}
\begin{enumerate}
    \item 官网:微信小程序
    \item 用途:查成绩,下载学籍证明、成绩证明的pdf版本
    \item 费用:以程序显示为准
\end{enumerate}

\section[学生票购买流程]{学生票购买\footnotemark 流程}
\footnotetext{仍保留线下优惠资质核验、学生票购买渠道,若操作遇到问题可直接前往线下售票处,通过工作人员核验与购买;详细的注意事项等参见12306官网或官方app。}
\begin{enumerate}
    \item 等待学校完成学籍注册,待学生证下发并加盖注册章
    \item 首次使用前请先在\uline{\href{https://www.chsi.com.cn/}{学信网}}查看学籍,确认信息已更新
    \item 打开12306app→进入“我的”页面
    \item →点击“学生优惠资质核验”旁的“点击查看”按钮
    \item →填写学生资质信息→等待审核结果(3个工作日内反馈)
    \item 价格:高铁75折,普通火车5折(仅二等座可使用优惠,详见相关规定)
    \item \textbf{注意:}每学年\footnotemark 有4次优惠机会;每学年需重新核验一次优惠资质
          \footnotetext{每年10月1日至下一年的9月30日为一个学年。}
\end{enumerate}

\section[文体中心预约教程]{文体中心预约教程}
\label{sports_center}
\begin{enumerate}
    \item 关注“潍医文体中心”公众号→点击“场地预约”→“预约入口”
    \item →点击“我的”→“校内登录”→使用cas认证系统的账号密码登录
    \item 进入个人中心页面,点击“人脸录入”,录入信息→
    \item 按需预约 游泳馆、羽毛球馆等,并付费→
    \item 在预定时间段前往场馆,向工作人员出示二维码 即可
    \item 费用:
          \begin{enumerate}
              \item 健体中心:3元/2小时
              \item 羽毛球馆\footnotemark:6元/1小时/片 3元/人/2小时
                    \footnotetext{羽毛球等多人运动项目预约方式为:单人预约,多人共享。详情可咨询负责相关场馆的教师。}
              \item 篮球馆:20元/1小时(半场)或40元/1小时(全场)
              \item 游泳馆\footnotemark:8元/场/2小时
                    \footnotetext{请注意,游泳馆不可帮他人预约,每人每场限购一张入场票。}
              \item 室内乒乓球场:2元/台/2小时
              \item 室内网球场:6元/片/1小时
          \end{enumerate}
    \item 特殊说明:因羽毛球馆和篮球馆共用同一场地,故此二者互斥;场馆开放具体时间以及价格变动以公众号通知为准
\end{enumerate}
