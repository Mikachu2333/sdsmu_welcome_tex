% !TeX encoding = UTF-8
% !TEX TS-program = xelatex

% 山东第二医科大学入学与生活指南 © 2023-2024 by LinkChou is licensed under CC BY-SA 4.0. To view a copy of this license, visit 

% 导入所需的package
\documentclass[a4paper,twoside,onecolumn,12pt,fontset=none]{ctexrep}%ctexreport文档类,fontset=none避免字体冲突警告
\usepackage{bxtexlogo}%LaTeX logo(禁止换用hvlogos,因其会重定义部分字体导致报错)
\usepackage{fancyhdr,titlesec}%标题、脚注格式
\usepackage{indentfirst,ulem,xcolor}%段落、文字格式
\usepackage{graphicx}%插入图片
\usepackage{float}%图表位置
\usepackage{tabularray}%表格排版
\usepackage{enumitem}%列表排版
\usepackage[a4paper, hmargin=1.5cm, top=1.5cm, bottom=2cm, headheight=0pt, ignorehead]{geometry}%页边距与排版
\usepackage[final=true, colorlinks, hidelinks, bookmarks, bookmarksnumbered=true]{hyperref}%超链接
\usepackage{texosquery}%获取秒数及时间
\usepackage{datetime2}%扉页基本时间及格式
%\usepackage{showframe}%显示页面布局与边框,仅Preview版本排查错误

% 重设定标题与正文间距
\ctexset{
    chapter={
        beforeskip=0ex,
        afterskip=4ex
    },
    section={
        beforeskip=0ex,
        afterskip=0ex
    },
    subsection={
        beforeskip=0ex,
        afterskip=0ex
    },
    subsubsection={
        beforeskip=0ex,
        afterskip=0ex
    }
}

% 重定义部分logo符号以免冲突
\bxtexlogoimport{XeLaTeX}
\bxtexlogoimport{LaTeXe}

% 重定义文章字体为梦源宋体(不使用思源宋体的原因是思源的行高过高)
\setmainfont[BoldFont=Dream Han Serif CN W20]{Dream Han Serif CN W7}
\setCJKmainfont{Dream Han Serif CN W7}[BoldFont=Dream Han Serif CN W20]
\setCJKsansfont[BoldFont=Dream Han Serif CN W20]{Dream Han Serif CN W7}

% 段间距与行间距重设定
\setlength{\parskip}{.75ex plus 5pt minus 8pt}
\setlength{\lineskip}{.5ex plus 5pt minus 8pt}

% 页眉页脚
\pagestyle{fancy}
\fancyhf{}
\fancyfoot[C]{\thepage}

% 脚注格式(无分割线)
\renewcommand{\headrulewidth}{0pt}
\renewcommand{\footrulewidth}{0pt}
\renewcommand{\thefootnote}{\fnsymbol{footnote}}

% 列表样式重定义
\setlist[enumerate,1]{label=\arabic*.}
\setlist[enumerate,2]{label=(\arabic*)}
\setlist[enumerate,3]{label=[\arabic*]}
\setlist[enumerate,4]{label=<\arabic*>}
\setlist[enumerate]{itemsep=0pt}

%重定义表格脚注字体大小并增加“no-caption”表格样式
\DefTblrTemplate{note-text}{default}{{\footnotesize\InsertTblrNoteText}}
\DefTblrTemplate{remark-tag}{default}{{\footnotesize\bfseries\InsertTblrRemarkTag}}
\DefTblrTemplate{remark-text}{default}{{\footnotesize\InsertTblrRemarkText}}
\NewTblrTheme{no-caption}{
    \SetTblrTemplate{caption}{empty}
}

%重定义表格换页时内容
\DefTblrTemplate{contfoot-text}{default}{续下页}
\DefTblrTemplate{conthead-text}{default}{(续)}

% 设置校徽命令
\newcommand{\BackgroundPic}{
    \noindent\parbox{\paperwidth}{
        \centering
        \vspace*{400pt}
        \includegraphics[width=5cm]{resources/sundry/logo.pdf}
    }
}

% 颜色重设定
\newcommand{\colored}[4]{
    \textcolor[RGB]{#1,#2,#3}{#4}
}

% 时间样式重设定
\newcommand{\CurrentCustomTime}{
    \DTMsavenow{Compiled_time}%保存编译时间
    \DTMfetchyear{Compiled_time}.%
    \DTMtwodigits{\DTMfetchmonth{Compiled_time}}.%
    \DTMtwodigits{\DTMfetchday{Compiled_time}}\quad%
    \DTMtwodigits{\DTMfetchhour{Compiled_time}}:%
    \DTMtwodigits{\DTMfetchminute{Compiled_time}}:%
    \DTMtwodigits{\DTMfetchsecond{Compiled_time}}\quad%
    GMT\DTMfetchTZhour{Compiled_time}:\DTMfetchTZminute{Compiled_time}
}

\begin{document}

% 扉页

% 添加标题校徽背景图
\AddToHookNext{shipout/background}{\BackgroundPic}

% 题目
\title{%
\normalsize
\vspace{100pt}
{\Huge\textbf{山东第二医科大学指南}}\\[10pt]
(简称:\colored{171}{79}{36}{321指南})\\[30pt]
%{\large\textbf{\colored{59}{100}{220}{4.0.0.0}}}\vspace*{-25pt}}
{\large\colored{255}{0}{0}{3768-Preview}}\vspace*{-25pt}}
\author{LinkChou\thanks{Compiled by \LaTeXe, CC BY-SA 4.0 LICENSE.\qquad%
        \uhref{Mailto:LinkChou@yandex.com}{Mailto: LinkChou@yandex.com}\\
        \textbf{敬告:}\\
        \indent\indent 力薄才疏,不免舛误,敬请绳愆纠缪。具体内容均以学校官方为准,如有变更恕不另行通知。}
    \and 山东第二医科大学频道}
\date{\colored{36}{98}{105}{\CurrentCustomTime}}
\maketitle

% 脚注样式重设定
\renewcommand{\thefootnote}{\arabic{footnote}}

\tableofcontents%标题、目录

% 特别感谢(尤其感谢我自己)
\chapter*{特别感谢}

\vspace{40pt}

% 使用表格排版感谢
\begin{table}[H]
    \centering
    \begin{minipage}[c][76ex][t]{.5\linewidth}%高度只影响minipage结束后下面的文字开始排版的位置,无论minipage内的内容是否超限都不会提示,所以只需关注底下的图片排版即可
        \centering
        \begin{tblr}[
                tall,
                theme = {no-caption},
                note{$\dag$} = {本栏内所有资料均已获编委(主编)授权。},
            ]{
                vline{2} = {.4pt},%手动设置中间线当作隔断,线需稍宽才能让左右两栏的边线重叠看不出来
                cells={c,m},
                columns={wd=.95\linewidth},%手动设置栏宽为minipage栏宽的0.95倍
                row{1,3,7,11}={abovesep+=12pt,},
            }
            {\large\textbf{审核校对}}                                                          \\
            周大为                                                                             \\
            {\large\textbf{内容改进}}                                                          \\
            \uhref{https://xchb.sdsmu.edu.cn}{山东第二医科大学临床医学院}                      \\
            \uhref{https://xchb.sdsmu.edu.cn}{山东第二医科大学党委宣传部}                      \\
            \uhref{https://pd.qq.com/s/7mekdr5ve}{山东第二医科大学频道}                        \\
            {\large\textbf{图片处理}}                                                          \\
            \uhref{https://affinity.serif.com/zh-cn/designer}{Affinity® Designer}              \\
            \uhref{https://www.gimp.org/}{GIMP}、%
            \uhref{https://krita.org/zh-cn}{Krita}                                             \\
            \uhref{https://www.adobe.com/cn/creativecloud/roc/business.html}{Adobe® Photoshop} \\
            {\large\textbf{项目结构与内容}}\TblrNote{$\dag$}                                   \\
            \uhref{https://zjuers.com/welcome}{%
            浙江大学本科新生指引}                                                              \\
            \uhref{https://github.com/SurviveSJTU/SurviveSJTUManual}{%
            上海交通大学生存手册}                                                              \\
            \uhref{https://github.com/SUSTech-Application/SUSTechapplication}{%
            南方科技大学飞跃手册}                                                              \\
            \uhref{https://github.com/SurviveSJTU/SJTU-Application}{%
            上海交通大学飞跃手册}                                                              \\
            \uhref{https://gitee.com/hanyaner/WITjsj}{%
            武工大计院新生入学建议}                                                            \\
        \end{tblr}
    \end{minipage}%
    \begin{minipage}[c][76ex][t]{.5\linewidth}
        \centering
        \begin{tblr}[
                tall,
                theme = {no-caption},
            ]{
                vline{1} = {.4pt},%同上,只是为了实际居中,理论上一条单线看不出来缺那么零点几毫米不居中
                cells={c,m},
                columns={wd=.95\linewidth},%略
                row{1,4,10}={abovesep+=12pt},
                row{Z}={abovesep+=34pt},
            }
            {\large\textbf{宣传发布}}                                                      \\
            \uhref{https://pd.qq.com/s/7mekdr5ve}{山东第二医科大学频道}                    \\
            \uhref{http://weixin.qq.com/r/mp/Vh0eBqPE8v6Nredl90hE}{闪医Spark Medicine}     \\
            {\large\textbf{格式与排版}}                                                    \\
            \uhref{https://ctan.org/pkg/latexindent}{latexindent.pl}                       \\
            \uhref{https://github.com/James-Yu/LaTeX-Workshop}{LaTeX Workshop}             \\
            \uhref{https://liam.page}{始终(Liam Huang)}                                  \\
            \uhref{https://github.com/lvjr/tabularray}{tabularray}、%
            \uhref{https://gitee.com/nwafu_nan/tabularray-doc-zh-cn}{tabularray-doc-zh-cn} \\
            \uhref{https://github.com/latex3/hyperref}{hyperref}                           \\
            {\large\textbf{疑难问题帮助}}                                                  \\
            \uhref{https://github.com/CTeX-org/forum/issues}{GitHub CTeX Forum}            \\
            \uhref{https://tex.stackexchange.com}{TeX StackExchange}                       \\
            \uhref{https://www.latexstudio.net}{LaTeX工作室}                               \\
            \large\textbf{\textcolor{red}{%
            其他所有为本指南、地图提供建议与改进意见的同学、教师等}}                       \\
        \end{tblr}
    \end{minipage}
\end{table}%特别感谢
% 前言
\chapter[指南简介]{指南简介}

\section[许可证及项目信息]{许可证及项目信息}
山东第二医科大学指南 © 2023-2025 by \textcolor{blue}{$LinkChou$} is licensed under CC BY-SA 4.0. To view a copy of this license, visit \uhref{http://creativecommons.org/licenses/by-sa/4.0}{http://creativecommons.org/licenses/by-sa/4.0}.

Gitee仓库地址:\uhref{https://gitee.com/LinkChou/sdsmu_welcome_tex}{https://gitee.com/LinkChou/sdsmu\_welcome\_tex}

Github仓库地址:\uhref{https://github.com/Mikachu2333/sdsmu_welcome_tex}{https://github.com/Mikachu2333/sdsmu\_welcome\_tex}

\section[声明]{声明}
\label{copyright}
\subsection[严正声明]{严正声明}
本人对且仅对\textbf{“由本人发布的”}、\textbf{“已通过审核的”}、\textbf{“正式版本的”}《山东第二医科大学指南》的言论负有直接责任。

在本项目的衍生项目中,除正式版“授权信息”小节列举的项目以外,由其他维护者(贡献者)添加的内容\textbf{必须经过本人审查后方可通过}。\textbf{未经审查的内容均与本人无关,本人不对其正确性、时效性、思想情况做出任何保证},由其引发的任何问题也不由本人负责,而是由负责该衍生项目的负责人负责。

\subsection[版权声明]{版权声明}
下列作品均由\textbf{周大为(LinkChou)}创作并保留所有权利,转载请标明作者姓名(笔名亦可)与出处。本文其他未做声明的内容均根据鄙人个人生活经验、学校公告及网络公开资料汇总整理并提炼精简,如有错漏敬请指明。

\begin{enumerate}
    \item \textbf{《山东第二医科大学指南》}
    \item \textbf{《山东第二医科大学地图(浮烟山校区)〔矢量版〕》}(鲁作登字-2024-K-00630461)
    \item \textbf{《山东第二医科大学地图(虞河校区)〔矢量版〕》}(鲁作登字-2024-K-00630460)
    \item \textbf{《山东第二医科大学地图(浮烟山校区敏行楼)〔矢量版〕》}(鲁作登字-2024-K-00630462)
\end{enumerate}

\subsection[授权信息等]{授权信息等}
详见正式版的相关内容,此处不做赘述。

其他未尽事宜请联系\uhref{Mailto:LinkChou@yandex.com}{Mailto: LinkChou@yandex.com}。

\section[\textcolor{red}{下载与更新}]{下载与更新}
\subsection{文档}
最新\textbf{正式版}下载地址:\textbf{\textcolor{red}{\uhref{https://docs.qq.com/s/ETcQ-ZFSrSsh6MK9bm773q}{腾讯文档}}} 或 \uhref{https://gitee.com/LinkChou/sdsmu_welcome_tex/releases/latest}{Gitee发行版} 或 \uhref{https://github.com/mikachu2333/sdsmu_welcome_tex/releases/latest}{GitHub发行版}

\textbf{Preview版}\footnotemark:\uhref{https://github.com/Mikachu2333/sdsmu_welcome_tex/actions}{GitHub Actions自动编译} 或 自行使用 $git\ clone$ 后本地编译
\footnotetext{Preview版本仅用于内测,请以正式版内容为准。}

\subsection[地图]{地图}
百度网盘:\uhref{https://pan.baidu.com/s/1JadGDpjB50g_b7P8CgjVjQ?pwd=7v6k}{https://pan.baidu.com/s/1JadGDpjB50g\_b7P8CgjVjQ?pwd=7v6k}

阿里云网盘:\uhref{https://www.alipan.com/s/dZMvgXwkxGp}{https://www.alipan.com/s/dZMvgXwkxGp}

\section[前言]{前言}
\subsection[行文核心]{行文核心}
“\textcolor{red}{\textbf{有备无患、举要治繁}}”为本文核心思想,敬请“\textbf{酌情增减}”。“\textbf{乐道\ 济世}”的校训和“\textbf{严谨、求是、勤奋、进取}”的校风是本指南编纂过程中的思想指引与指路明灯。
%前言

% 校历(需要每学期更新一次)
\chapter[校历]{校历}
\label{calendar}
\noindent\makebox[\textwidth]{
    \centering
    \includegraphics[width=\textwidth]{resources/date/calender.pdf}
}%校历

% 新生入校指南

\chapter[新生报到]{新生报到}

% 入学准备
\section[入学准备]{入学准备}

\subsection[必要证件及物品]{\uul{必要证件及物品}}
\begin{enumerate}
    \item \textbf{\uul{高考准考证、录取通知书\footnotemark}}(必须带!!!)
          \footnotetext{录取通知书共两份,一份自行留存收藏,一份上交。}
    \item \textbf{\uul{高中档案、党(团)组织关系档案\footnotemark}(档案丢失或私自拆封需要全部补办完成后再入学,切记勿忘勿丢勿拆!)}
          \footnotetext{组织关系转接通常在开学后以班级为单位统一处理,请以各班团支书通知为准。}
    \item 身份证原件及其正反面复印件共4份(建议提前自行复印,其中一份需附带本人签名)
    \item 公费医学生需按\uhref{http://www.shandong.gov.cn/art/2021/10/14/art_100623_39304.html}{《山东省医学生公费教育实施办法》}的相关要求携带已正确填写完毕的\\%
          \uhref{http://app.shandong.gov.cn/attach/2021/28/60-1.pdf}{《山东省医学生公费教育协议书》}一式四份到校,无协议书或填写不规范的不能报到\footnotemark
          \footnotetext{详情政策见学校与网站具体要求,此处仅为概述。}
    \item 少量零散现金(100元左右即可)\footnotemark
          \footnotetext{校内各商店、超市、食堂均支持微信、支付宝支付;食堂(餐厅)支持数字人民币及云闪付支付。}
    \item \textbf{证件照红、蓝、白底,1寸各8张、2寸各4张;以及各电子版}(开学后办理证件,如学生证、图书馆借阅证、社团会员证;各类手续,如团关系转接、学生档案转接、学生会入会申请表、体检报告等均需频繁使用;宿舍门禁系统登记时需要提供白底照片电子版)
    \item \textbf{手机及配件}(充电器、充电宝、耳机,4~6根数据线〔长度在0.5m~1.5m均可〕)
    \item U盘(方便在校期间打印文件,避免异地登录QQ、微信泄露信息〔8G左右即可〕)
    \item \textbf{常备药物}(碘酒、创可贴、医用棉签、感冒灵、布洛芬、维生素、藿香正气水、止泻药、止咳药、跌打损伤药等)\textbf{\uul{症状严重务必及时前往校医院就医,切勿自行用药耽误正规治疗}!}
\end{enumerate}

\subsection[选带证件及物品]{选带证件及物品}
\begin{enumerate}
    \item 户口本复印件1份,户口本本人页复印件4份\footnotemark
          \footnotetext{仅少部分院系要求报到时携带,具体要求以录取通知书为准。}
    \item 成绩证明页面的打印版\footnotemark
          \footnotetext{若录取通知书丢失,可通过该材料及身份证、高考准考证联合证明身份。\label{lost_offer}}
    \item 相机以及配套储存卡、读卡器(⚠贵重物品,请妥存)
    \item 平板\footnotemark、\textbf{笔记本电脑}(班长团支、学生会及社团成员必需,其他同学按需)
          \footnotetext{按照既往经验,有25\%的同学前期刷剧追番玩游戏的频率远超学习。}
    \item 病历本、住院证明等(仅因病无法参加军训者必备,具体请参考前文相关说明)
\end{enumerate}

\subsection[推荐生活用品]{推荐生活用品}

\subsubsection[日用品]{日用品}
\begin{enumerate}
    \item 驱蚊花露水、风油精(严禁使用蚊香及电蚊香以防火灾)
    \item 洗面奶、护手霜、防晒霜等护肤品(不宜过多)
    \item 雨伞(2把,⚠易丢)
    \item 洗衣液/洗衣皂/洗衣粉及肥皂盒(建议带盖)、柔顺剂、消毒液
    \item 1.5L~2L的保温小水壶(冬天教学楼饮水机会上冻,提前接热水去自习好一些)
\end{enumerate}

\subsubsection[宿舍用品]{宿舍用品}
\begin{enumerate}
    \item 腰带(军训服尺码偏大)
    \item 袜子、鞋垫(军训用)、内衣内裤和夏秋季换洗衣物(袜子至少10双,内衣内裤至少6套以免背阴面宿舍阴天无法及时晾干)
    \item 毯子或空调被(可以在学校相关通知中查看本年度的配套被褥\footnotemark 价格,用料和质量都比较实在,据自身需求选订选带)
          \footnotetext{一般含棉被(1.5㎏、2.25㎏)、棉褥、枕芯、床垫、暖水瓶、塑料盆各一;被套,床单,枕套,枕巾各二;床单为蓝白格子,尺寸230×120㎝;被套为蓝色,尺寸220×155㎝;枕套、枕巾均为蓝色,枕巾尺寸为75×46㎝;订购后会直接在开学时放到宿舍内,订购教程见\uref{freshman_query}。\label{bedding_set}}
    \item 凉席、床垫\footnotemark(出汗严重或睡眠有特殊需求同学可自行选带)
          \footnotetext{宿舍床铺相对硬。}
    \item 蚊帐(注意:因四角支撑杆可能不全甚至全无,上铺的同学挂蚊帐比较困难;下铺相对方便。如有相关需求可前往校园周边五金店定制相关配件)
    \item 小锁(1把,用于锁柜子)
    \item \textbf{插排、转换器}(宿舍壁插数量少,刚需转换器;为确保安全,推荐购买公牛等知名品牌产品)
    \item 小台灯或手电筒(可用作考前熬夜学习或突然断电的应急措施)
    \item 衣服撑子、衣服夹子、粘钩(用于晾衣服,衣服撑子10个左右,夹子建议大小均有,被子夹6个,小夹子10个左右,粘钩有无均可)
    \item \textbf{几个干净的大型快递纸箱}(因学校的柜子比较脏且容易掉灰,可以先裁开纸箱子铺到自己的柜子里面再放置物品、衣物等以免弄脏)
    \item 樟脑球3个(不宜过多,放在柜子里以防生虫或衣物因为长期阴天而发霉)
\end{enumerate}

\subsection[学习用品]{学习用品}
\begin{enumerate}
    \item 书包(1个,仅用于上集体课)
    \item 手提袋子(1个,要求是耐脏便宜结实,当作书包用,仅用于实验课\footnotemark)
          \footnotetext{解释说明详见\uref{schoolbag}。}
    \item 红蓝铅笔(中华牌很好,相对不容易断。医学实验课上全都是红蓝铅笔绘图)
    \item 订书机及订书钉或回形针(少用,\st{但用一次头疼一次,因为很少有人有……})
    \item 美工刀、剪刀、胶水、双面胶、透明胶(拆快递、贴证件照常用)
    \item 记号笔(给课本封面写名字,课本也很容易丢)
    \item 4/6级英语考试单词书与练习题(4级6级基本都属于考研基础了,\st{但愿大家能用上吧,不过\linebreak[3]二手书摊经常能买到全新的二手书,可见很多人都没怎么学})
\end{enumerate}

\subsection[相关注意事项]{相关注意事项}
\begin{enumerate}
    \item \textbf{\uul{学校主要使用QQ进行联系以及官方通知的下发},且无“勤工俭学部”“兼职部”等学生会部门和社团,如遇类似组织推荐兼职请务必谨慎!}社团详情参见\uref{community_summary},若不确定可联系本班班长或教师确认,谨防上当受骗!
    \item 安卓手机信号强度在学校停电时相对强(新生入学期间因违规电器导致跳闸频繁)
    \item 充电宝、充电器建议买名牌产品,杂牌劣质产品易出现用电事故;曾因杂牌劣质充电器短路导致多次宿舍全楼大跳闸
    \item 小台灯应当可充电或使用电池供电,以免停电期间无法使用
\end{enumerate}%入学准备
% 入学报到
\section[入学报到]{入学报到}

\subsection[地址与快递]{\uuline{地址与快递}}
\subsubsection[浮烟山校区]{浮烟山校区}
地址:山东省潍坊市潍城区望留街道宝通西街7166号山东第二医科大学浮烟山校区

快递地址:与学校地址相同

邮编:261053

取快递/寄快递:菜鸟驿站(宿舍楼\#12正北方约70米处,位置参见\uline{\ref{map_fuyanshan_holistic}}地图)

\subsubsection[虞河校区]{虞河校区}
地址:山东省潍坊市奎文区广文街道胜利东街4948号山东第二医科大学虞河校区

快递地址:与学校地址相同

邮编:261042

取快递/寄快递:北门菜鸟驿站(在校内,位置见\uline{\ref{map_yuhe_holistic}})、东门菜鸟驿站(详见\uline{\ref{common_locations_yuhe}})

\subsection[报到时间]{报到时间}
详见本人录取通知书

\subsection[浮烟山校区报到流程]{浮烟山校区报到流程}
\begin{enumerate}
    \item \textbf{\uuline{在“山东第二医科大学学生之家”公众号进行预报到},并查询自己的宿舍编号,学号,班级,院系等信息}\footnotemark(教程详见\uline{\ref{freshman_query}})
          \footnotetext{公众号查询入口开放日期约为8月20日前后,每年不一。}
    \item 通过学校通知、各级学生会、贴吧或频道创建的官方群内指引或者等待各位带班学长学姐加好友后进入班级群\footnotemark
          \footnotetext{若未能寻找到自己的班级,也可在报到时通过带班学长学姐加入班级官群。}
    \item 根据\uline{\ref{goto_school}}前往山东第二医科大学(浮烟山校区)北校门,在志愿者\footnotemark 的带领下先前往宿舍一楼(各宿舍楼位置见\uline{\ref{map_fuyanshan_holistic}}),在宿管阿姨处领取宿舍钥匙并登记信息
          \footnotetext{注意,\textbf{男女宿舍不互通},请尽量找相同性别的志愿者帮忙;浮烟山校区在新生报到时允许家长入校参观、帮助搬运行李,具体时间及相关政策详询学校。}
    \item 暂时放置行李物品,进行物品的初步整理
    \item 携带本人身份证件、录取通知书\footnotemark 及中性笔一只,前往杏林路D区方位(位置参见\uline{\ref{map_fuyanshan_holistic}})
          \footnotetext{录取通知书丢失处理办法见此\uline{\ref{lost_offer}}。}
    \item 在公安设置的身份核验处查验自己的身份证\footnotemark 信息
          \footnotetext{身份证丢失或失效请立即在学校110警务处申请临时身份证。}
    \item 根据杏林路西侧的指示牌寻找自己班级的带班学长、学姐报到并签字,他们将指引你完成接下来的报到流程\footnotemark
          \footnotetext{如果前往学校的时间过晚或未能及时购票,可前往学院官网,通过官网最下方的联系方式咨询本学院教师解决方案;按照往年惯例,未能及时登记报到的同学会统一补齐相关记录。}
    \item 返回宿舍或前往菜鸟驿站领取并整理自己的被褥等个人物品
\end{enumerate}%入学报到
% 交通指引

\section[交通指引]{交通指引}
\label{goto_school}
\subsection[火车(高铁)]{火车(高铁)}
起点:家乡的火车站

终点:优选“潍坊站”,其次“潍坊北站”(因北站距离学校较远,公交线路\hyperref[bus]{见此})

提示:在购票窗口出示你的录取通知书(不可使用复印件)后,高铁打七五折,火车票五折

出站口选择:
\begin{enumerate}
    \item 潍坊站:如需学校迎新官方接送或乘坐公交请选择“北出站口”,打车请往“南出站口”;需要前往虞河校区请走“北出站口”
    \item 潍坊北站:请先乘25或106路公交至潍坊站,再按上面的指引前往
\end{enumerate}

\subsection[大巴(公交)]{大巴(公交)}

\subsubsection[官方接送]{官方接送}
学校每年都会在潍坊火车站、潍坊汽车站设立咨询服务点,有专车接送新生到浮烟山校区。接待点内的工作人员均会佩戴山东第二医科大学校徽及其他相关标识

\subsubsection[自行前往]{自行前往}
\label{bus}
\begin{enumerate}
    \item 浮烟山校区
          \begin{enumerate}
              \item 在潍坊站下火车的同学可从“北出站口”出站,乘坐69、71或101路公交车在“山东第二医科大学北门”站下车(预计30~50分钟左右)
              \item 在潍坊北站下火车的同学可先乘坐25或106路公交车前往火车站(潍坊站),再以上述方式前往报到(潍坊北站至潍坊站约需2小时)
          \end{enumerate}
    \item 虞河校区
          前往虞河校区可从“北出站口”出站,乘坐13路或75路公交车,在“鸢飞路胜利街路口北”站下车(预计20~30分钟左右)
\end{enumerate}

\subsection[出租车]{出租车}
浮烟山校区:建议乘坐正规出租车,地点为“山二医新校区北门”,正规出租车打表价格约25元(预计时间20~25分钟)

虞河校区:地点为“山二医老校区(虞河校区)北门”,打表价格约20元(预计时间15~20分钟)

\subsection[浮烟山校区新生报到提示]{浮烟山校区新生报到提示}
浮烟山校区南门提供摆渡车(观光车)且有志愿者帮助运送行李,但因距离宿舍较远、搬运行李不便,推荐走北门报到(报到结束后,可乘观光车在南门处拍照留念)%交通指引

% 军训
\section[军训]{军训}

\subsection[军训概况]{军训概况}
\begin{enumerate}
    \item 军训时间:3周,报到后马上开始
    \item 内容\footnotemark:站军姿、蹲下、跨立、齐步走、原地踏步、齐步跑、踢正步、坐马扎
          \footnotetext{不打靶、不摸枪;少数同学会参与战术方阵,女生有擒敌拳。}
    \item 地点:多数在杏林路树荫下军训(一般晒不着,但仍然建议多备防晒霜)
    \item 教官:按往年情况,一般由部队官兵担任
    \item 着装:穿着统一的军训套装\footnotemark
          \footnotetext{同被褥套装订购教程请看\uref{freshman_query}。含T恤、外套、裤子(很肥)、腰带(扎在外套外面,需另备一条腰带以免掉裤子)、鞋子(底薄、磨脚,推荐鞋子大一号多垫几双鞋垫)和马扎(军训完不要扔,考试月背书很好用)。}
    \item \textbf{如果因为身体原因不能参加军训,需要提前拿病历、住院证明等材料前往校医院开具相关证明\label{exercise_unattend}}(位置见\uref{map_fuyanshan_holistic})
    \item 若出现身体不适要及时向教官报告,较严重的可申请前往校医院或附属医院进行详细检查
    \item 特点:军训期间时间紧、任务重,少有可以自由安排的时间
    \item 注意:\textbf{\uul{可能出现长时间阴天}},因此推荐多带袜子、内裤、内衣、T恤以避免一件衣服穿一周的窘境
\end{enumerate}

\subsection[军训作息时间表]{军训作息时间表\footnotemark}
\footnotetext{以2024级作息为例,具体以各学院安排为准。}
\begin{table}[H]
    \centering
    \begin{tblr}[
            theme = {no-caption},
        ]{
            cells = {c,m},
            hlines,
            vlines,
        }
        06:20        & 起床 & 12:20~13:50 & 午休   \\
        06:50~07:20 & 早操 & 14:20~17:20 & 集训   \\
        07:25~07:50 & 早餐 & 17:30~18:30 & 晚餐   \\
        08:00~11:20 & 集训 & 18:30~21:30 & 晚自习 \\
        11:30~12:00 & 午餐 & 22:30        & 熄灯   \\
    \end{tblr}
\end{table}
%军训

% 费用
\section[费用、ATM与银行卡]{费用、ATM与银行卡}

\subsection[生活费]{生活费}
\begin{enumerate}
    \item 下述额外支出含:\textbf{书费等费用}、\textbf{水果蔬菜}、\textbf{空调费用}、\textbf{喝纯净水}、健身运动、零食饮料、各类聚餐团建、游玩、网购(任意物品)等
    \item 一般生活费极限底线约900元(仅吃饭,无其他任何额外支出\footnotemark)
          \footnotetext{\textbf{注意:该标准仅供参考、极不推荐且不含任何突发事件的弹性缓冲空间!}本标准下可供参考的饮食方案(不构成医疗建议):早餐8元、午餐12元、晚餐10元,可能偶有超限。}
    \item 常规生活标准约1200-1500元,基本涵盖额外支出部分,书费(详见\uref{tuition})等特殊支出可能仍会使生活费超限
    \item 虞河校区物价稍高,各标准需上浮约200元;其他见习医院不做说明
\end{enumerate}

\subsection[学费]{学费\footnotemark}
\footnotetext{学费缴费系统使用教程详见\uref{fee_pay},财务处地址:行政楼1层西侧。}
\label{tuition}
\begin{enumerate}
    \item 学费通常分为四部分\footnotemark
          \footnotetext{详见财务处文件\uhref{https://cwch.sdsmu.edu.cn/_upload/article/files/47/c4/47b017924d219d29d49dd079335a/f04e22ac-aa91-4c49-bceb-f53280c93164.pdf}{《学分制收费管理办法》}、\uhref{https://cwch.sdsmu.edu.cn/_upload/article/files/ce/67/093a0208463a8a5bb333e3117fc6/dfb39999-8683-49e7-b8b0-0ed7af8a6150.pdf}{《关于印发山东省高等学校住宿费收费管理办法的通知》}等。}
          \begin{enumerate}
              \item 专业注册学费:不同专业不一,按学年收取,临床为每年3500元
              \item 学分学费:根据各人选修课情况收费,按学期收取,每学分100元
              \item 住宿费:不同校区、楼栋标准不一,按学年收取
              \item 书费:订书方式多样,费用与收费时间均不确定;通常是第一学期交两次钱款(本学期与下学期),之后的每学期提前交下学期书费;请以班级群内通知为准
          \end{enumerate}
    \item 依惯例,新生入学时需要进行缴纳第一学期的专业注册学费与住宿费,请按录取通知书说明以及新生预报到系统\footnotemark 的提示进行,首次学分学费缴纳将于9月末第一次选课结束后收取
          \footnotetext{预报到系统使用教程参考\uref{freshman_query}。}
    \item 学费的收缴工作在开学后按照学校财务处通知进行,通常以班级为单位进行通知,可开具电子发票
    \item 公费医学生与助学贷款学生无须缴纳专业注册学费、学分学费、住宿费3类费用
    \item 申请助学贷款、生源地贷款等有特殊情况的同学详询财务处收费管理科(联系方式详见官网)
    \item 选修课学费按照学分进行收费,所以同一个班的同学学费亦可不同,个人学费以\textbf{“山东第二医科大学财务处”公众号(同“山东第二医科大学校园统一缴费平台”)}中的数据为准
    \item 选修课多选课多交钱,少选课少交钱(希望大家如果看到了自己希望进一步学习的课程不要吝啬那几百块钱,选课机会只有一次,课程不会重开!)
\end{enumerate}

\subsection[助学项目与困难补助]{助学项目与困难补助}
\subsubsection[助学项目一览]{助学项目一览}
敬请参照山东省教育厅学生资助管理中心\uhref{https://sdxszz.sdei.edu.cn/Show/7784}{《山东省普通高等学校资助政策简介》}或咨询学校教师
\subsubsection[困难补助一览]{困难补助一览}
详见学校下发的《山东第二医科大学新生报到指南》

\subsection[ATM]{ATM\footnotemark}
\footnotetext{根据业务动态调整,详情见各银行官网。}
\begin{enumerate}
    \item 工商银行:教学楼E区,靠近杏林路侧
    \item 邮政银行:中和广场,A103对过
\end{enumerate}

\subsection[银行卡]{银行卡}
学校统一为各位同学办理一张中国建设银行卡(每年银行可能不同),随录取通知书发放。该卡包含学生个人信息,将用于大学生活中用于奖助学金等其他费用的发放,极其建议激活此卡

\textbf{\uul{严禁出借银行卡,否则可能导致违法犯罪!!!}}
%费用相关

% 新生常见问题
\section[常见问题]{常见问题}
\begin{enumerate}
    \item 第一临床医学院与临床医学院的区别\\
          “第一临床医学院”为学校与第一附属医院(潍坊市人民医院)合署招生,实习与见习将定向至该医院。新生将随机分配至临床医学院与第一临床医学院。
    \item 宿舍分配\label{random_allocation}\\
          无规律,宿舍以专业为单位随机分配宿舍楼、班内宿舍按姓氏顺序分配,与录取分数无关;宿舍床位为自行挑选,如有不妥可在辅导员知情同意的情况下宿舍内协商解决
\end{enumerate}


%其他常见问题%新生报道

\input{in_school/content_fuyanshan.tex}%浮烟山章节
% 虞河校区目录

\chapter[虞河校区]{虞河校区}

% 校区概况

\section[校区概况]{校区概况}
\begin{enumerate}
    \item 仅有部分专业的部分年级
    \item 有北、东两个校门,开放式校园\footnotemark
          \footnotetext{北门24小时开放,东门开门时间为?。}
    \item 宿舍条件较差,无独卫独浴、限电(但不停电),宿舍按时关门
    \item 教学建筑:教学楼3座,实验楼1座,校史馆1座
    \item 生活建筑:餐厅1个,浴室(地下)男女各一,小超市2个
    \item 接近市中心,周围可供饮食、娱乐、购物等,应有尽有
    \item 学校内建有整形医院、生殖中心、眼科等科室,人员流动频繁
\end{enumerate}%总述
% 地图
\newpage
\section[地图]{地图}

\subsection[整体(虞河校区)]{整体(虞河校区)}
\begin{figure}[H]
    \centering
    \vspace{5em}
    \includegraphics*[width=\linewidth]{resources/map/虞河校区.pdf}
    \label{map_yuhe_holistic}
\end{figure}

\newpage
\subsection[整体(人民医院)]{整体(人民医院)}
\begin{figure}[H]
    \centering
    \vspace{8em}
    \includegraphics*[width=\linewidth]{resources/map/人民医院.pdf}
    \label{map_yuhe_renmin_hospital}
\end{figure}

\newpage
\subsection[整体(附院)]{整体(附院)}
\begin{figure}[H]
    \centering
    test
    %\vspace{8em}
    %\includegraphics*[width=\linewidth]{resources/map/副院.pdf}
    %\label{map_yuhe_renmin_hospital}
\end{figure}

\newpage%地图及人民医院地图
% 宿舍
\section[宿舍条件]{宿舍条件}

\subsection[整体情况介绍]{整体情况介绍}
\begin{enumerate}
    \item 本科宿舍无阳台,无独立卫生间,无独立浴室,面积较浮烟山校区小
    \item 宿舍内有4个上下床(但只住6人)
    \item 宿舍楼06:00开门,23:00关门
    \item 宿舍不断电但限制功率\footnotemark
          \footnotetext{每插座约900W,每宿舍共两个插座接口,且\textbf{无地线}。}
    \item 公共厕所每层一个,晾衣服也在此处
    \item 饮水机仅单数楼层有,位于公厕旁,为自来水净化后烧开,24小时供应
    \item 学校洗浴中心在3号楼负1层(从北侧进入),各宿舍楼内无洗浴设施,浴位少;使用教程\uref{shower_software_y}{见此})
    \item 其余见浮烟山校区相关章节
\end{enumerate}

\subsection[住宿注意事项]{住宿注意事项}
\begin{enumerate}
    \item 学校为全开放式校园,无出入限制,\textbf{请务必锁门,谨防失窃!}
    \item 宿舍隔音效果较差,\textbf{严禁在走廊大声喧哗!}
    \item 因影响他人休息,\textbf{禁止在12:00~14:00、23:00~07:00打篮球!}
\end{enumerate}%宿舍
% 生活
\section[学校生活]{学校生活}
\begin{enumerate}
    \item 购物:学校距离泰华、万达均较近
    \item 吃饭:虞新街、西虞巷以及周边有诸多餐厅
    \item 自习:因座位短缺,仅可在自己班教室自习
    \item 交通:详见\uline{\ref{free_bus}}处的相关说明
    \item 其他特殊说明:虞河校区人民医院科教楼东侧的餐厅晚上暂停营业
\end{enumerate}
%学校生活%虞河章节

%安全
\chapter[安全方面]{安全方面}

\section[用电安全]{用电安全}
\begin{enumerate}
    \item 使用安全的充电器、充电线、插排
    \item 如遇空调、电灯、电风扇等器具出现短路、跳闸、停电等现象时切勿自行维修,请按照\uline{\ref{repair_report}}教程处理
    \item 切勿在宿舍使用冰箱(断电且宿管检查),\textbf{\uuline{如果各位同学需要冷藏保存药物(如胰岛素等),请\\前往校医院或大服2楼的药店处}}(地点参见\uline{\ref{common_locations}}),具体收费情况请咨询相关人员
    \item 切勿私改电路,如确有需要,应提前向宿管及公寓管理委员会\footnotemark 报备并获得相应许可
          \footnotetext{位于2号公寓南门处。}
    \item 宿舍人走灭灯,无人则关电、切断插座电源
\end{enumerate}

\section[防火安全]{防火安全}
\begin{enumerate}
    \item 切勿使用蚊香以免发生火灾
    \item 切勿在宿舍内烹饪,禁止在宿舍使用各类燃气灶、固体酒精便携灶等易燃易爆危险品
    \item \textbf{禁止在宿舍内吸烟},若实在无法抑制请前往本楼层公共厕所或在宿舍外吸烟完毕后再进宿舍
    \item 请妥善保管打火机、打火石、镁条、火柴等易燃物
\end{enumerate}

\section[出行安全]{出行安全}
\begin{enumerate}
    \item 学校周边基础设施尚在完善建设过程中(\sout{属实是兔葵燕麦、雨井烟垣}),且频繁有货车高速通过。如无特殊情况,\textbf{尽量不要骑公共自行车或者电动车去市里}(泰华城)等,推荐乘坐71路公交车(预计行程30分钟左右)或打车前往
    \item 货车转弯盲区大,极其容易发生安全事故,在等红绿灯时请务必远离“站立禁止区域”
    \item 乘坐出租车(尤其是拼车时)请妥善保管自身财物;如遇失窃请尽快报警,避免正面冲突(谨防持刀伤人)
    \item \textbf{\uuline{严格遵守交通规则,仔细观察周围情况,切忌边看手机边前进,切忌闯红灯}!}
\end{enumerate}

\section[食品安全]{食品安全}
\begin{enumerate}
    \item 如果在餐厅吃坏了肚子或者发现食物质量问题,可前往餐厅一楼东北角的值班室寻求工作人员的帮助
    \item 出现急性腹泻、血便、米泔水样腹泻等情况请尽快前往校医院就诊,切勿拖延
\end{enumerate}

\section[防诈骗及其他注意事项]{防诈骗及其他注意事项}
\begin{enumerate}
    \item \textbf{\uuline{校园贷毁一生,远离高利贷}!}
    \item \textbf{\uuline{杜绝黄赌毒!不要高估自己的意志力}!}
    \item \textbf{刷单就是诈骗!}
    \item 不要贪小便宜乱扫码,信息泄露吃大亏!
    \item \textbf{\uuline{在正式开学、由带班学长学姐拉入年级的官方QQ群之前,所有的主动拉人入群的各类“通\\知群”“官方群”“学生公告群”等均不可信}!}
    \item 如果碰到一些人自称是市场营销专业、经商专业的,需要卖笔卖本子\footnotemark (总之是找人要钱)才能完成期末考试的,千万不要相信!可以直接联系保卫处
          \footnotetext{一支批发0.1$¥$的笔卖10$¥$呢,比百乐斑马这种外国牌子都贵,利润高达10000\%……}
    \item 女生晚上尽量不要独自前往人烟稀少的地方,尤其是西门附近的桃李路等
    \item 谨防诈骗,\textbf{学校永远不会以任何名义,以邮件表格或短信链接的形式通知填写银行卡取款密码!}绝对不要相信以“更新银行卡信息”、“填表申请助学金”为由窃取密码以及验证码的骗局!如果不确定消息是否属实请致电本班班长,班主任或学工办老师确认
    \item 各位家长在向学生转账时应\textbf{确认钱款具体用途、打款账户是否正确},同时应当\textbf{电话询问是否属实},谨防盗号诈骗
\end{enumerate}
%安全

% 学习
\chapter[学习方面]{学习方面\footnotemark}

\footnotetext{本节内容参考自以下文件:山二医教字〔2025〕4号《山东第二医科大学普通全日制本科学生学士学位授予工作办法(修订)》、山二医教字〔2025〕5号《山东第二医科大学考试工作管理办法(修订)》、山二医教字〔2025〕6号《山东第二医科大学课程重修管理办法》、山二医教字〔2025〕7号《山东第二医科大学本科生学业预警管理办法(修订)》、山二医教字〔2025〕12号《山东第二医科大学临床医学类专业“三阶段”综合考试实施方案(修订)》、教务处《山东第二医科大学本科专业人才培养方案(2021版)》、教务处《山东第二医科大学2024年普通全日制本科学生转专业工作方案》等。因文章长度所限此处无法一一列举,不甚清晰之处敬请参阅各文件的对应章节。}

\section[成绩、绩点与学分]{成绩、绩点与学分}

\subsection[成绩]{成绩}
\label{score}
注:本节所用的“成绩”一词如无特殊说明均指代\textbf{“综合成绩}”。
\begin{enumerate}
    \item 各科成绩满分为100分,及格为60分
    \item 成绩由“平时成绩”与“期末成绩”组成,此两项满分均为100分,及格均为60分;最终加权得到综合成绩,详见下表
    \item 非限选课的“划线”(即,计算综合成绩时计入平时成绩所需的最低期末考试分数)由任课教师及相关教研室决定
    \item 成绩及格即可取得该科的学分与绩点,否则需要补考或重修
\end{enumerate}

\noindent\makebox[\textwidth][c]{\includegraphics[width=\textwidth]{resources/sundry/单科成绩计算.pdf}}

\vspace*{-15pt}

\subsection[绩点]{绩点}
\label{gpa} % grade point average
\begin{enumerate}
    \item \textbf{平均学分绩点(GPA) ≥ 2}是获得学位证的要求之一
    \item 绩点计算方式如下:
          \begin{enumerate}
              \item \textbf{课程绩点}(综合成绩不及格,绩点记为0分):
                    \begin{equation}
                        \frac{课程综合成绩}{10} - 5
                    \end{equation}
              \item \textbf{平均学分绩点(GPA)}(平均绩点):
                    \begin{equation}
                        \frac{\sum (课程绩点 \times 课程学分)}{\sum 课程学分}
                    \end{equation}
          \end{enumerate}\pagebreak[3]
    \item 补考合格的科目按60分计,换算后为1绩点,其他特殊情景参考\uref{retake}小节
\end{enumerate}

\subsection[学分]{学分}
\begin{enumerate}
    \item 需修满对应的学分方可授予学位证
    \item 学分还用于学费计算(每分100元,其余不再赘述)
    \item 特殊事项:“公共选修课联盟”的公共选修课程不算学分、不收学费
    \item 各类学分要求详表\footnotemark 如下图所示
\end{enumerate}
\footnotetext{因不同学制、学院、年级要求各不相同,本图仅以\textbf{2021级临床医学院临床医学系普通5年制本科}为例,其他内容请参阅对应的“培养方案”。}

\noindent\makebox[\textwidth][c]{%
    \includegraphics[width=\textwidth]{resources/sundry/学分.pdf}%
    \label{credit}
}

\subsection[留学、国际交流]{留学、国际交流}
如需出国留学或进行国际交流时该校要求出示平均绩点等相关证明,请咨询教务处相关事宜。

\section[缓考、补考与重修]{缓考、补考与重修}
\label{retake}

\subsection[补考]{补考}
\begin{enumerate}
    \item 必修课、限定选修课不及格(50 ≤ 分数 < 60)可以申请补考1次,仅计算卷面成绩,若及格则按60分计入综合成绩(折算为1绩点)
    \item 非限选课、公共选修课不及格需改选或重选,不予补考
    \item 每门课程仅1次补考机会(缺考除外),不额外收费
\end{enumerate}

\subsection[重修]{重修}
\begin{enumerate}
    \item 缓考或补考不及格者需要重修对应科目
    \item 必修课、限定选修课不及格(分数 < 50)需重修对应科目
    \item 非限选课、公共选修课不及格可以选择重修或直接改选其他科目
    \item 最终分数以实际为准(计入新的平时分)
\end{enumerate}

\subsection[缓考]{缓考}
缓考科目的分数以考试为准(计入平时分),缓考申请教程参见《山东第二医科大学指南》,此处不再赘述。

\section[关于学位证、毕业]{关于学位证、毕业}
\begin{enumerate}
    \item \textbf{存在以下情况将无法获得学位证}
          \begin{enumerate}
              \item 修不够每项对应的学分
              \item 平均绩点 < 2分
              \item 受过记过及以上处分
              \item 其他不符合相关要求的情况
          \end{enumerate}
    \item 留级:累计两次获得三级学业预警的将会留级
    \item 无法毕业:三阶段考试最终分数不及格
\end{enumerate}

\section[班级综测]{班级综测}
\label{class_evaluation}
\begin{enumerate}
    \item 综测是综合素质测评的简称,一般包含学习、活动、比赛等大类,加权相加后为得分,各年级要求不一,详见学生手册
    \item 班级综测是用于考核各班表现的评判指标,与班级评优评先、班集体评优评先名额分配、见习点分配\footnotemark 强相关,每学期计算一次,截至见习
          \footnotetext{以截至见习之前的各学期平均班级综测成绩计算班级排名,公费班级根据本年级政策进行。}
    \item 班级综测直接决定本班级同学后期学习和生活所在的医院规模、等级和生活条件,请大家务必重视\\
          例如:宿舍有无空调、暖气、洗衣机,是否提供插座,是否有早操和晚查寝\footnotemark 等
          \footnotetext{这意味着能否在外租房居住。}
\end{enumerate}

\section[早操、跑步打卡与晚自习]{早操、跑步打卡与晚自习}
\begin{enumerate}
    \item 早操时间一般为06:00~07:00,晚自习时间通常为19:00~21:00,教室22:00关闭
    \item 早操与晚自习贯穿整个大一\footnotemark,并且跑操与晚自习均有院系学生会不定时进行抽查出勤率等指标,最终结果计入班级综测(详见\uref{class_evaluation})的评分
          \footnotetext{按照惯例,麻醉专业无早操,只需要大一、大二早晨七点签到。}
    \item 学校已开始试行跑步打卡,详见相关通知
\end{enumerate}

\section[其他事项]{其他事项}

\subsection[转专业]{转专业\footnotemark}
\footnotetext{注:本小节依据教务处2024年5月21日发布的《山东第二医科大学2024年普通全日制本科学生转专业工作方案》简化。该规定每年不一,如有变动,以教务处政策为准。}

\subsubsection[转专业条件]{转专业条件}
\begin{enumerate}
    \item 取得学籍的普通全日制一年级本科在校生
    \item 未受处分、思想过硬、符合体检要求
    \item 第一学年必修课和专业限定选修课平均成绩在本专业排名前30\%,且未挂科
    \item 以特殊招生形式录取的学生,国家有相关规定或者录取前与学校有明确约定的(如公费医学生等),不得申请转专业
\end{enumerate}

\subsubsection[考试科目]{考试科目}
思政、英语\footnotemark、数学,比例为1:1:1,考试时间150分钟,总分为150分
\footnotetext{高考小语种考生可选择小语种替代英语,需在报名时备注,不备注者参加英语考试。}

\subsubsection[流程]{流程}
\begin{enumerate}
    \item 核算学生第一学年学习成绩并公示各类数据
    \item 8月31日开始报名
    \item 9月5日组织考试(地点详见相关通知)
    \item 9月6日~10日公示录取名单\footnotemark
          \footnotetext{按照“分数优先,遵循志愿,一次投档”的原则进行录取,成绩排名在专业应考人员中居前20\%者取得转专业资格。}
    \item 9月11日~12日报到
\end{enumerate}

\subsection[关于选修课的补充说明]{关于选修课的补充说明}
\begin{enumerate}
    \item 选修课分专业选修和公共选修两大类(详见学分组成表\uref{credit}),\textbf{推荐在大一全年、大二上学期就把各类选修学分全都修满},这样就不用在后面学业愈重的情况下兼顾选修课的学习了,可以专心针对专业课程进行深入学习
    \item 专业选修课有限选和非限选之分,限选的课程无需操心,教务系统会自动选课,只需要保证非限选的课程学分达标即可
    \item 公共选修课每一类都要选至少一门,且需要满足总分,其中部分类别还有额外要求(详见学分组成表),国防教育类有国家限选课程(到时候看具体通知,会说的很明白的)
    \item 公共选修课有一部分是在教室上的,还有一部分是线上课程(使用“知到”app进行学习,大多数有平时分),可以根据自己的实际情况选择
\end{enumerate}

\subsection[自习禁止事项]{自习禁止事项}
\begin{enumerate}
    \item \textbf{禁止在教室内亲嘴、喧哗、频繁说话}
    \item \textbf{禁止在自习室内抖腿}
    \item 禁止外放音乐、视频等
    \item 禁止在大服、非本班上课常用教室占座
    \item 禁止在教室内食用气味浓的食物(例如榴梿、辣条等)
    \item 禁止长时间占用教室内电源插座,仅允许应急充电
    \item 禁止在教室以及教室旁厕所内吸烟
\end{enumerate}

\section[临床医学专业三阶段综合考试]{临床医学专业三阶段综合考试}
大一的基础课程是一切的根基。

因临床医学专业无毕业论文,为检测学生阶段知识掌握情况,学校实行了临床医学专业三阶段综合考试(下简称“三阶段考试”)的方法。此外,三阶段考试如果最终综合分数不及格,将无法毕业;若二阶段考试不及格,将无法进入医院实习;其他规定详见本节最后的\uref{other_rules_exam}小节。

\subsection[第一阶段综合考试(基础医学综合考试)]{第一阶段综合考试(基础医学综合考试)}
\begin{enumerate}
    \item 考察内容:组织胚胎学、病理学、病理生理学、解剖学、生理学、生物化学、药理学、医学免疫学、医学微生物学
    \item 考查时间与方式:第6学期进行,机考或笔试
    \item 其他信息:时长2.5小时,共300题,满分100分
\end{enumerate}

\begin{tblr}[
        long,
        caption = {一阶段考试详表},
    ]{
        cells = {c,m},
        rowhead = {1},
        row{1,Z} = {cmd=\bfseries},
        hline{1-2,Z} = {-}{1pt},
    }
    考试内容     & A1型题   & A2型题   & B1型题   & 总计     \\
    组织胚胎学   & 45~50\% & 35~40\% & 15~20\% & 5~10\%  \\
    病理学       & 45~50\% & 35~40\% & 15~20\% & 10~15\% \\
    病理生理学   & 45~50\% & 35~40\% & 15~20\% & 10~15\% \\
    解剖学       & 45~50\% & 35~40\% & 15~20\% & 10~15\% \\
    生理学       & 45~50\% & 35~40\% & 15~20\% & 10~15\% \\
    生物化学     & 45~50\% & 35~40\% & 15~20\% & 10~15\% \\
    药理学       & 45~50\% & 35~40\% & 15~20\% & 10~15\% \\  %避免自动换行不对劲导致出现vbox warning
    医学免疫学   & 45~50\% & 35~40\% & 15~20\% & 10~15\% \\
    医学微生物学 & 45~50\% & 35~40\% & 15~20\% & 10~15\% \\
    总计         & 45~50\% & 35~40\% & 15~20\% & 100\%
\end{tblr}

\subsection[第二阶段综合考试(临床医学综合考试)]{第二阶段综合考试(临床医学综合考试)}
\begin{enumerate}
    \item 考察内容:
          \begin{enumerate}
              \item 理论考试:
                    \begin{enumerate}
                        \item 基础医学:病理学、解剖学、生理学、生物化学、药理学、医学免疫学、医学微生物学
                        \item 临床医学:内科学、外科学、妇产科学、儿科学
                        \item 医学人文:医学伦理学、医学心理学
                        \item 预防医学:预防医学
                    \end{enumerate}
              \item 技能考试:病史采集、体格检查、基本操作技能、沟通交流能力、人文关怀
          \end{enumerate}
    \item 详情考试形式、考试时间、试题数量以“国家医学考试中心”通知为准
    \item 下附往年二阶段理论考试详表
\end{enumerate}

\begin{tblr}[
        long,
        caption = {二阶段理论考试详表},
    ]{
        cells = {c,m},
        rowhead = {1},
        row{1,Z} = {cmd=\bfseries},
        hline{1-2,Z} = {-}{1pt},
    }
    考试内容 & A1型题   & A2型题   & B1型题   & 总计     \\
    基础医学 & 45~50\% & 35~40\% & 15~20\% & 40~45\% \\
    临床医学 & 20~25\% & 70~75\% & 3~5\%   & 40~45\% \\
    医学人文 & 45~50\% & 35~40\% & 15~20\% & 5~10\%  \\
    预防医学 & 35~40\% & 55~60\% & 3~5\%   & 5~10\%  \\
    总计     & 30~35\% & 50~55\% & 10~15\% & 100\%
\end{tblr}

\begin{tblr}[
        long,
        caption = {二阶段实践考试详表},
        note{1} = {沟通能力、人文关怀等医学人文素养的考核融合到各站,分值约占15\%。},
    ]{
        cells = {c,m},
        rowhead = {1},
        row{1,Z} = {cmd=\bfseries},
        hline{1-2,Z} = {-}{1pt},
    }
    考站 & 考试内容\TblrNote{1} & 考试方式 & 考试时间 & 分值 \\
    1站  & 病史采集             & SP       & 10分钟   & 20   \\
    2站  & 病史采集             & SP       & 10分钟   & 20   \\
    3站  & 体格检查             & 操作     & 10分钟   & 15   \\
    4站  & 体格检查             & 操作     & 10分钟   & 15   \\
    5站  & 基本操作             & 操作     & 10分钟   & 15   \\
    6站  & 基本操作             & 操作     & 10分钟   & 15   \\
    合计 &                      &          & 60分钟   & 100  \\
\end{tblr}

\subsection[第三阶段综合考试(毕业综合考试)]{第三阶段综合考试(毕业综合考试)}
\begin{enumerate}
    \item 考察内容:
          \begin{enumerate}
              \item 理论考试:内科学、外科学、妇产科学、儿科学
              \item 技能考试:职业素质、病史采集、体格检查、基本操作、辅助检查、病例分析
          \end{enumerate}
    \item 考查时间与方式:实习结束、毕业前进行,笔试或机考、面试
    \item 其他信息:理论考试时长2小时,共200题,满分100分;技能考试时长1小时,形式为客观结构化临床考试(OSCE),共12~16站,满分100分
\end{enumerate}

\begin{tblr}[
        long,
        caption = {三阶段理论考试详表},
    ]{
        cells = {c,m},
        rowhead = {1},
        row{1,Z} = {cmd=\bfseries},
        hline{1-2,Z} = {-}{1pt},
    }
    考试内容 & A1型题   & A2型题   & A3型题   & 总计     \\
    内科学   & 45~50\% & 25~30\% & 25~30\% & 35~40\% \\
    外科学   & 45~50\% & 25~30\% & 25~30\% & 35~40\% \\
    妇产科学 & 45~50\% & 25~30\% & 25~30\% & 10~15\% \\
    儿科学   & 45~50\% & 25~30\% & 25~30\% & 10~15\% \\
    总计     & 30~35\% & 50~55\% & 10~15\% & 100\%
\end{tblr}

\subsection[相关规定]{相关规定}
\label{other_rules_exam}
\begin{enumerate}
    \item 补考:一阶段由基础医学院组织补考;二阶段由教务处组织考试;三阶段由临床医学院组织补考
    \item 成绩折合与毕业:各阶段理论考试成绩按照2:4:4的比例折算后,作为毕业考试理论成绩;二、三阶段技能考试成绩按5:5折算后作为毕业考试技能成绩;不及格者不予毕业
    \item 综测与奖学金:一、二阶段成绩纳入个人综测和奖学金评定
    \item 实习:二阶段合格者可以进入医院实习,否则需要通过实践教学管理处和临床医学院组织的实习准入考试
\end{enumerate}

\section[杂项]{杂项}
\begin{enumerate}
    \item \textbf{大一不允许参加四级考试,大二才能报名四六级}
    \item 关于奖学金:本校有国家奖学金、校长奖学金、校级3等级奖学金等\\
          请注意,目前学校仅能通过本人身份开通的、已开户的、账户已激活的、无异常的工商银行储蓄卡发放奖学金,详情政策可咨询学校财务处。
    \item 新生开学考试\footnotemark 的内容为高中英语、高中数学,旨在让各位同学收心
          \footnotetext{按照往年惯例,不公布具体成绩。}
    \item \textbf{关于档案填写:}入档资料具有不能涂改、不能标记、不能修正的特性,因此严禁使用修正液或修正带对其涂改(涂改修正后立即作废)。填表时务必确保各类时间填写正确、与原始档案一致。此外,建议初次填写时使用铅笔轻轻填写,待负责人确认无误后方可擦除后使用黑色中性笔或中油笔填写(入档纸张为特殊A4纸,厚、重、滑,难以自行复印)
    \item 关于实验课与实验服(白大褂):在实验课上课时务必按要求正确洗手并佩戴头套、口罩、手套,着实验服,\textbf{\uul{严禁任何人在任何其它地点(尤其是餐厅)着实验服}!}因在实验室中需频繁接触实验动物、微生物,\textbf{极其推荐将实验课书包与普通课书包分开}!\label{schoolbag}
    \item 关于实验课:因为敏行楼实验室门牌号极其混乱,因此极其推荐提前20~30分钟前往确认教室地点,详情参见\uref{map_fuyanshan_minxing}处的敏行楼地图
\end{enumerate}
%学习方面
% 就业创业
\chapter[就业创业]{就业创业}

\section[考研]{考研}
\section[规培]{规培}
\section[学士学位]{学士学位}
\section[考公考编]{考公考编}
\section[药企]{药企}
\section[科研助理]{科研助理}
\section[征兵入伍]{征兵入伍}
\section[国家计划]{国家计划}
\subsection[三支一扶]{三支一扶}
\subsection[西部计划]{西部计划}
\section[自由职业]{自由职业}
\section[其他]{其他}%就业创业
%???与竞赛
\chapter[???与竞赛]{???与竞赛}
\section[???]{???}
\subsection[社会实践]{社会实践}
\subsubsection[三下乡]{三下乡}
\subsubsection[青鸟计划(返家乡)]{青鸟计划(返家乡)}
\subsection[???]{???}
\subsubsection[文化艺术节]{文化艺术节}
\subsubsection[科技文化月]{科技文化月}
\subsubsection[百团纳新]{百团纳新}
\subsubsection[宿舍文化]{宿舍文化}
\section[竞赛]{}
\subsection[大创]{大创}
\subsection[实践技能]{实践技能}
\subsection[泰山杯]{泰山杯}
\subsection[组胚]{组胚}

%???与竞赛

% 通用指南目录

\chapter[通用教程与常用信息汇总]{通用教程与信息汇总}

% 教程
\section[常用教程]{常用教程}

\subsection*{特别说明}
\begin{enumerate}
    \item 本节中所有上标“㊕”的网址仅在连接校园网时可成功访问相关服务。
    \item \textbf{严禁使用QQ、微信直接打开本文提及的一切网址,必须使用正常更新的主流浏览器打开!}
\end{enumerate}

\subsection[新生信息查询]{\uuline{新生信息查询}}
\label{freshman_query}
\begin{enumerate}
    \item 关注“山东第二医科大学学生之家”公众号或登录\uhref{https://zhxg.sdsmu.edu.cn}{智慧学工系统}
    \item 点击菜单栏“新生报到”并登录系统(注:账号为身份证号,初始密码为Sddeykdx+身份证号后六位)
    \item →根据其中的相关指引完成预报到流程
    \item →查看宿舍号、学号、班级、院系等相关信息,并根据个人需求选择是否订购军训套装、被褥套装\footnotemark 等
          \footnotetext{如自带则必须按学校有关规定购买尺寸、颜色一致的样式;详请参见\uref{bedding_set}{此处}与录取通知相关材料等说明。}
    \item 智慧学工系统的操作办法同上
\end{enumerate}

\subsection[校园网]{校园网}
\subsubsection[无线连接]{无线连接}
\label{wifi_register}
\begin{enumerate}
    \item 新生账号激活:
          \begin{enumerate}
              \item 到校报到完成后,连接名称为“sdsmu-net”的网络
              \item 在浏览器中打开\uhref{https://slzfw.sdsmu.edu.cn:8800}{https://slzfw.sdsmu.edu.cn:8800$^㊕$}
              \item 点击“用户激活”,账号为身份证号
              \item 按照流程完成设置密码\footnotemark 等一系列步骤即可激活账号
                    \footnotetext{密码需10位以上17位以下,同时含有含数字、大小写字母和特殊字符。}
          \end{enumerate}
    \item 登陆:在浏览器中打开\uhref{https://slrz.sdsmu.edu.cn}{https://slrz.sdsmu.edu.cn$^㊕$}或\uhref{http://210.44.80.65}{http://210.44.80.65$^㊕$}并登陆即可
    \item 忘记密码:在浏览器中打开\uhref{https://slzfw.sdsmu.edu.cn:8800}{https://slzfw.sdsmu.edu.cn:8800$^㊕$},点击右下角“忘记密码”并按提示进行重置,如果因更换手机号或手机号停机等原因导致无法重置,请联系班长或本班网络信息员。
    \item \textbf{充值\footnotemark}:点击\uhref{https://slzfw.sdsmu.edu.cn:8800}{https://slzfw.sdsmu.edu.cn:8800$^㊕$},并登录;或在校园网的登陆页面点击“自助服务”按钮,登陆后即可使用支付宝充值(微信支付暂时无法使用)
          \footnotetext{若点击“自助服务”按钮后网页显示“403 Error”,请将网址前缀“http”改为“https”,按回车即可。如发现无法正确弹出支付宝支付二维码(通常因校园网额度耗尽导致),请换用热点,并刷新该付款二维码页面。}
    \item 当前政策为每月免费60G,超出部分按0.5¥/G收费,最多额外购买60G流量
    \item \textbf{警告:}一个账号最多允许3台设备同时在线
\end{enumerate}

\subsubsection[有线连接]{有线连接}
\begin{enumerate}
    \item 按照\uref{wifi_register}{此处}的教程激活校园网
    \item 将网线插入宿舍内的“H3C”盒子的底部右侧的接口内并正确连接到电脑\footnotemark
          \footnotetext{若始终无法连接,应检查网线的内部线排列顺序,从左到右应为“白橙橙,白绿蓝,白蓝绿,白棕棕”。}
          \begin{enumerate}
              \item 首次连接:
                    \begin{enumerate}
                        \item 打开“设置”→“网络和Internet”→“拨号”→“设置新连接”
                        \item →“连接到Internet”→点击“否,创建新连接(C)”→选择“宽带(PPPoE)(R)”
                        \item →输入自己的校园网帐号以及密码→勾选“记住密码”
                    \end{enumerate}
              \item 再次连接:
                    \begin{enumerate}
                        \item 打开“设置”→“网络和Internet”→“拨号”
                        \item →选中之前设置的网络→连接即可
                    \end{enumerate}
          \end{enumerate}
    \item  注:上述均为Win10/Win11教程,Mac教程暂无。
\end{enumerate}

\subsection[校园手机卡(校园卡)]{校园手机卡(校园卡)}
\begin{enumerate}
    \item \textbf{开通与否全凭自愿,是否开通校园卡不影响校园网的使用。}随录取通知书一并寄出。如遇强制开通可告知带班学长或自行反馈
    \item 套餐\footnotemark 通常内含至少100分钟全国通话、80G校园流量(仅山东省内所有高校可用)
          \footnotetext{详情优惠政策可咨询营业厅,如追求更多流量建议学校各账号不绑定校园手机卡,每年根据新优惠政策调整手机卡(需及时注意注销旧手机卡以防欠费导致的信用记录问题)。}
    \item 开学报到当天可前往大服,免费领取礼品
    \item 欠费销户提醒:大部分均需要主动销户,公告明确支持欠费销户的仅有移动的部分套餐。不需要继续使用手机卡请及时销户,如长期欠费可能影响以后办理手机卡号;欠费极其严重的甚至可能被列入征信黑名单中!
    \item \textbf{\uuline{出租手机卡可能造成违法犯罪!!!}}
\end{enumerate}

\subsection[高校edu教育邮箱]{高校edu教育邮箱}
\label{email}
\begin{enumerate}
    \item 学校可免费开通以“@sdsmu.edu.cn”结尾的教育邮箱
    \item 具体开通方式与使用说明详见学校官网公告栏
    \item 附2023学年下学期通知以供参考:\uhref{https://www.sdsmu.edu.cn/2024/0308/c14a128552/page.htm}{《关于为在校生开通校内邮箱的通知》}
\end{enumerate}

\subsection[空调使用教程]{空调使用教程}
\label{air_control}
\begin{enumerate}
    \item 微信关注“海享租”公众号,点击公众号菜单“在线租赁”,并注册、登录
    \item 点击“扫一扫”→扫描空调右下角二维码进行租赁\footnotemark
          \footnotetext{若提示租赁失败,请按照软件提示联系同宿舍的学长/学姐退租,也可咨询学长学姐或向宿管反馈。}
    \item →租赁完成后,点击“设备”→“空调图标”→“时长”,进行充值
    \item →点击“设备”→“空调图标”→“成员管理”,在此页面下将宿舍全部成员权限设置为均可管理空调开关即可
    \item \textbf{注意:}空调使用时长收费(0.55元/小时),具体收费及租赁政策详见“海享租”公众号
    \item \textbf{警告:}请各位同学在搬离校区或毕业前\textbf{退租空调},否则将导致下一届学弟学妹无法租赁!
\end{enumerate}

\subsection[浴室预约与使用]{浴室预约与使用}
\subsubsection[浮烟山校区]{浮烟山校区}
\label{shower_software_f}
\begin{enumerate}
    \item 软件基础设置:
          \begin{enumerate}
              \item 在手机应用市场下载“大白U帮”app
              \item 按照实际住宿情况注册
              \item 授予并开启“定位”与“蓝牙”权限
          \end{enumerate}
    \item 本楼层小浴室使用:
          \begin{enumerate}
              \item 带好洗浴物品前往公共厕所旁边的浴室排队
              \item 进入浴室,点击如右图所示的按钮(\noindent\mbox{\includegraphics[height=2.4ex]{resources/sundry/bath.pdf}})→选择“蓝牙设备”→“点击进行时”
              \item →“洗澡”→“搜索洗澡”\footnotemark
                    \footnotetext{搜索不到设备请务必开启“蓝牙”功能,学校的设备无法扫码连接。}
              \item →选择设备\footnotemark →“开始洗澡”
                    \footnotetext{距离厕所入口最近的是1号,远的是2号;不确定可以询问学长。}
              \item 结束后点击“结束洗澡”按钮,并结算
          \end{enumerate}
    \item 一层公共大浴室预约:
          \begin{enumerate}
              \item 在软件初始界面根据实际情况选择“X号楼1层”的浴室
              \item →点击一个浴位,并点击“预约”按钮(若已满请选择“排队”)
              \item →在8分钟内前往浴室,并点击“开始洗浴”
              \item →结束后点击“结束洗澡”按钮,并结算
          \end{enumerate}
    \item 费用:以程序显示为准,详情收费标准略
    \item 申诉:如果在洗澡时突然停电导致无法结束洗澡而被扣费,请按照软件打开时弹出的公告,联系相关工作人员处理
\end{enumerate}
\subsubsection[虞河校区]{虞河校区}
\label{shower_software_y}
\begin{enumerate}
    \item 支付宝搜索“住理生活”小程序,并按照提示开通账户(建议支付宝绑定银行卡)
    \item →搜索“潍坊医学院虞河校区”,并绑定账号到自己性别的浴室
    \item →扫描设备二维码即可使用淋浴
    \item 也可在小程序主页面点击“洗浴”→“切换设备”→手动选择设备进行洗浴
    \item \textbf{注意:浴室营业时间为10:00~21:45,22:00停水,浴室位置较少,请错峰洗澡!}
\end{enumerate}
\subsubsection*{注意事项}
\begin{enumerate}
    \item 如果未点击“结束洗澡”按钮便直接离开可能会被多扣费
    \item 严令禁止在浴室内大便!!!
\end{enumerate}

\subsection[洗衣机/洗鞋机使用教程]{洗衣机/洗鞋机使用教程}
\subsubsection[浮烟山校区]{浮烟山校区}
\label{washing_machine_f}
\begin{enumerate}
    \item 在微信小程序搜索“海乐生活”,并注册、开启相机与定位权限
    \item 预约方法(也可直接使用):
          \begin{enumerate}
              \item 在小程序内点击“附近营业点”→找到“潍坊医学院X号楼”
              \item →选择相应的楼层→选中洗衣机并下单→等待上次洗衣结束
              \item →前往洗衣机→输入验证码→放入衣物与洗衣粉/洗衣液并缴费
          \end{enumerate}
    \item 扫码直接使用(不可预约):
          \begin{enumerate}
              \item 前往洗衣机→在小程序内点击“扫码使用”按钮
              \item →扫描洗衣机上的二维码→选择并下单
              \item →放入衣物与洗衣粉/洗衣液,输入验证码后缴费即可
          \end{enumerate}
    \item 收费标准详见软件说明
    \item 洗衣机错误处理办法(若无相关经验切忌自行操作):
          \begin{enumerate}
              \item 拨打洗衣机旁边的报修电话;
              \item E1:洗衣机断电后开门,打开洗衣机右下角小门,旋开阀门,使用镊子等工具伸入并清除其中堵塞管道的杂物,恢复原样即可;
              \item E4:旋开洗衣机后方的水管阀门即可。
          \end{enumerate}
\end{enumerate}
\subsubsection[虞河校区]{虞河校区}
\label{washing_machine_y}
\begin{enumerate}
    \item 微信小程序搜索“智慧笑联”→按照提示注册并授予定位权限
    \item →绑定“潍坊医学院(虞河校区)”→绑定至自己所在的宿舍楼
    \item →再次扫描洗衣机上的二维码,按提示操作并付款即可
    \item 收费标准详见软件说明
\end{enumerate}

\subsubsection[禁止事项]{禁止事项}
\begin{enumerate}
    \item 禁止向第二格内倾倒洗衣粉、洗衣液,\textbf{第二格是放柔顺剂的}!
    \item \textbf{禁止使用洗衣机洗鞋},请用旁边的洗鞋机!
    \item 禁止在洗衣机上堆放杂物
    \item \textbf{\uuline{禁止将袜子、内衣内裤等贴身衣物机洗!}}
\end{enumerate}

\subsection[烘干机使用教程]{烘干机使用教程}
\label{dry_machine}
\begin{enumerate}
    \item 注册等步骤详见\uref{washing_machine_f}{此处}
    \item 按照预约的方法,选择宿舍楼后,选择“烘干机”即可,其他步骤与洗衣相似
    \item 推荐烘干配置
          \begin{enumerate}
              \item 高温60分钟:大部分轻薄的衣物(例如T恤、卫衣、浴巾等)
              \item 高温120分钟:薄被(如夏凉被)
          \end{enumerate}
    \item \textbf{注意:}\textbf{使用前后务必控干水箱并清理滤网。}棉被、羽绒服等禁止使用烘干机烘干以免损坏及不必要的危险情况发生。
    \item 收费标准详见软件提示
\end{enumerate}

\subsection[吹风机使用教程]{吹风机使用教程}
\label{hair_drier}
\begin{enumerate}
    \item 每层公共浴室旁边有两个公用吹风机,需扫码\footnotemark 租赁使用
          \footnotetext{部分吹风机屏幕二维码可能有缺损,不易扫描成功,多次尝试即可。(推荐使用浏览器扫码并复制到微信内收藏该网址,下次直接在微信内点击即可使用。)}
    \item →待手机发出“滴--滴”的声音后租赁成功
    \item →将手机扬声器对准吹风机租赁器方可正常使用
    \item 收费标准:详见软件提示,1分钱起步
\end{enumerate}

\subsection[空闲教室查询]{空闲教室查询}
\label{spare_classroom}
\begin{enumerate}
    \item 根据\uref{cas_system}{下文}的CAS认证系统教程,登录山二医app(下载链接见\uref{sdsmu_app}{此处})
    \item 点击下方菜单栏“应用”→空闲教室查询
    \item 点击右上角“\ 〉”按钮→根据自己的需求进行筛选
    \item \textbf{注意:}在临近期末考试时,因考试教室占用等原因查询结果可能不准确。
\end{enumerate}

\subsection[图书馆座位预约教程]{图书馆座位预约教程}
\label{library_book}
\begin{enumerate}
    \item 微信小程序搜索“青栀校园”→微信注册登录并绑定学号→允许小程序通知
    \item →在小程序内点击“座位预约”或扫描图书馆座位上的二维码即可
    \item \textbf{座位暂离的注意事项:}
          \begin{enumerate}
              \item 如因各种原因需要长时间离开的,请在小程序上选择“暂时离开”,否则按违规处理
              \item 离馆时也需要在小程序内确认
              \item 如发现已预约的座位被他人占据,请扫描桌面上的二维码并在小程序内举报,工作人员将尽快处理
          \end{enumerate}
    \item \textbf{违规说明:}
          \begin{enumerate}
              \item 已预约而未按时到位的记一次违规
              \item 未选择暂离而离开座位被举报的记一次违规
              \item 停止使用后未选择退馆的记一次违规
              \item \textbf{三次违规后将取消座位预约资格三天}
          \end{enumerate}
    \item 其他禁止事项
          \begin{enumerate}
              \item 禁止占用其他人已经预约的座位
              \item 禁止在图书馆内喧哗、吸烟
              \item 禁止在图书馆的非背诵区域内背诵、朗诵、频繁交流
              \item 禁止在图书馆内谈恋爱、亲嘴
          \end{enumerate}
\end{enumerate}

\subsection[设施报修方式枚举]{设施报修方式枚举}
\label{repair_report}
\begin{enumerate}
    \item 宿舍维修(浮烟山校区):
          \begin{enumerate}
              \item 加入各宿舍楼的QQ报修群,在群内反映具体故障
              \item 前往一层宿管处填表报修
              \item 在宿舍一层宿管旁边的公告栏处查看相关负责人的电话,直接拨打即可
              \item 拨打学生公寓管理中心电话反馈
              \item 询问带班学长、学姐
          \end{enumerate}
    \item 教室维修(浮烟山校区):
          \begin{enumerate}
              \item 拨打后勤管理处的电话报修
              \item 拨打教室管理中心的电话报修(仅限多媒体及饮水机)
              \item 拨打物业电话报修
              \item 在“诉求留言”微信小程序内反馈
          \end{enumerate}
    \item 宿舍维修(虞河校区):在宿管处填写报修单据并送往物业维修中心(位置见\uref{common_locations_yuhe}{此处})
    \item 教室维修(虞河校区):拨打教室内张贴的报修电话
\end{enumerate}

\subsection[缓考申请教程]{缓考申请教程\footnotemark}
\footnotetext{各学院要求不一:临床医学院大部分课程可以只填写钉钉、教务系统,仅少数要求同时提交纸质表格;其他学院以教师要求为准。}
\begin{enumerate}
    \item 填写钉钉\footnotemark(详细步骤如下)
          \footnotetext{需在学校统一将大家拉入钉钉的“山东第二医科大学”企业后方可使用。}
          \begin{enumerate}
              \item 打开钉钉→点击左上角选择主企业为“山东第二医科大学”
              \item →点击页面最下方菜单栏“工作台”→点击“OA审批”
              \item →在“学风建设”类选择“学生缓考审批表”(或直接搜索“缓考”)
          \end{enumerate}
    \item 在教务处下载《\uhref{https://jwch.sdsmu.edu.cn/_upload/article/files/f7/d0/c172c4f74eecba307f700cde1a21/99599310-0254-48be-adc5-fcafa99e7341.doc}{山东第二医科大学学生缓考审批表}》,打印3份并按照要求填写完毕
    \item 填写教务系统(详细步骤如下)
          \begin{enumerate}
              \item 进入教务系统(仅校园网,详情参见\uref{academic_affairs_system}{此处})
              \item 点击左侧菜单“考试报名”→“我的申请”→“缓考申请”
              \item →选择“学年学期”和“活动名称”后,直接点击“搜索”(不要填写科目名称)
              \item →在弹出的菜单中选择缓考科目并填写申请\footnotemark
                    \footnotetext{若因病缓考体测,需要上传病历本等相关材料,并前往校医院开具证明,再前往学工办向教师当面说明情况。}
          \end{enumerate}
    \item 前往学工办交表并等待审批
\end{enumerate}

\subsection[教学楼多媒体教室/乐道济世书院第二课堂申请流程]{教学楼多媒体教室/乐道济世书院第二课堂申请流程}
\begin{enumerate}
    \item 提前做好《活动策划案》、《活动安全应急预案》
    \item 请指导老师(通常为班主任或学工办老师)审核上述文件并确认钉钉具体审批负责人
    \item 打开钉钉→点击左上角选择主企业为“山东第二医科大学”
    \item →点击页面最下方菜单栏“工作台”→点击“OA审批”
    \item →在“校园文化活动与社会实践”类选择对应申请表并填写(或直接搜索“教室”/“书院”)
    \item 前往教室E区2层(见\uref{map_fuyanshan_teach_building}{此处})靠近A区处的“教室管理中心”签字盖章
    \item 前往预约的教室,拨打讲台上教室管理员的电话提前沟通说明
\end{enumerate}

\subsection[学生会校级格式]{学生会校级格式}
\begin{enumerate}
    \item 用途:校级格式是用于校内各类\textbf{正式文稿}(如活动通知、综测条例、学生会文件等)的标准
    \item 具体要求:
          \begin{enumerate}
              \item 纸张大小:A4纸
              \item 装订:页面左侧上下各 $\frac{1}{4}$ 处,距左边界0.3~0.5㎝处,钉与纸张左边界平行
              \item 页边距:上下左右均为2.5㎝
              \item 页码:两页及以上的材料,在页面底端居中插入
              \item 行间距:固定值,26磅
              \item 标点:均为中文标点,除特殊情况\footnotemark 外不应使用英文标点
                    \footnotetext{特殊情况举例:时间(22:05),英文活动(So, let us practice English!)等。}
              \item 文章标题格式:方正小标宋简体,小二号,居中
              \item 标题与正文之间空一行
              \item 正文格式:仿宋-GB2312,三号,两端对齐,首行缩进2字符
              \item 正文一级标题格式:“\textbf{一、}”;黑体,三号,居左,首行缩进2字符
              \item 正文二级标题格式:“\textbf{(一)、}”;楷体-GB2312,三号,居左,首行缩进2字符
              \item 正文三级标题格式:“\textbf{1、}”;其他要求与正文相同
              \item 正文四级标题格式:“\textbf{(1)}”;其他要求与正文相同
              \item 正文与落款之间空2行
              \item 落款机构(个人姓名)格式:仿宋-GB2312,三号,右对齐
              \item 落款时间格式:仿宋-GB2312,三号,右对齐,时间格式范例:2024年02月10日
          \end{enumerate}
\end{enumerate}

\subsection[小组汇报PPT制作指南(初级)]{小组汇报PPT制作指南(初级)}
\begin{enumerate}
    \item 整体要求
          \begin{enumerate}
              \item 选择软件:请使用大多数人使用的微软\ Office办公软件,或者金山WPS最新版;并在电脑上进行编辑,手机当且仅当用于ppt的简单查看。如无特殊情况不要使用LibreOffice、OpenOffice、腾讯文档等办公套件以免不兼容。
              \item \textbf{明确比例}:PPT(也称Slide)有4:3与16:9两种主流比例\footnotemark
                    \footnotetext{注:4:3比例像正方形,16:9是明显的长方形;下述的各类参数(如:24/32号字体)均以4:3、16:9的顺序进行。}
              \item 首页规范:首页应含有所有的必要信息(详情见下文)
              \item 目录规范:PPT应当在第二页含有一个简洁明了的目录
              \item 内容规范:\textbf{PPT内仅应含有所讲内容的关键部分}而非一昧照抄原文
              \item 字体与段落规范:\textbf{PPT字体不应过小,间距不应过密}(详情建议见下文);禁止使用文字阴影;\textbf{特殊字体必须内嵌于PPT}(例如艺术字、书法体等),禁止现场在演示的电脑上下载补全需要的字体(操作步骤可搜索“在PPT内嵌入字体”)
              \item 图片规范:每张图片不应大于10M\footnotemark,如发现图片模糊,需手动检查PPT设置,将PPT更改为“禁止自动压缩图片”,操作步骤请自行搜索
                    \footnotetext{如果图片过大,可以使用\uhref{https://gitee.com/LinkChou/rimage_gui/releases/latest}{Rimage\_GUI}适当缩小图片体积(软件开源,如被报毒请自行分辨)。}
              \item 动画规范:\textbf{PPT不应有过多动画以及元素堆叠}(例如,绝对禁止PPT中的一页内含15张大图片,依靠动画一张张切换,以免软件突然崩溃)
              \item 配色规范:请\textbf{使用经典的“文字—背景”配色}(\sout{虽然确实难看}),例如“黑—白”,“红—白”,\linebreak[3]“白—黑”,“蓝—白”等,切勿使用“橙—白”、“红—蓝”等投影后效果一塌糊涂的配色\footnotemark
                    \footnotetext{请注意ppt配色以使色觉感知局限的同学能顺畅接受信息。}
              \item 文件命名规范:要求PPT文件名\textbf{简单易懂,包含所有必须信息}
              \item 文件保存规范\footnotemark:\textbf{必须同时以“.pptx”后缀与“.ppt”后缀各保存一份}以免部分电脑无法正常打开,\textbf{禁止保存为“.dps”、“.odp”等特殊格式},详情见下
                    \footnotetext{\textbf{若无法看到文件名称后缀,请搜索“电脑设置显示文件扩展名”。}}
              \item 学校校徽及图标等标识使用规范:详情见\uhref{https://www.sdsmu.edu.cn/4229/list.htm}{《山东第二医科大学VIS视觉识别系统手册》}(由校宣传部印发)
          \end{enumerate}
    \item 详细要求:
          \begin{enumerate}
              \item 首页:需注明小组成员姓名及学号,日期,课程名称,授课教师等必要信息并\textbf{仔细检查是否有误},标题的字体不应小于80/75号,小组成员及其他信息字体不应小于25/25号
              \item 目录页:简短凝练,总字数不应超过35个字;字体不应小于80/55号
              \item 内容页:字体\footnotemark 不应小于55/40号,上下左右应各留出 $\frac{1}{10}$ 左右的间距;如内容过多应自行分页,\textbf{严禁为节省页数而缩小字号};PPT显示的内容与口述补充的内容在6:4或7:3左右最佳,各类关键数据的引用(例如学术数据、课标外的公式定义等)应当按照\linebreak[3]《\textbf{GB/T}\ 7714—2015 信息与文献 参考文献著录规则》的相关标准\textbf{标明出处}
                    \footnotetext{下面的字体皆以微软雅黑为标准,楷体、仿宋等纤细字体请自行增加字号。}
              \item 文件命名\footnotemark:推荐使用“\textbf{20XX级临床X班X组关于XXX的汇报(终稿).pptx}”此类命名,\textbf{严禁使用默认名称以防混淆}(例如“新建 Microsoft PowerPoint 演示文稿.pptx”)。此外,在ppt未定稿时,推荐使用一些默认的规范进行命名(例如“\textbf{关于XXX的汇报草稿-4.pptx}”),以方便小组成员确定PPT版本,而非使用默认的“新建 Microsoft PowerPoint 演示文稿(1)(2)(5).pptx”这种高血压命名。
                    \footnotetext{文件名称后缀的“.pptx”、“.ppt”等各种类型并非手动添加,而是文件自带,\textbf{严禁手动修改后缀名!}}
              \item 文件保存\footnotemark:如需在“.ppt”“.pptx”两种格式间相互转换请使用PowerPoint或WPS等办公软件,“.dps”格式必须使用WPS才能转换为“.ppt”或“.pptx”格式。
                    \footnotetext{\textbf{转换方法}:在左上角的“文件”菜单选择“另存为”,在下拉框中选择“.pptx”并保存。}
          \end{enumerate}

\end{enumerate}

\subsection[钉钉请假流程]{钉钉请假\footnotemark 流程}
\label{leave_dingtalk}
\footnotetext{需在学校统一将大家拉入钉钉的“山东第二医科大学”企业后方可使用。\textbf{各学院要求不一,仅以临床医学院为例。}}
\begin{enumerate}
    \item 线下请假步骤(正常情况):
          \begin{enumerate}
              \item 前往学工办或班主任办公室,当面请假并获得假条
              \item →根据老师要求扫描相关二维码→钉钉填表
              \item →刷脸进出校门,并将请假条之一交给保卫处
              \item →返校后,在钉钉的电子假条处,以评论的方式销假
          \end{enumerate}
    \item 线上请假步骤:
          \begin{enumerate}
              \item 打开钉钉→点击左上角选择主企业为“山东第二医科大学”
              \item →点击页面最下方菜单栏“工作台”→点击“OA审批”
              \item →在“学生日常事务管理”类选择“浮烟山校区本科学生请假单”→填表
              \item →电话联系班主任老师或学工办老师,说明请假事由并等待审批
              \item →审批通过后刷脸进出校门
              \item →返校后,在钉钉的电子假条处,自觉以评论的方式销假
          \end{enumerate}
          \textbf{注意:}请大家自觉销假,切忌一再拖延。
\end{enumerate}

\subsection[家长、校友进校参观指南]{家长、校友进校参观指南\footnotemark}
\footnotetext{注:本节仅适用浮烟山校区;虞河校区为开放校区,除宿舍、教学楼外,均可参观。}
\begin{enumerate}
    \item 学校在每年8月30日前后允许家长参观校园\footnotemark,具体政策以学校官方说明为准
          \footnotetext{允许社会车辆在参观时段在校园内划定的区域内停放,(通常)允许家长在规定时间内参观宿舍环境,具体政策每年不同,具体情况以学校官方说明为准。}
    \item 其他校友若有返校需求可凭本人毕业证、学生证等有效证件(电子版也可),或通过当年所在院系的老师联系保卫处,在校门口登记后即可入校参观
\end{enumerate}

\subsection[学费缴费教程]{学费缴费教程}
\label{fee_pay}
\begin{enumerate}
    \item 官方:微信公众号“山东第二医科大学财务处”或“\uhref{https://tyzfpt.sdsmu.edu.cn/xysf/login.aspx}{山东第二医科大学校园统一缴费平台}”
    \item 用途:学费缴纳、卡号绑定等
    \item 注意:缴费系统仅在部分时间段开放,请按学校通知按时缴费
    \item 学费缴纳教程:
          \begin{enumerate}
              \item 前往公众号菜单,点击右下角“缴费管理”→“支付平台”
              \item →登录系统(新生的账号为高考考生号,密码格式同老生;老生的帐号为学号,初始密码为姓首字母大写加身份证后六位)
              \item →按照提示修改初始密码(请务必牢记)→进行缴费
          \end{enumerate}
    \item 银行卡绑定教程:
          \begin{enumerate}
              \item 目的:学校仅在初次使用时收集一次卡号并存储数据,以便下次直接使用\footnotemark
                    \footnotetext{详情见学校官方说明。}
              \item 打开“山东第二医科大学财务处”公众号,点击“财务中心”
              \item →使用帐号密码登录并绑定微信号(帐号为学号,初始密码为000000)
              \item →点击“卡号维护”→“管理”
              \item →按照提示填写相关信息,确认信息无误后提交即可
          \end{enumerate}
\end{enumerate}

\subsection[学工系统(微信小程序)]{学工系统(微信小程序)}
\begin{enumerate}
    \item 用途:学工系统主要用于晚点名、返校信息填报等日常工作\footnotemark
          \footnotetext{原“请假审批”、“外出审批”工作已基本转移至“钉钉”,教程参见\uref{leave_dingtalk}{此处}。}
    \item 使用方式:
          \begin{enumerate}
              \item 打开“定位权限”并允许微信使用→在微信搜索“智慧学工”小程序
              \item 根据学校下发的账号密码进行登录(推荐立即与微信绑定以免忘记密码)
          \end{enumerate}
\end{enumerate}

\subsection[教务系统]{\textbf{\uuline{教务系统}}}
\label{academic_affairs_system}
\begin{enumerate}
    \item 官网:\uhref{https://jwgl.sdsmu.edu.cn}{https://jwgl.sdsmu.edu.cn$^㊕$}
    \item 用途:\textbf{选课,缓考申请,成绩查询},查看(导出)课程表,空闲教室查询
    \item \textbf{注意:}仅限校内访问,如需在外使用教务系统,参见\uref{cas_system}{此处}条目
\end{enumerate}

\subsection[CAS资源访问控制系统(校内VPN)]{\textbf{\uuline{CAS资源访问控制系统(校内VPN)}}\footnotemark}
\footnotetext{在校内时可通过校园网直接访问相应系统,无需使用本系统中转。}
\label{cas_system}
\begin{enumerate}
    \item 官网:\uhref{https://webvpn.sdsmu.edu.cn}{https://webvpn.sdsmu.edu.cn}
    \item 说明:(也称CAS认证系统,智慧校园系统)本系统主要用于\textbf{在校外访问校内网络信息资源},如:教务系统(查成绩、课表)、知网、临床医学虚拟仿真实验中心\footnotemark 等
          \footnotetext{查阅文献推荐使用CARSI系统,速度快效果好,教程详见\uref{carsi_system}{此处}。}
    \item 异地登录教务系统教程(其它系统同理):
          \begin{enumerate}
              \item 打开网站,点击“统一身份认证登录”,使用学号+密码登录或手机号验证登陆、扫描登陆等方式均可(推荐绑定微信,初始密码为sdsmu@身份证后六位)
              \item →找到应用中心→“教务系统—非单点登录”\footnotemark →使用教务系统账号密码登录即可
                    \footnotetext{请注意,在校外时点击“教务系统”无法登录,只有“非单点”能校外登录!}
          \end{enumerate}
\end{enumerate}

\subsection[CARSI系统]{\textbf{\uuline{CARSI系统}}}
\label{carsi_system}
\begin{enumerate}
    \item 官网:\uhref{https://ds.carsi.edu.cn}{https://ds.carsi.edu.cn}
    \item 说明:(与CAS认证系统作用不同)用于快速访问学校订阅的各类数据库,如百度文库、知网、万方、维普等
    \item 使用教程:
          \begin{enumerate}
              \item 进入官网→搜索“山东第二医科大学”并勾选“记住我的选择”→点击后进入登陆界面\footnotemark
                    \footnotetext{请注意,不要收藏登录界面的网址!每次登录网址都不一样,只能从官网重新进入!}
              \item →使用\textbf{CAS认证系统}的账号密码登录系统,出现各类弹窗一律选择“Accept”即可\footnotemark
                    \footnotetext{仅推荐在自己的电脑上如此设置,如必须在网吧等公共场所的电脑上使用,请审慎阅读相关提示,并谨慎进行登录,如因账号泄露造成损失,一切责任自负。}
              \item →登录完成后,点击任意资源链接即可进入相应网站并获取论文
          \end{enumerate}
\end{enumerate}

\subsection[校园一卡通系统]{校园一卡通系统}
\label{union_card}
\begin{enumerate}
    \item 用途:校园支付(如:浮烟山校区杏林餐厅、虞河校区乐道餐厅\footnotemark)、图书馆(借阅)
          \footnotetext{餐厅完全支持支付宝支付与微信支付,普通学生使用此付款码频率较低。}
    \item 另:付款码负责餐厅消费,身份码负责图书馆借阅
    \item 注:初始密码为身份证后六位,X替换为0
\end{enumerate}

\subsection[校务行(微信小程序)]{校务行(微信小程序)}
\label{cert_prover}
\begin{enumerate}
    \item 官网:微信小程序
    \item 用途:查成绩,下载学籍证明、成绩证明的pdf版本
    \item 费用:以程序显示为准
    \item 教程:
          \begin{enumerate}
              \item 搜索小程序“校务行”
              \item →点击右上角“点击登录”(帐号为学号,密码为身份证后六位)\footnotemark
                    \footnotetext{新生可能无法在开学后即刻使用本程序,需等待学校将信息录入完毕。}
              \item →按需选择“电子成绩单”或“电子证明”→按照小程序提示进行即可
          \end{enumerate}
\end{enumerate}

\subsection[档案查询]{档案查询}
\subsubsection[档案远程服务利用系统(学校)]{档案远程服务利用系统(学校)}
\begin{enumerate}
    \item 官网:\uhref{https://dangan.sdsmu.edu.cn/service-utilization/web/management/index}{https://dangan.sdsmu.edu.cn/service-utilization/web/management/index}
    \item 微信公众号入口(与官网作用相同):“山东第二医科大学”公众号→微服务→档案查询
    \item 业务范围:
          \begin{enumerate}
              \item 招生录取名册(学生登记表)
              \item 学生成绩
              \item 其他学籍档案
              \item 查档预约、查档咨询
          \end{enumerate}
    \item 其他注意事项详见官网
\end{enumerate}
\subsubsection[高校档案查询利用平台]{高校档案查询利用平台}
\begin{enumerate}
    \item 官网:\uhref{http://gxda.dag.shandong.gov.cn:81/index}{http://gxda.dag.shandong.gov.cn:81/index}
    \item 业务范围:同“档案远程服务利用系统”
\end{enumerate}

\subsection[心理健康教育中心预约教程]{心理健康教育中心预约教程}
\begin{enumerate}
    \item 作用:纾解心理压力、提供心理咨询等
    \item 工作时间(周一至周五):08:00~11:30、14:00~17:30
    \item 提示:预约后1个工作日左右,咨询中心将回电联系并协商心理咨询时间,请注意接听电话
    \item 线下预约:至浮烟山校区E206填表后等待咨询中心回电
    \item 电话预约:拨打0536-8462130
    \item 线上预约:填写\uhref{https://www.wjx.cn/vm/YOHd59S.aspx}{心理咨询信息登记表}后等待回电

\end{enumerate}

\subsection[常见证明申请]{常见证明申请}
\begin{enumerate}
    \item 说明:适用于学籍证明\footnotemark、成绩证明、个人档案
          \footnotetext{在考试时,学生证与学籍证明拥有相同效力,丢失学生证可使用学籍证明代替。}
    \item 线下打印
          \begin{enumerate}
              \item 携带身份证、手机前往D区的打印机,详情位置\uref{common_locations_fuyanshan}{见此}
              \item 按照机器的说明填写相关信息并打印(无纸时请联系工作人员)
          \end{enumerate}
    \item 线上下载
          \begin{enumerate}
              \item 学籍证明:请按\uref{cert_prover}{校务行教程}申请下载至邮箱中
              \item 个人档案:按照\uref{cert_prover}{此处}的教程申请并下载,自行前往打印店打印即可(请注意及时删除打印完毕的文件以免泄漏隐私)
          \end{enumerate}
\end{enumerate}

\subsection[监控录像调取等申请]{监控录像调取等申请}
\label{sdsmu_app}
\begin{enumerate}
    \item 用途:调取监控录像、申请课程录制等
    \item 按照 \uref{cas_system}{此处}的说明登录“山东第二医科大学App”
    \item 点击下方菜单栏“应用”→“业务申请”
    \item 按需选择“保卫处调阅监控录像审批流程”等申请
\end{enumerate}

%\newpage % 必须使用newpage而非pagebreak,因为pagebreakv会使下方的表格尝试排版到上一页导致underfull vbox 警告
%尽量手动控制此节的排版……
\subsection[公交信息与免费乘车指南]{公交信息与免费乘车指南}
\label{free_bus}
运营时间通常为6:30~19:00

免费线路\footnotemark:9、13、19、29、63、69、71、101、109、166、167
\footnotetext{据《滨海及浮烟山两地大学城在校学生免费乘坐校区至中心城区公交车实施方案》(潍交城〔2024〕5号),免费公交政策暂行时间为:2024.04.29~2025.04.28。在学校提交相关信息后,可通过\textbf{与提交信息相符}的手机号注册认证“潍坊公交大学生乘车码”小程序免费乘车,新生可持学生证或录取通知书临时代替。}

\begin{tblr}[
        long,
        caption = {常用站点名称对应关系一览表},
    ]{
        cells = {c,m},
        rowhead = 1,
        row{1} = {cmd=\bfseries},
        vlines,
        hlines,
        cell{2,8,30,32}{1} = {r=2}{},
        cell{10,20,23}{1} = {r=3}{},
        cell{4,13}{1} = {r=4}{},
        cell{28,31}{3} = {r=2}{},
        hline{1-2,8,13,17,Z} = {-}{1pt},
    }
    %尽量手动控制,尽量避免因自动断页导致的underfull vbox警告
    常见目的地       & 具体方位     & 站点名                   \\
    浮烟山校区       & 南门         & 山东第二医科大学南门     \\*
                     & 北门         & 山东第二医科大学北门     \\\nopagebreak[2]
    虞河校区         & 北门         & 胜利街虞河路路口西       \\*
                     & 东北侧       & 虞河路胜利街路口北       \\*
                     & 东门         & 虞河路胜利街路口南       \\*
                     & 西侧         & 鸢飞路胜利街路口北       \\
    附属医院         & 西门         & 附属医院                 \\*
                     & 南侧         & 福寿街虞河路路口东       \\
    人民医院         & 本部东门     & 人民医院东门             \\*
                     & 本部西侧     & 人民医院                 \\*
                     & 北辰院区南门 & 北辰院区                 \\
    火车站           & 北广场       & 火车站                   \\*
                     & 东侧         & 青年路铁路桥北           \\*
                     & 东北侧       & 火车站                   \\*
                     & 潍坊北站     & 高铁北站                 \\
    人民公园         & 西门         & 人民公园西门             \\
    市中医院         & 东门         & 市中医院                 \\
    妇幼保健院       & 东门         & 潍坊市妇幼保健院潍城院区 \\
    泰华城           & 北侧         & 泰华城                   \\*
                     & 西南侧       & 青年路胜利街路口南       \\*
                     & 南侧         & 风筝广场                 \\
    万达广场         & 西侧         & 鸢飞路福寿街路口南       \\*
                     & 南侧         & 潍坊中学                 \\*
                     & 北侧         & 福寿街鸢飞路路口东       \\
    谷德广场         & 南侧         & 谷德广场                 \\
    奎文防疫站       & 无           & 奎文防疫站               \\
    鲁台会展中心     & 北门         & 鲁台会展中心             \\*
    蓝海大饭店       & 南门         &                          \\*
    谷德锦           & 南侧         & 北宫街清平路路口西       \\*
                     & 东侧         & 潍坊交通职业中等专业学校 \\*
    奥林匹克体育公园 & 东侧         &                          \\*
                     & 南侧         & 奥体中心                 \\
    十笏园           & 西侧         & 十笏园
\end{tblr}

\begin{tblr}[
        long,
        caption = {常用路线汇总表},
        note{1} = {加粗线路为上述免费乘车线路。},
    ]{
        rowhead = 1,
        cells = {c,m},
        row{1} = {cmd=\bfseries},
        cell{1}{2} = {c=3}{},
        cell{2,7,9,11,15,20,26}{1} = {r=2}{},
        cell{4}{1} = {r=3}{},
        cell{4,18,19,22,24,28}{1} = {}{cmd=\bfseries},
        vlines,
        hlines,
        hline{1-2,Z} = {-}{1pt},
    }
    路线\TblrNote{1} & 常见目的地         &                          &                  \\*
    5                & 火车站(北广场)   & 市中医院(东门)         & 风筝广场         \\*
                     & 万达广场(西侧)   &                          &                  \\
    13               & 浮烟山校区(南门) & 虞河校区(北门/东北侧)  & 火车站(东侧)   \\*
                     & 附院(西门)       & 泰华城(南侧)           & 妇幼保健院       \\*
                     & 奎文防疫站         & 人民公园                 & 风筝广场         \\
    22               & 泰华城(北侧)     & 十笏园                   & 谷德锦           \\*
                     & 奥林匹克公园       &                          &                  \\
    23               & 火车站(北广场)   & 附院(南侧)             & 十笏园           \\*
                     & 万达广场(北侧)   &                          &                  \\
    25               & 虞河校区(东北侧) & 附院(西门)             & 火车站(东北侧) \\*
                     & 火车站(潍坊北站) & 奎文防疫站               & 妇幼保健院       \\
    D28              & 人民医院(东门)   & 人民医院(北辰院区南门) &                  \\
    56               & 火车站(北广场)   & 泰华城(北侧)           & 谷德广场         \\
    59               & 虞河校区(北门)   & 虞河校区(东门)         & 人民医院(东门) \\*
                     & 谷德锦(南侧)     & 奥林匹克公园(东南侧)   &                  \\
    66               & 虞河校区(北门)   & 市中医院                 & 万达广场(南侧) \\
    69               & 浮烟山校区(北门) & 火车站(东侧)           & 火车站(北广场) \\
    71               & 浮烟山校区(北门) & 火车站(东侧)           & 火车站(北广场) \\
    75               & 虞河校区(西侧)   & 万达广场(西侧)         & 风筝广场         \\*
                     & 火车站(北广场)   &                          &                  \\
    101              & 浮烟山校区(北门) & 火车站(北广场)         &                  \\
    106              & 火车站(东北侧)   & 火车站(潍坊北站)       &                  \\
    109              & 浮烟山校区(南门) & 泰华城(东北侧)         &                  \\
    132              & 人民医院(西侧)   & 万达广场(西侧)         &                  \\
    162              & 虞河校区(东门)   & 人民医院(东门)         & 附院(西门)     \\*
                     & 奎文防疫站         &                          &                  \\
    167              & 浮烟山校区(南门) & 鲁台会展中心(北门)     & 蓝海大饭店
\end{tblr}

%\newpage %手动控制断页,原因不再赘述
\subsection[高铁学生票购买流程]{高铁学生票\footnotemark 购买流程}
\footnotetext{仍保留线下优惠资质核验、学生票购买渠道,若操作遇到问题可直接前往线下售票处,通过工作人员核验与购买;详细的注意事项等其他相关内容参见12306官网或其官方app。}
\begin{enumerate}
    \item 关注班级群内信息,及时填写“火车学生优惠申请”类的表格\footnotemark
          \footnotetext{仅入学时填写一次,如因搬迁等原因变更优惠区间请咨询班长。}
    \item 待学校完成学籍注册并下发学生证后,确认学生证背面贴有“火车票学生优惠卡”\footnotemark
          \footnotetext{若学生证丢失请以学籍证明替代,并前往人工窗口办理相关业务;学生证每学期以班级为单位统计一次并统一补办。}
    \item 首次使用请先在\uhref{https://www.chsi.com.cn}{学信网}确认学籍无误且已更新
    \item 打开12306 app→进入“我的”页面→点击“学生优惠资质核验”旁的“点击查看”按钮
    \item →填写学生资质信息→等待审核结果(3个工作日内反馈)
    \item 价格:高铁75折,普通火车5折(仅二等座可使用优惠,详见相关规定)
    \item \textbf{注意:}
          \begin{enumerate}
              \item 每学年\footnotemark 有4次优惠机会
                    \footnotetext{每年10月1日至下一年的9月30日为一个学年。}
              \item 每学年需重新核验\footnotemark 一次优惠资质
                    \footnotetext{现均已升级为线上自动核验,通常情况下无需操心。}
          \end{enumerate}
\end{enumerate}

\subsection[文体中心预约教程]{文体中心预约教程}
\label{sports_center_book}
\begin{enumerate}
    \item 关注“山二医文体中心”公众号→点击“场地预约”→“预约入口”
    \item →点击“我的”→“校内登录”→使用CAS认证系统的账号密码登录
    \item 进入个人中心页面,点击“人脸录入”,录入信息→
    \item 按需预约游泳馆、羽毛球馆等,并付费→
    \item 在预定时间段\footnotemark 前往场馆,向工作人员出示二维码即可
          \footnotetext{场馆开放时间见\uref{sports_center_operating_hours}{此处}。}
    \item 费用(校内学生):
          \begin{enumerate}
              \item 健体中心:3元/2小时
              \item 羽毛球馆\footnotemark:6元/时/片 3元/人/2小时
                    \footnotetext{羽毛球等多人运动项目预约方式为“单人预约,多人共享”,详情可咨询负责相关场馆的教师。}
              \item 篮球馆:20元/1小时(半场)或40元/1小时(全场)
              \item 游泳馆\footnotemark:8元/场/2小时
                    \footnotetext{注意:游泳馆仅限本人购票,代购无效,每人每场限购一张。}
              \item 室内乒乓球场:2元/台/2小时
              \item 室内网球场:6元/片/1小时
          \end{enumerate}
    \item 特殊说明:因羽毛球馆和篮球馆共用同一场地,故此二者互斥;场馆具体开放时间以及价格变动以公众号通知为准
\end{enumerate}
%常用教程
% 校级社团
\newpage

\section[各级组织信息汇总]{各级组织信息汇总}
\subsection[校级社团]{校级社团}
\label{community_summary}
\begin{table}[H]
    \centering
    \vspace{2em}%手动粗略居中
    \noindent\begin{tblr}{
            cells = {c,m},
            row{1} = {font=\bfseries},
            cell{2}{4} = {r=3}{},
            cell{5}{4} = {r=6}{},
            cell{11}{4} = {r=5}{},
            cell{16}{4} = {r=2}{},
            cell{18}{4} = {r=8}{},
            column{1}={2em},
            column{2}={21em},
            column{3}={6em},
            column{4}={5em},
            vlines,
            hline{1-2,5,11,16,18,26} = {-}{1.2pt},
            hline{3-4,6-10,12-15,17,19-25} = {1-3}{},
        }
        序号 & 名称                                       & 成立时间   & 类型       \\
        1    & 习近平新时代中国特色社会主义思想学生研习社 & 2017年11月 & 思想政治类 \\
        2    & 大学生尽美诗社                             & 2021年06月 &            \\
        3    & 学生模拟政协协会                           & 2022年10月 &            \\
        4    & 大学生青年志愿者协会                       & 2002年05月 & 志愿公益类 \\
        5    & 大学生红十字协会                           & 2007年04月 &            \\
        6    & 大学生红丝带志愿者协会                     & 2008年11月 &            \\
        7    & “生命阳光”互助协会                         & 2010年11月 &            \\
        8    & 大学生信息安全志愿者协会                   & 2017年05月 &            \\
        9    & 大学生心跳行动急救协会                     & 2022年06月 &            \\
        10   & 大学生鸢飞阁文学社                         & 1992年10月 & 学术科技类 \\
        11   & 大学生英语爱好者协会                       & 2003年05月 &            \\
        12   & 大学生计算机协会                           & 2008年02月 &            \\
        13   & 大学生读者协会                             & 2010年11月 &            \\
        14   & 大学生心理学社                             & 2017年04月 &            \\
        15   & 大学生科技协会                             & 1994年06月 & 创新创业类 \\
        16   & 大学生职业发展协会                         & 2014年03月 &            \\
        17   & 大学生手语协会                             & 2006年02月 & 自律互助类 \\
        18   & 大学生针灸推拿协会                         & 2009年10月 &            \\
        19   & 大学生国旗护卫队                           & 2011年03月 &            \\
        20   & 大学生营养与健康协会                       & 2011年05月 &            \\
        21   & 大学生环境保护协会                         & 2013年09月 &            \\
        22   & 大学生健康养生协会                         & 2014年03月 &            \\
        23   & 大学生防痨协会                             & 2019年11月 &            \\
        24   & 大学生青春健康同伴社                       & 2021年10月 &
    \end{tblr}
\end{table}

\newpage
\begin{table}[H]
    \centering
    \noindent\begin{tblr}{
            cells = {c,m},
            row{1} = {font=\bfseries},
            cell{2}{4} = {r=30}{},
            column{1}={2em},
            column{2}={21em},
            column{3}={6em},
            column{4}={5em},
            vlines,
            hline{1-2,32} = {-}{1.2pt},
            hline{3-31} = {1-3}{},
        }
        序号 & 名称                   & 成立时间   & 类型       \\
        25   & 大学生书法美术协会     & 1986年09月 & 文化体育类 \\
        26   & 大学生棋类协会         & 1988年10月 &            \\
        27   & 大学生乒乓球协会       & 1995年09月 &            \\
        28   & 大学生手工制作协会     & 2001年07月 &            \\
        29   & 大学生次方动漫社       & 2003年10月 &            \\
        30   & 大学生礼仪队           & 2003年05月 &            \\
        31   & 大学生羽毛球协会       & 2004年03月 &            \\
        32   & 大学生足球协会         & 2004年08月 &            \\
        33   & 大学生吉他爱好者协会   & 2005年03月 &            \\
        34   & 大学生五月剧社         & 2005年05月 &            \\
        35   & 大学生武术协会         & 2005年06月 &            \\
        36   & 大学生篮球协会         & 2008年04月 &            \\
        37   & 大学生魔术协会         & 2010年10月 &            \\
        38   & 大学生英语俱乐部       & 2012年02月 &            \\
        39   & 大学生网球协会         & 2013年10月 &            \\
        40   & 大学生健美协会         & 2013年07月 &            \\
        41   & 大学生梅花桩协会       & 2013年09月 &            \\
        42   & 大学生台球交流协会     & 2013年09月 &            \\
        43   & 大学生形体芭蕾协会     & 2015年09月 &            \\
        44   & 大学生国风协会         & 2015年09月 &            \\
        45   & 大学生微电影协会       & 2015年09月 &            \\
        46   & 大学生茶韵社           & 2015年09月 &            \\
        47   & 大学生轮滑协会         & 2015年09月 &            \\
        48   & 大学生摄影影艺协会     & 2016年05月 &            \\
        49   & 大学生排球社           & 2017年11月 &            \\
        50   & 大学生户外运动协会     & 2017年06月 &            \\
        51   & 大学生社交演讲协会     & 2017年09月 &            \\
        52   & 大学生街舞协会         & 2018年11月 &            \\
        53   & 大学生流行爵士舞蹈协会 & 2018年11月 &            \\
        54   & 大学生乐道长跑协会     & 2023年09月 &            \\
    \end{tblr}
\end{table}

\newpage
\subsection[院级组织]{院级社团}
因名单时有变动故不在此一一列出,详询本学院团委。

典型的有本学院的学生会等。

\textbf{特别提醒:}如遇“勤工俭学”、“特殊兼职机会”请务必谨慎对待,如有必要应询问班长、老师以核验信息真伪。

\subsection[其他组织]{其他组织}
\begin{table}[H]
    \centering
    \begin{tblr}{
            cells={c,m},
            row{1} = {font=\bfseries},
            column{2-3} = {6em},
            cell{2}{1,4} = {r=2}{},
            cell{4}{1,4} = {r=6}{},
            cell{4}{2-3} = {}{font=\bfseries},
            vlines,
            hline{1-2,4,10-11} = {-}{1.2pt},
            hline{3,5-9} = {2-3}{},
        }
        组织名称     & 部门   & 部门(续)   & 注释          \\
        学生会       & 组织部 & 学习部       & {校级组织     \\(谨防以学生会为名的诈骗陷阱)} \\
                     & 卫生部 & 主席团(等) &               \\
        大学生艺术团 & 职能类 & 演艺类       & 校级组织      \\
                     & 剧务部 & 舞蹈队       &               \\
                     & 化妆部 & 合唱队       &               \\
                     & 服装部 & 主朗队       &               \\
                     & 宣传部 & 曲艺队       &               \\
                     & 事务部 & 器乐队       &               \\
        运动会       & 体操队 & 啦啦队       & {由学生会管理 \\(运动会后解散)}
    \end{tblr}
\end{table}
%社团
%老乡群
\newpage
\section[老乡群QQ号汇总]{老乡群QQ号汇总}
\vspace{8ex}
\begin{tblr}[
    long,
    theme = {no-caption},
    remark{敬告} = {请自行甄别群内消息的真伪,谨防电信诈骗!},
    ]{
    cells = {c,m},
    width = {.85\linewidth},
    colspec = {X[1]X[1.25]X[1]X[1.25]X[1]X[1.25]},
    cell{1}{1} = {c=2}{},
    cell{1}{3} = {c=2}{},
    cell{1}{5} = {c=2}{},
    cell{11}{3} = {c=2}{font=\bfseries},
    row{1} = {cmd=\bfseries},
    rows = {3.5ex},
    vlines,
    hlines,
    hline{1-2,Z} = {-}{1pt},
    hline{11-12} = {3-4}{1pt},
    vline{1,3,5,7} = {-}{1pt},
    }
    山东省老乡群号 &            & 潍坊市老乡群号   &           & 省外地区老乡群号(续) &            \\
    济南           & 904663629  & 潍坊总群         & 637883540 & 浙沪                   & 95657552   \\
    青岛1群        & 563428281  & 临朐             & 41400858  & 甘肃                   & 221050739  \\
    青岛2群        & 452136059  & 诸城             & 179287264 & 吉林                   & 383548342  \\
    临沂大群       & 343985896  & 青州             & 252017758 & 贵州                   & 224246288  \\
    临沂小群       & 892755471  & 高密             & 218591086 & 福建                   & 1040097548 \\
    威海           & 257361947  & 昌乐             & 348044656 & 福建                   & 826310289  \\
    济宁           & 608841050  & 安丘             & 612377914 & 河北                   & 89591427   \\
    菏泽           & 121772446  & 寿光             & 315710196 & 广东                   & 90804205   \\
    禹城           & 560497311  & 寿光迎新         & 687403533 & 海南                   & 117470688  \\
    日照           & 87132843   & 省外地区老乡群号 &           & 黑龙江                 & 276232923  \\
    枣庄           & 572915613  & 重庆             & 467613651 & 湖北新群               & 1104698842 \\
    淄博           & 452570842  & 江苏             & 304478885 & 湖北老群               & 175329442  \\
    聊城           & 262105269  & 广西             & 414785353 & 湖南                   & 66795629   \\
    东营           & 839450408  & 宁夏             & 150640532 & 辽宁                   & 252665043  \\
    德州           & 328364159  & 河南             & 119687693 & 青海                   & 87228512   \\
    沂水           & 108169413  & 安徽             & 592275507 & 陕西                   & 82826946   \\
    泰安           & 1149046796 & 四川             & 158962429 & 天津                   & 253848136  \\
    日照           & 87132843   & 江西             & 364678088 & 西藏                   & 644183908  \\
    滨州           & 673569972  & 山西             & 176374915 & 新疆                   & 516849351  \\
    烟台           & 775045068  & 内蒙古           & 480132318 & 云南                   & 789173514
\end{tblr}

\newpage%老乡群
%常用网站
\chapter[常用网站及应用汇总]{常用网站及应用汇总}

\section*{特别说明}
\begin{enumerate}
    \item 本节中所有上标“㊕”的网址仅在连接校园网时可成功访问相关服务。
    \item \textbf{严禁使用QQ、微信直接打开本文提及的一切网址,必须使用正常更新的主流浏览器打开!}
\end{enumerate}

\section[网站]{网站}
\subsection[主要站点]{主要站点}
\begin{enumerate}
    \item 山东第二医科大学:\uhref{https://www.sdsmu.edu.cn}{https://www.sdsmu.edu.cn}
    \item 马克思主义学院:\uhref{https://mksxy.sdsmu.edu.cn}{https://mksxy.sdsmu.edu.cn}
    \item 临床医学院:\uhref{https://lcyxy.sdsmu.edu.cn}{https://lcyxy.sdsmu.edu.cn}
    \item 第一临床医学院:\uhref{https://dylcyxy.sdsmu.edu.cn}{https://dylcyxy.sdsmu.edu.cn}
    \item 麻醉学院:\uhref{https://mzxxy.sdsmu.edu.cn}{https://mzxxy.sdsmu.edu.cn}
    \item 护理学院:\uhref{https://hlxy.sdsmu.edu.cn}{https://hlxy.sdsmu.edu.cn}
\end{enumerate}

\subsection[教辅与管理机构]{教辅与管理机构}
\begin{enumerate}
    \item 党委办公室(学校办公室):\uhref{https://dwbgs.sdsmu.edu.cn}{https://dwbgs.sdsmu.edu.cn}
    \item 纪委、监察专员办公室:\uhref{https://jiwei.sdsmu.edu.cn}{https://jiwei.sdsmu.edu.cn}
    \item 学生工作处(大学生就业指导中心、武装部):\uhref{https://xshch.sdsmu.edu.cn}{https://xshch.sdsmu.edu.cn}
    \item 教务处:\uhref{https://jwch.sdsmu.edu.cn}{https://jwch.sdsmu.edu.cn}
    \item 财务处:\uhref{https://cwch.sdsmu.edu.cn}{https://cwch.sdsmu.edu.cn}
    \item 保卫处:\uhref{https://bwch.sdsmu.edu.cn}{https://bwch.sdsmu.edu.cn}
    \item 后勤管理处:\uhref{https://zwch.sdsmu.edu.cn}{https://zwch.sdsmu.edu.cn}
    \item 网络信息中心:\uhref{https://nic.sdsmu.edu.cn}{https://nic.sdsmu.edu.cn$^㊕$}
    \item 实践教学管理处:\uhref{https://yyglc.sdsmu.edu.cn}{https://yyglc.sdsmu.edu.cn}
\end{enumerate}

\subsection[日常使用]{\textbf{\uul{日常使用}}}
\begin{enumerate}
    \item 校园网:\uhref{http://210.44.80.65}{http://210.44.80.65$^㊕$}或\uhref{https://slrz.sdsmu.edu.cn}{https://slrz.sdsmu.edu.cn$^㊕$}
    \item 校园网充值:\uhref{https://slzfw.sdsmu.edu.cn:8800/home}{https://slzfw.sdsmu.edu.cn:8800/home$^㊕$}
    \item 统一支付平台(学费缴费):\uhref{https://tyzfpt.sdsmu.edu.cn/xysf/login.aspx}{https://tyzfpt.sdsmu.edu.cn/xysf/login.aspx}
    \item 教务系统:\uhref{https://jwgl.sdsmu.edu.cn}{https://jwgl.sdsmu.edu.cn$^㊕$}
    \item 学生邮箱:\uhref{https://edu.icoremail.net}{https://edu.icoremail.net}
    \item 学工系统:\uhref{https://pjpy.sdsmu.edu.cn}{https://pjpy.sdsmu.edu.cn$^㊕$}
    \item 智慧学工(新生报到相关、新学工系统):\uhref{https://zhxg.sdsmu.edu.cn}{https://zhxg.sdsmu.edu.cn}
    \item 网上共青团(智慧团建):\uhref{https://zhtj.youth.cn/zhtj/signin}{https://zhtj.youth.cn/zhtj/signin}
    \item 安全中心(智慧校园系统):\uhref{https://cas.sdsmu.edu.cn:4102}{https://cas.sdsmu.edu.cn:4102}
    \item 校园邮箱(SSL安全登录):\uhref{https://edu.icoremail.net}{https://edu.icoremail.net}
    \item \item 校园邮箱(普通登录):\uhref{https://mail.stu.sdsmu.edu.cn}{https://mail.stu.sdsmu.edu.cn}
\end{enumerate}

\subsection[论文检索与下载]{论文检索与下载}
\begin{enumerate}
    \item 图书馆:\uhref{https://tsg.sdsmu.edu.cn}{https://tsg.sdsmu.edu.cn$^㊕$}
    \item CARSI:\uhref{https://ds.carsi.edu.cn/login/index.html}{https://ds.carsi.edu.cn/login/index.html}
    \item 资源访问控制系统:\uhref{https://webvpn.sdsmu.edu.cn}{https://webvpn.sdsmu.edu.cn}
    \item 统一身份认证平台:\uhref{https://cas.sdsmu.edu.cn}{https://cas.sdsmu.edu.cn}
    \item 网上办事大厅:\uhref{https://portal.sdsmu.edu.cn}{https://portal.sdsmu.edu.cn$^㊕$}
\end{enumerate}

\subsection[大创、竞赛等]{大创、竞赛等\footnotemark}
\footnotetext{部分网址仅在特定开放时段可用,敬请注意。}
\subsubsection[大创]{大创}
\begin{enumerate}
    \item 国家级大学生创新训练计划平台:\uhref{http://gjcxcy.bjtu.edu.cn}{http://gjcxcy.bjtu.edu.cn}
    \item “互联网$^+$”大学生创新创业大赛:\uhref{http://cxcyds.zju.edu.cn}{http://cxcyds.zju.edu.cn}
    \item 省级大创管理系统:\uhref{https://cxcy.sdei.edu.cn}{https://cxcy.sdei.edu.cn}
\end{enumerate}
\subsubsection[竞赛]{竞赛}
\begin{enumerate}
    \item 中国大学生医学技术技能大赛:\uhref{https://medu.bjmu.edu.cn/bjmu/show.do?code=index}{https://medu.bjmu.edu.cn/bjmu/show.do?code=index}
    \item 全国大学生生命科学竞赛:\uhref{https://culsc.cn}{https://culsc.cn}
    \item “挑战杯”全国大学生课外学术科技作品竞赛:\uhref{https://www.tiaozhanbei.net}{https://www.tiaozhanbei.net}
    \item “挑战杯”中国大学生创业计划竞赛:同上,\uhref{https://www.tiaozhanbei.net}{https://www.tiaozhanbei.net}
    \item 山东省大学生医学技术技能大赛:\uhref{https://item.micecube.com/yxjs/index.html}{https://item.micecube.com/yxjs/index.html}
    \item 全国大学生英语竞赛:\uhref{https://www.chinaneccs.cn}{https://www.chinaneccs.cn}
    \item 外研社・国才杯:\uhref{https://ucc.fltrp.com}{https://ucc.fltrp.com}
    \item “数字人杯”基础医学综合实验技能竞赛:详见下发文件的相关要求
    \item “泰山杯”全国医学影像技术专业大学生(本科)实践技能大赛:见文件相关要求
    \item 山东省高等医学院校大学生专业技能竞赛:参见下发文件的相关要求
    \item 山东大学解剖知识技能大赛:见下发文件的相关二维码,无固定链接
    \item 医学形态学读片和人体解剖标本辨识技能大赛:见下发文件的相关要求
    \item 外教社・词达人杯:见下发文件的相关要求及“词达人”微信公众号相关要求
    \item “中国电信杯”山东省大学生国家安全知识竞赛:见下发文件的相关要求及“山东学校安全基地”公众号说明
\end{enumerate}
\subsubsection[其他]{其他}
\begin{enumerate}
    \item 全国大学生社会实践管理服务平台:\uhref{https://shsj.5idream.net}{https://shsj.5idream.net}
    \item 青鸟计划:详见公众号“青鸟计划”及学校下发的相关文件说明
\end{enumerate}

\subsection[学习与考试]{学习与考试}
\begin{enumerate}
    \item 智慧树(知到):\uhref{https://zhihuishu.com}{https://zhihuishu.com}
    \item 长江雨课堂:\uhref{https://changjiang.yuketang.cn/web}{https://changjiang.yuketang.cn/web}
    \item 人卫医学题库:\uhref{https://tk.ipmph.com/exam/a/adminlogin}{https://tk.ipmph.com/exam/a/adminlogin}
    \item 讯飞考试平台:\uhref{https://www.fifedu.com/iplat/html/index.html}{https://www.fifedu.com/iplat/html/index.html}
\end{enumerate}

\subsection[资格水平考试]{资格水平考试}
\begin{enumerate}
    \item 全国大学英语四、六级考试:\uhref{https://cet-kw.neea.edu.cn}{https://cet-kw.neea.edu.cn}
    \item 国家普通话水平测试:\uhref{https://www.cltt.org}{https://www.cltt.org}
    \item 全国计算机水平考试:\uhref{https://ncre.neea.edu.cn}{https://ncre.neea.edu.cn}
    \item 国家医学考试网:\uhref{https://www1.nmec.org.cn}{https://www1.nmec.org.cn}
    \item 中小学教师资格考试:\uhref{https://ntce.neea.edu.cn}{https://ntce.neea.edu.cn}
    \item 托福:\uhref{https://www.toefl.cn}{https://www.toefl.cn}
    \item 雅思:\uhref{https://www.chinaielts.org}{https://www.chinaielts.org}
    \item 日本语能力测试:\uhref{https://jlpt.neea.cn/index.do}{https://jlpt.neea.cn/index.do}
\end{enumerate}

\subsection[学籍查询]{学籍查询}
\label{student_status_query}
中国高等教育学生信息网(学信网):\uhref{https://www.chsi.com.cn}{https://www.chsi.com.cn}

%\newpage
\section[应用]{应用}
\noindent\makebox[\textwidth][c]{%
    \begin{minipage}{.6\textwidth}
        \subsection[手机app]{手机app}
        \centering
        \label{sdsmu_app}
        \begin{tblr}[
            tall,
            theme = {no-caption},
            remark{注} = {山二医App下载链接\uhref{http://ydxy.sdsmu.edu.cn/mobileapi_ydxy/open/goDownload}{见此},其他App请在手机自带的应用市场下载。},
            ]{
            width = 0.9\linewidth,
            colspec = {X[1]X[1]X[1]},
            vlines,
            hlines,
            hline{1,Z} = {-}{1pt},
            cells={c,m}
            }
            课程表  & 大白U帮      & \textbf{山二医App} \\
            人卫    & 知到         & 医考帮             \\
            钉钉    & 思维导图     & 腾讯会议           \\
            词典    & 翻译         & Edge浏览器         \\
            学信网  & 学习强国     & 山东教育发布       \\
            fif口语 & 大英思博英语 & 批改网
        \end{tblr}
    \end{minipage}

    \begin{minipage}{.4\textwidth}
        \subsection[小程序]{小程序}
        \centering
        \begin{tblr}[
            theme = {no-caption},
            ]{
            cells={c,m},
            width=0.8\linewidth,
            colspec={X[1]X[1]X[1]},
            cell{1,4}{1} = {c=2}{cmd=\bfseries},
            cell{5}{1} = {c=2}{},
            vlines,
            hlines,
            hline{1-2,4-5,Z} = {-}{1pt},
            }
            QQ小程序   &          \\
            腾讯文档   & 金山文档 \\
            群投票统计 & 收集表   \\
            微信小程序 &          \\
            长江雨课堂 &          \\
            校务行     & 智慧学工 \\
            建议诉求   & 海乐生活
        \end{tblr}
    \end{minipage}
}

\subsection[微信公众号]{微信公众号}
\begin{table}[H]
    \centering
    \begin{tblr}[
        theme = {no-caption},
        ]{
        width=0.6\linewidth,
        colspec={X[1]X[1]},
        cells={c,m},
        vlines,
        hlines,
        hline{4} = {-}{0.8pt},
        hline{1,7-8,X,Z} = {-}{1pt},
        }
        山东第二医科大学                  & ~报                 \\
        ~学生之家(\textbf{新生预报到}) & ~智慧校园           \\
        ~财务处(\textbf{学费缴费})     & ~学生会             \\
        山二医文体中心                    & ~图管会             \\
        青春~                            & ~就业服务           \\
        平安~                            & ~学工助手           \\
        青春山东(青年大学习)            & 潍坊公交大学生乘车码 \\
        闪医 Spark Medicine               & 银成医考             \\
        天天师兄                          & 医客亮哥             \\
        对分易                            & 长江雨课堂           \\
        流云海印(自助打印)              & 海享租(空调)
    \end{tblr}
\end{table}

\subsection[常用课程表对比]{常用课程表对比}
\label{schedule_app}
\begin{table}[H]
    \centering
    \begin{tblr}[
            theme = {no-caption},
        ]{
            cells = {c,m},
            row{1} = {cmd=\bfseries},
            rowhead=1,
            vlines,
            hlines,
            hline{1-2,Z} = {-}{1pt},
        }
        程序名称     & 广告 & 难易程度 & 使用人数 & 各类功能                           \\
        Wakeup课程表 & 无   & 中等     & 中等     & 分享,教务导入,桌面小工具,调课等 \\
        超级课程表   & 有   & 容易     & 多       & 同上,另有“蹭课”等功能,部分需会员 \\
        山二医app    & 无   & 容易     & 较少     & 极少,但与官方教务系统同步,不折腾
    \end{tblr}
\end{table}

%常用网站及app%通用指南目录

% 后记
\chapter[后记]{后记}
\section[地图绘制]{地图绘制}
刚入学时,我就始终因为不熟悉学校布局,且实验室与教室门牌号复杂,需要提前许多才能按时进入教室而苦恼;后来,在需要去各个办公室提交申请资料时,又发现找不到办公室的地点。

当时,网络上的主流校园整体地图是由“林弄人”在2015年绘制的\uline{\href{https://www.zcool.com.cn/work/ZMTgxMDQwMjg=.html}{《手绘潍医》}},但其在鄙人看来略显抽象,部分建筑的大小、空间关系因美观而妥协的情况使得地图无法真实反应现实,也缺少了许多重要的细节。此外,学校的校园情况、建筑也在按实际需要逐步扩建、更改,更加让本应承担“指路”功能的地图易于“迷路”。

因此,在2022年10月左右,本人四处寻找适宜用作地图的图纸原稿,最终决定以“UG潍医”微信小程序的开发者提供的《潍坊医学院新校区修建性详细规划(调整)》为初稿的参考,通过使用不同颜色的色块标识建筑与路面,根据实际情况增加路名与建筑名称,以学校实际建设情况为据增添新建筑等形成了初稿并发布($Mika$ 为地图的曾用署名)。

而后,随着使用量的增加,各类建议也纷至沓来,地图的细节亦不断完善。此后,在各位校领导以及学校宣传部张主任的指导下,地图的用语更加规范、格式更加正式。2024年7月,借助高德的粗略卫星图再次重绘了浮烟山校区的地图,并在宣传部各领导的帮助下增添了许多细节。此外,我又在21级临床医学院学工办刘主任的帮助下重制了教学楼的内部详细地图。

因虞河校区缺少地图,故此,在2023年寒假期间本人开始着手绘制虞河校区及人民医院见习点的整体轮廓地图。在先前绘制积累的经验的帮助下,虞河校区地图于2024年3月10日正式完成。鄙人谨在此再一次向所有指导、审核与提供帮助的各位领导、教师以及同学致以诚挚的感谢!

在2024年7月,为方便新生寻找实验室具体位置,本人使用 Affinity® Designer 突破性地增加了敏行楼的详细门牌号地图(虽然因为敏行楼上下楼层门牌号不对应、各门牌号排布极其杂乱而难以辨认,\sout{但有总比没有好})。

\section[指南写作]{指南写作}

在2022年末,山东第二医科大学频道(曾用名:潍坊医学院表白墙QQ频道)希望我基于自己改制的地图与其提供的原始大纲,编撰一份入学指南,我也很愉快地答应了。但是,在整合的过程中,我发现其中的众多内容不符合当今实际、许多语句措辞不当、各文章病句繁多、各文档内排版混乱等问题,我也一并进行了修正。

但是由于各种错误层出不穷,逐个修正费时耗力且事倍功半,我当即决定以供稿内容为基础,自行抽象出共性内容与特性内容,仅将供稿视作文章目录并系统地重写了一份,以便使文章逻辑顺畅、文风统一,也就有了今天的《山东第二医科大学入学与生活指南》。

在此过程中,学校各部门的领导、老师给予了我很大的帮助,没有各位领导、老师的鼎力相助也就不可能有本指南的诞生,因文章长度所限无法一一致谢还请谅解。

\section[效果优化]{效果优化}
\subsection[文字显示]{文字显示}
自豪地使用\uline{\href{https://www.maoken.com/freefonts/15311.html}{梦源宋体}}(
\uline{\href{https://github.com/adobe-fonts/source-han-serif}{思源宋体}}的改版,降低了文字的行高)进行排版,作为宋体糟糕显示效果的替代品。

\subsection[图片压缩]{图片压缩}
在地图的制作与指南的发布过程中,因地图清晰度过高、图像画布过大导致的图片文件过大\footnotemark 的问题始终困扰着我。在反复研究后,我通过非文字区域的马赛克化、复用色彩、更换 $MozJpeg$ 图片算法、图片文字矢量化等多种方法尽力减小图像体积,力求在地图“保真”的前提下缩小图片体积使之易于传播。最终,我将各地图以 $pdf$ 文件的形式嵌入到了本指南中,且所有地图文件总大小不超过2㎆,实现了质的飞跃。
\footnotetext{浮烟山校区原始整体地图编辑并导出后的png文件大小可达200㎆以上。}

在2024年2月,再次更改导出图片的方式,通过 GIMP 将 $xcf$ 文件直接导出为 $pdf$ 图像(内嵌矢量文本)的方式再次减小了文件体积。

在2024年7月,使用 Affinity® Designer 重绘了浮烟山校区的整体地图,借助矢量图形的特性再次大幅缩小了文件体积同时提高了清晰度。

\subsection[书签、区域和表格的管理与排版]{书签、区域和表格的管理与排版}
一开始,本文采用Microsoft® Word 2021进行排版并保存为 $docx$ 文件。但是,在文稿频繁交换意见、改进与审核的交换过程中出现了——不同版本的程序显示不同,高分屏与普通屏排版不同,Word、LibreOffice Writer与WPS重复打开并保存后格式混乱需要全面重新排版,书签与超链接难以进行统一管理等诸多问题。因此,早在2022年末,我便已开始计划用\TeX 语言进行重写,但因种种原因未能付诸行动。

在2023年暑假期间,我着手使用\LaTeX 进行全面重构,在“庚午版 2023.7.21”在新生群发布以后,经慎重考虑,鄙人谨决定彻底放弃维护Word版本(原Word版本由“山东第二医科大学频道”接手维护),用全部精力维护LaTeX版本以保证文稿质量。最终,在2023年8月13日,LaTeX初版维护完毕,并不断跟随学校实际更新。

\subsection[编辑]{编辑}
为避免文章引用混乱、格式在不同Office版本上不断改变,本文章全部使用\TeX 语言撰写,使任何电脑都能在同时使用\uline{\href{https://tug.org/texlive}{Tex Live套件}}与\uline{\href{https://code.visualstudio.com}{VSCode编辑器}}\footnotemark 的情况下获得完全相同的编辑体验(\sout{虽然相较Word编辑的入门门槛也高了亿点点})。最终发布时通过 $latexmk$ 自动调用 \XeLaTeX 多次编译以获得 $pdf$ 文件,保证无论是纸质版还是电子版都能在相同的格式下进行阅读。
\footnotetext{安装与编辑说明详见\uline{\href{https://gitee.com/LinkChou/sdsmu_welcome_tex/blob/master/README.md}{README.md}}。}

\section[实际应用]{实际应用}
最后,特别感谢刘书记,在他的支持下,大多数临床医学院2023级新生在入学前通览了本文,避免了众多常见错误的再犯,显著减少了因使用违规电支持器导致的停电次数,极大地降低了学生因不熟悉校园而产生的各类问题的发生率!
%后记
\end{document}
