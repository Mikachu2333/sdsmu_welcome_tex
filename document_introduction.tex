%chapter1
\chapter[简介]{简介\vspace{-2em}}
\section[文章说明]{文章说明}
\vspace{-1em}
\subsection[前言]{前言}
\vspace{-1em}
自豪地使用梦源宋体与梦源黑体进行排版,作为宋体与黑体糟糕显示效果的替代品。

为避免文章引用混乱与格式在不同Office版本上的不断改变,本文章全部使用\TeX 语言撰写,从而保证无论任何电脑都能在同时使用Tex Live套件VSCode编辑器\footnotemark 的情况下获得完全相同的编辑体验(\sout{虽然相较Word编辑的入门门槛也高了超多})。最终发布时选用\XeLaTeX 引擎二次编译以获得 $pdf$ 文件,保证无论是纸质版还是电子版都能在相同的格式下进行阅读。
\footnotetext{需加载LaTeX Workshop插件。}

本文以 \textbf{CC-BY-SA-4.0} 许可开源,仓库地址为\uline{\href{https://gitee.com/mikazo/guide_for_freshman}{Gitee}},欢迎各位同学参与项目的贡献!

本文的 $pdf$ 版本\textbf{\uuline{最新版本下载地址}}为:\textbf{\uline{\href{https://docs.qq.com/s/ETcQ-ZFSrSsh6MK9bm773q}{腾讯文档}}(推荐)}或前往Gitee仓库的“发行版”下载。

\subsection[中心思想]{中心思想}
\vspace{-1em}
本文始终围绕\textbf{“有备无患”}的思想编写,因此部分数据可能略有超出通常实际所需,敬请\textbf{酌情增减}。此外,本文力图\textbf{“举要治繁”},常规事物本文不再赘述。

\subsection[多版本对比]{多版本对比}
\vspace{-1em}
本文章(即\textbf{《山二医入学与生活指南》})原始大纲由“\textbf{山东第二医科大学表白墙QQ频道}”提供,后经过“\textbf{\uuline{LinkChou}}”对原始主体部分进行多次全面重写与扩充而成,并保持积极的更新趋势,力求与学校实际以及同学需求保持同步。

而\textbf{《山二医新生开学指南》}是由“浮烟山小麻花”“vv”“花海”等人在本文的基础上进行重排版而成的新版本,通常以一年为周期进行版本更新。该版本相较本版,存在部分别字及过时内容。

\textbf{\uuline{简而言之,各位读者可视《山二医入学与生活指南》为最新版本,《山二医新生开学指南》\\为现代化的主流发布版本。}}

\section[内容概要]{内容概要}
\vspace{-1em}

各位新同学可以在本文章中了解新生报到的各类要求,推荐准备的物品,学校的地图,宿舍的简要情况,周边的吃喝玩乐地点等。

各位学长与学姐可以把本文作为一个快捷的查阅工具以快速查找一些常见问题的解决方案。
\bigbreak
本文难免有疏漏之处,敬请斧正。

本文具体内容均以学校官方为准,内容如有变更恕不另行通知。