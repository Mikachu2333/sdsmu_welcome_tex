%chapter1
\chapter[简介]{简介}

\section[许可证及项目信息]{许可证及项目信息}
山二医入学与生活指南 © 2023 by LinkChou is licensed under CC BY-SA 4.0. To view a copy of this license, visit \uline{\href{http://creativecommons.org/licenses/by-sa/4.0/}{http://creativecommons.org/licenses/by-sa/4.0/}}

仓库地址为\textbf{\uline{\href{https://gitee.com/mikazo/guide_for_freshman}{Gitee}}},欢迎各位参与项目的贡献!

\section[\textcolor{red}{下载与更新}]{下载与更新}
本文的 $pdf$ \textbf{\uuline{最新版本下载地址}}为:\textbf{\uline{\textcolor{red}{\href{https://docs.qq.com/s/ETcQ-ZFSrSsh6MK9bm773q}{腾讯文档}}}(推荐)}或前往\uline{\href{https://gitee.com/mikazo/guide_for_freshman}{Gitee仓库}}的“\uline{\href{https://gitee.com/mikazo/latex_version/releases/latest}{发行版}}”下载。如需更新请直接使用最新版本文件。

\section[前言]{前言}
\subsection[题外话]{题外话}
自豪地使用梦源宋体与梦源黑体进行排版,作为宋体与黑体糟糕显示效果的替代品。

为避免文章引用混乱与格式在不同Office版本上的不断改变,本文章全部使用\TeX 语言撰写,使任何电脑都能在同时使用Tex Live套件VSCode编辑器\footnotemark 的情况下获得完全相同的编辑体验(\sout{虽然相较Word编辑的入门门槛也高了超多})。最终发布时选用\XeLaTeX 引擎二次编译以获得 $pdf$ 文件,保证无论是纸质版还是电子版都能在相同的格式下进行阅读。
\footnotetext{需加载\LaTeX\ Workshop插件。}

\subsection[中心思想]{中心思想}
本文始终围绕\textbf{“有备无患”}的思想编写,部分数据可能略超实际所需,敬请\textbf{酌情增减}。此外,本文力图\textbf{“举要治繁”},常规事物不再赘述。

\subsection[多版本对比]{多版本对比}
本文章(即\textbf{《山二医入学与生活指南》})原始大纲由“\textbf{山二医校园频道}”提供,后经过“\textbf{\uuline{LinkChou}}”对原始主体部分进行多次全面重写与扩充而成,并保持积极的更新趋势,力求与学校实际以及同学需求保持同步。

而\textbf{《山二医新生开学指南》}是由“浮烟山小麻花”“vv”“花海”等人在本文的基础上进行重排版与校园卡优惠政策的增添而成的新版本,通常以一年为周期进行版本更新。该版本相较本版,可能存在部分别字及过时内容。

\textbf{\uuline{简而言之,各位读者可视《山二医入学与生活指南》为最新版本,《山二医新生开学指南》\\为现代化的主流发布版本。}}

\section[特别感谢]{特别感谢}

\begin{enumerate}
    \item 潍坊医学院表白墙提供的原始大纲与改进思路;
    \item 宣传部张老师的指导以及路名、建筑名规范化的提议;
    \item 临床医学院党委书记刘主任的指导以及提供试点的机会;
    \item 临床医学院学工办刘主任提供的教学楼地图原稿;
    \item 所有为本指南、地图提供建议、改进意见的所有同学、教师。
\end{enumerate}
