%chapter1
\chapter[简介]{简介\vspace{-2em}}
\section[文章说明]{文章说明}
\vspace{-1em}
\subsection[前言]{前言}
\vspace{-1em}
自豪地使用梦源宋体与梦源黑体进行排版。

为避免文章引用混乱与格式在不同Office版本上的不断改变,本文章全部使用\TeX 语言撰写,保证无论任何电脑都能在同时使用Tex Live套件VSCode编辑器和LaTeX Workshop的情况下获得完全相同的编辑体验(\sout{虽然相较Word编辑的入门门槛也高了许多})。最终发布时选用\XeLaTeX 引擎二次编译以获得 $pdf$ 文件,保证无论是纸质版还是电子版都能在相同的格式下进行阅读。

本文以 \textbf{CC-BY-SA-4.0} 许可开源,仓库地址为\uline{\href{https://gitee.com/mikazo/guide_for_freshman}{Gitee}},欢迎各位同学参与项目的贡献!

本文的 $pdf$ 版本\textbf{\uuline{最新版本下载地址}}为:\textbf{\uline{\href{https://docs.qq.com/s/ETcQ-ZFSrSsh6MK9bm773q}{腾讯文档}}(推荐)}或前往Gitee仓库的“发行版”下载。

\subsection[中心思想]{中心思想}
\vspace{-1em}
本文始终围绕\textbf{“有备无患”}的思想编写,因此部分数据可能略有超出通常实际所需,敬请酌情增减。此外,本文力图\textbf{“举要治繁”},常规事物本文不再赘述。

\subsection[多版本对比]{多版本对比}
\vspace{-1em}
本文章(即\textbf{《潍坊医学院新生入学指南》})原始大纲部分由“\textbf{潍坊医学院表白墙QQ频道}”提供,后经过“\textbf{\uuline{Mika}}”对原始主体部分进行多次全面重写与扩充而成,并保持积极的更新趋势,力求与学校实际以及同学需求保持同步。

而\textbf{《潍医新生开学指南》}是由“浮烟山小麻花”“vv”“花海”等人在本文的基础上进行重排版而成的新版本,通常以一年为周期进行版本更新。该版本相较本版,存在部分别字及过时内容。

\textbf{\uuline{简而言之,各位读者可视《潍坊医学院新生入学指南》为最新版本,《潍医新生开学指南》为\\现代化的主流发布版本。}}

\section[内容概要]{内容概要}
\vspace{-1em}

各位新同学可以在本文章中了解新生报到的各类要求,推荐准备的物品,学校的地图,宿舍的简要情况,周边的吃喝玩乐地点等。

本文难免有疏漏之处,敬请斧正。

本文具体内容均以学校官方为准,内容如有变更恕不另行通知。