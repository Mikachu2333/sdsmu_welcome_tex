%chapter1
\chapter[简介]{简介}

\section[许可证及项目信息]{许可证及项目信息}
山二医入学与生活指南 © 2023 by LinkChou is licensed under CC BY-SA 4.0. To view a copy of this license, visit \uline{\href{http://creativecommons.org/licenses/by-sa/4.0/}{http://creativecommons.org/licenses/by-sa/4.0/}}

仓库地址为\textbf{\uline{\href{https://gitee.com/mikazo/guide_for_freshman}{Gitee}}},欢迎各位参与项目的贡献!

\section[\textcolor{red}{下载与更新}]{下载与更新}
本文的 $pdf$ \textbf{\uuline{最新版本下载地址}}为:\textbf{\uline{\textcolor{red}{\href{https://docs.qq.com/s/ETcQ-ZFSrSsh6MK9bm773q}{腾讯文档}}}(推荐)}或前往\uline{\href{https://gitee.com/mikazo/guide_for_freshman}{Gitee仓库}}的“\uline{\href{https://gitee.com/mikazo/latex_version/releases/latest}{发行版}}”下载。如需更新请直接使用最新版本文件。

\section[前言]{前言}
\subsection[题外话]{题外话}
自豪地使用梦源宋体与梦源黑体进行排版,作为宋体与黑体糟糕显示效果的替代品。

为避免文章引用混乱与格式在不同Office版本上的不断改变,本文章全部使用\TeX 语言撰写,使任何电脑都能在同时使用Tex Live套件VSCode编辑器\footnotemark 的情况下获得完全相同的编辑体验(\sout{虽然相较Word编辑的入门门槛也高了超多})。最终发布时选用\XeLaTeX 引擎二次编译以获得 $pdf$ 文件,保证无论是纸质版还是电子版都能在相同的格式下进行阅读。
\footnotetext{需加载\LaTeX\ Workshop插件。}

\subsection[前因后果]{前因后果}
刚入学时,我就始终因为不熟悉学校布局,且实验室与教室门牌号复杂,需要提前许多才能按时进入教室而苦恼;后来,在需要去各个办公室提交申请资料时,又发现找不到办公室的地点。当时,网络上的主流校园整体地图是由“林弄人”在2015年绘制的\uline{\href{https://www.zcool.com.cn/work/ZMTgxMDQwMjg=.html?}{《手绘潍医》}},但其在鄙人看来略显抽象,部分建筑的大小、空间关系因美观而妥协的情况使得地图无法真实反应现实,也缺少了许多重要的细节。%
此外,学校的校园情况、建筑也在按实际需要逐步扩建、更改,更加让本应承担“指路”功能的地图易于“迷路”。

因此,在2022年10月左右,本人四处寻找适宜用作地图的图纸原稿,最终决定以“UG潍医”微信小程序的开发者提供的《潍坊医学院新校区修建性详细规划(调整)》为初稿,通过使用不同颜色的色块标识建筑与路面,根据实际情况增加路名与建筑名称,以学校实际建设情况为据增添新建筑等形成了初稿并发布\footnotemark。后期,又通过非重要区域的马赛克化、复用色彩、更换图片算法等多种方法尽力减小图像体积,力求在地图“保真”的前提下减小图片体积。%
\footnotetext{Mika为地图的曾用署名。}%
而后,随着使用量的增加,各类建议也纷至沓来,地图的细节亦不断完善。此后,在各位校领导以及学校宣传部张主任的指导与不断帮助中,地图的用语更加规范,格式更加正式;此外,我又在临床医学院学工办刘主任的帮助下重制了教学楼的内部详细地图。鄙人谨在此向所有指导、审核与提供帮助的各位领导、教师以及同学致以诚挚的感谢!

在2022年末,山二医校园频道\footnotemark,希望我基于自己改制的地图与其提供的原始大纲,编撰一份入学指南,我也很愉快地答应了。但是,在整合的过程中,我发现了其中的众多内容不符合当今实际、许多语句措辞不当、各文章病句繁多、各文档内排版混乱等问题,我也一并进行了修正。%
\footnotetext{曾用名:潍坊医学院表白墙QQ频道。}%
奈何修正到最后我不耐烦了,决定以此为基础,系统地重写一份,以便使文章逻辑通顺、文风统一,也就有了今天的《山二医入学与生活指南》\footnotemark。
\footnotetext{曾用名:《潍坊医学院新生入学指南》、《潍坊医学院新生入学与生活指南(浮烟山校区)》。}%
在此特别感谢临床医学院刘书记,在他的支持下,临床医学院2023级新生基本在入学前通览了本文,避免了众多常见错误的再犯,显著减少了因使用违规电器导致的停电次数,极大地降低了学生因不熟悉校园而产生的各类问题的发生率!

一开始,本文采用的是Microsoft® Word 2021版进行的排版。但是,在文稿频繁审核、改进的交换过程中出现了不同版本Word显示不同,高分屏与普通屏排版不同,Word、LibreOffice Writer与WPS打开并保存以后格式混乱需要全面重新排版,书签与超链接难以进行管理等诸多问题。因此,早在2022年末,我便已开始计划用\TeX 语言对此进行重写,但因种种原因未能付诸行动。%
在2023年暑假期间,我着手使用\LaTeX 进行全面重构,在“庚午版 2023.7.21”在新生群发布以后,经慎重考虑,鄙人谨决定彻底放弃维护Word版本\footnotemark,用全部精力维护\LaTeX 版本以保证文稿质量。最终,在2023年8月13日,\LaTeX 初版维护完毕,并不断跟随学校实际更新。
\footnotetext{原本的Word版本由“山二医校园频道”接手维护。}

\subsection[中心思想]{中心思想}
本文始终围绕\textbf{“有备无患”}的思想编写,部分数据可能略超实际所需,敬请\textbf{酌情增减}。此外,本文力图\textbf{“举要治繁”},常规事物不再赘述。

\subsection[多版本对比]{多版本对比}
本文章(即\textbf{《山二医入学与生活指南》})原始大纲由“\textbf{山二医校园频道}”提供,后经过“\textbf{\uuline{LinkChou}}”对原始主体部分进行多次全面重写与扩充而成,并保持积极的更新趋势,力求与学校实际以及同学需求保持同步。

而\textbf{《山二医新生开学指南》}是由“浮烟山小麻花”“vv”“花海”等人在本文的基础上进行重排版与校园卡优惠政策的增添而成的新版本,通常以一年为周期进行版本更新。该版本相较本版,可能存在部分别字及过时内容。

\textbf{\uuline{简而言之,各位读者可视《山二医入学与生活指南》为最新版本,《山二医新生开学指南》\\为现代化的主流发布版本。}}
