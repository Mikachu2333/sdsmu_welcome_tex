%前言
\chapter[简介]{简介}

\section[许可证及项目信息]{许可证及项目信息}
山二医入学与生活指南 © 2023 by $LinkChou$ is licensed under CC BY-SA 4.0. To view a copy of this license, visit \uline{\href{http://creativecommons.org/licenses/by-sa/4.0/}{http://creativecommons.org/licenses/by-sa/4.0/}}

仓库地址在\textbf{\uline{\href{https://gitee.com/mikazo/guide_for_freshman}{Gitee}}},欢迎各位参与项目的贡献!

\section[\textcolor{red}{下载与更新}]{下载与更新}
本文 $pdf$ \textbf{\uuline{最新正式版下载地址}}为:\textbf{\uline{\textcolor{red}{\href{https://docs.qq.com/s/ETcQ-ZFSrSsh6MK9bm773q}{腾讯文档}}}(推荐)}或前往\uline{\href{https://gitee.com/mikazo/guide_for_freshman}{Gitee仓库}}的“\uline{\href{https://gitee.com/mikazo/latex_version/releases/latest}{发行版}}”下载。如需更新请直接使用最新版 $pdf$ 文件。

Preview版仅供内测,请以正式版内容为准。

\section[前言]{前言}
\subsection[题外话]{题外话}
自豪地使用梦源宋体(思源宋体的改版)进行排版,作为宋体糟糕显示效果的替代品。

为避免文章引用混乱与格式在不同Office版本上的不断改变,本文章全部使用\TeX 语言撰写,使任何电脑都能在同时使用Tex Live套件VSCode编辑器\footnotemark 的情况下获得完全相同的编辑体验(\sout{虽然相较Word编辑的入门门槛也高了亿点点})。最终发布时选用\XeLaTeX 引擎二次编译以获得 $pdf$ 文件,保证无论是纸质版还是电子版都能在相同的格式下进行阅读。
\footnotetext{需加载\LaTeX\ Workshop插件。}

\subsection[行文核心]{行文核心}
“\textcolor{red}{\textbf{有备无患}}”为本文核心思想,敬请“\textbf{酌情增减}”。此外,本文力图“\textcolor{red}{\textbf{举要治繁}}”,常规事物不再赘述。“\textbf{乐道\ 济世}”的校训,和“\textbf{严谨、求是、勤奋、进取}”的校风是本指南编纂过程中的思想指引与指路明灯。\textbf{\uuline{\textcolor{red}{浮烟山校区是本文叙述的重点,如无特别说明均指浮烟山校区}}}。

\subsection[多版本说明]{多版本说明}
本文的原始大纲为“\textbf{山二医校园频道}”提供,由“\textbf{\uuline{LinkChou}}”对原始主体部分进行多次全面重写与扩充而成。更新较为频,力求与学校实际以及同学需求保持同步。

而由“山二医校园频道”发布的新生指南在本文的基础上进行了\textbf{格式美化}并\textbf{添加了各类优惠政策},通常以一年为周期进行版本更新。

\textbf{简而言之,二者各有千秋,敬请各位同学依照自己的实际需要进行选择。}
