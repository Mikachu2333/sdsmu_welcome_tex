%军训
\chapter[军训]{军训}

\section[军训概况]{军训概况}
\begin{enumerate}
    \item 军训时间:3周,报到后马上开始
    \item 内容\footnote{不打靶、不摸枪;少数同学会参与战术方阵,女生有擒敌拳。}:站军姿,蹲下,跨立,齐步走,原地踏步,齐步跑,踢正步,坐马扎
    \item 地点:多数在杏林路树荫下军训(一般晒不着,但仍然建议多备防晒霜)
    \item 教官:通常由退役复员的学长学姐担任,也有部分现役武警(或现役解放军)组成,一般年纪都不是特别大,大部分是90后
    \item 着装:穿着统一的军训套装\footnotemark
          \footnotetext{统一收费。包括T恤,外套,裤子(很肥),腰带(扎在外套外面,需另备一条腰带以免掉裤子),鞋子(底薄、磨脚,推荐鞋子大一号多垫几双鞋垫)和马扎(军训完马扎不要扔,考试月背书的时候很好用)。}
    \item \textbf{如果因为身体原因不能参加军训,需要提前拿病历、住院证明等材料前往校医院开具相关证明\label{exercise_unattend}}(位置参见\uline{\ref{map_a}})
    \item 若出现身体不适要及时向教官报告,较严重的可申请前往校医院或附属医院进行详细检查
    \item \sout{如果顺拐和左右不分情况极其严重的,可以早早申请成为“飞虎队”\footnotemark 的一员}
          \footnotetext{谐音梗,音同废物队;是对严重顺拐同学的无恶意调侃。}
    \item 特点:军训期间时间紧、任务重,少有可以自由安排的时间
    \item 注意:\textbf{\uuline{可能出现长时间阴天}},因此推荐多带袜子,内裤,内衣,T恤以避免一件衣服穿一周(\sout{熏死啦,想想就头大})、袜子晾干以后变得硬邦邦的(\sout{长蘑菇了都})的悲伤结局
\end{enumerate}

\section[军训作息时间表]{军训作息时间表\footnote{以往届作息为例,具体以各学院安排为准。}}
\begin{table}[H]
    \centering
    \begin{tabular}{|c|c|c|c|}
        \hline
        5:30        & 起床 & 12:30-14:00 & 午休  \\
        \hline
        6:00-7:00   & 早操 & 14:00-17:30 & 集训  \\
        \hline
        7:00-7:50   & 早餐 & 17:30-18:30 & 晚餐  \\
        \hline
        7:50-11:00  & 集训 & 18:30-21:30 & 晚自习 \\
        \hline
        11:00-12:30 & 午餐 & 23:00       & 熄灯  \\
        \hline
    \end{tabular}
\end{table}
