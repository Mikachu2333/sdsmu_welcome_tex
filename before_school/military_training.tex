% 军训
\section[军训]{军训}

\subsection[军训概况]{军训概况}
\begin{enumerate}
    \item 军训时间:3周,报到后马上开始
    \item 内容\footnotemark:站军姿、蹲下、跨立、齐步走、原地踏步、齐步跑、踢正步、坐马扎
          \footnotetext{不打靶、不摸枪;少数同学会参与战术方阵,女生有擒敌拳。}
    \item 地点:多数在杏林路树荫下军训(一般晒不着,但仍然建议多备防晒霜)
    \item 教官:按往年情况,一般由部队官兵担任
    \item 着装:穿着统一的军训套装\footnotemark
          \footnotetext{同被褥套装订购教程请看\uref{freshman_query}。含T恤、外套、裤子(很肥)、腰带(扎在外套外面,需另备一条腰带以免掉裤子)、鞋子(底薄、磨脚,推荐鞋子大一号多垫几双鞋垫)和马扎(军训完不要扔,考试月背书很好用)。}
    \item \textbf{如果因为身体原因不能参加军训,需要提前拿病历、住院证明等材料前往校医院开具相关证明\label{exercise_unattend}}(位置见\uref{map_fuyanshan_holistic})
    \item 若出现身体不适要及时向教官报告,较严重的可申请前往校医院或附属医院进行详细检查
    \item \st{如果顺拐和左右不分情况极其严重的,可以早早申请成为“飞虎队”\footnotemark 的一员}
          \footnotetext{谐音梗,音同废物队;是对严重顺拐同学的无恶意调侃。}
    \item 特点:军训期间时间紧、任务重,少有可以自由安排的时间
    \item 注意:\textbf{\uul{可能出现长时间阴天}},因此推荐多带袜子、内裤、内衣、T恤以避免一件衣服穿一周(\st{熏死啦,想想就头大})、袜子晾干以后变得硬邦邦的(\st{长蘑菇了都})的悲伤结局
\end{enumerate}

\subsection[军训作息时间表]{军训作息时间表\footnotemark}
\footnotetext{以2024级作息为例,具体以各学院安排为准。}
\begin{table}[H]
    \centering
    \begin{tblr}[
            theme = {no-caption},
        ]{
            cells = {c,m},
            hlines,
            vlines,
        }
        06:20        & 起床 & 12:20~13:50 & 午休   \\
        06:50~07:20 & 早操 & 14:20~17:20 & 集训   \\
        07:25~07:50 & 早餐 & 17:30~18:30 & 晚餐   \\
        08:00~11:20 & 集训 & 18:30~21:30 & 晚自习 \\
        11:30~12:00 & 午餐 & 22:30        & 熄灯   \\
    \end{tblr}
\end{table}
