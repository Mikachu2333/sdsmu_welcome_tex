% 费用
\section[费用、ATM与银行卡]{费用、ATM与银行卡}

\subsection[生活费]{生活费}
一般生活费范围约1000~1500元(仅吃饭和购买水果、牛奶等常规生活情景)

\subsection[学费]{学费\footnotemark}
\footnotetext{学费缴费系统使用教程详见\uline{\ref{fee_pay}},财务处地址:行政楼1层西侧。}
\subsubsection[一般情况说明]{一般情况说明}
\begin{enumerate}
    \item 学费通常分为四部分\footnotemark
          \footnotetext{详见财务处文件\uline{\href{https://cwch.sdsmu.edu.cn/_upload/article/files/47/c4/47b017924d219d29d49dd079335a/f04e22ac-aa91-4c49-bceb-f53280c93164.pdf}{《学分制收费管理办法》}}、\uline{\href{https://cwch.sdsmu.edu.cn/_upload/article/files/ce/67/093a0208463a8a5bb333e3117fc6/dfb39999-8683-49e7-b8b0-0ed7af8a6150.pdf}{《关于印发山东省高等学校住宿费收费管理办法的通知》}}等。}
          \begin{enumerate}
              \item 专业注册学费:不同专业不一,按学年收取
              \item 学分学费:根据各人选修课情况收费,按学期收取
              \item 住宿费:不同校区、楼栋标准不一,按学年收取
              \item 书费:订书方式多样,费用与收费时间均不确定
          \end{enumerate}
    \item 依惯例,新生入学时需要进行缴纳第一学期的专业注册学费与住宿费,请按录取通知书说明以及新生预报到系统\footnotemark 的提示进行
          \footnotetext{预报到系统使用教程见此\uline{\ref{freshman_query}}。}
    \item 学费的收缴工作在开学后按照学校财务处通知进行,通常以班级为单位进行通知,可开具电子发票
    \item 如需申请助学贷款、生源地贷款等有特殊情况的同学详询财务处收费管理科(联系方式详见官网)
    \item 选修课学费按照学分进行收费,所以同一个班的同学学费亦可不同,个人学费以\textbf{“山东第二医科大学财务处”公众号(山东第二医科大学校园统一缴费平台)}中的数据为准
    \item 选修课学分当前规定为1分/100元,多选课多交钱,少选课少交钱(希望大家如果看到了自己希望进一步学习的课程不要吝啬那几百块钱,选课机会只有一次,课程不会重开!)
    \item 此外,一定结合培养方案的要求(参见\uline{\ref{score}})修够学分,\textbf{否则无法毕业}
\end{enumerate}

\subsubsection[助学项目一览表]{助学项目一览表}
\begin{tblr}[
        long,
        theme = {no-caption},
    ]{
        cells  = {c,m},
    }
    test & test \\
    test & ???
\end{tblr}

\subsection[ATM]{ATM\footnotemark}
\footnotetext{根据业务动态调整,详情见各银行官网。}
\begin{enumerate}
    \item 工商银行:教学楼E区,靠近杏林路侧
    \item 邮政银行:中和广场,A103对过
\end{enumerate}

\subsection[银行卡]{银行卡}
学校随录取通知书寄出未激活银行卡一张,请按照官方提示进行操作

该银行卡将在大学5年中用于奖助学金和助学贷款发放等用途

\textbf{\uuline{严禁出借银行卡,否则可能导致违法犯罪!!!}}
