% 费用
\section[费用、ATM与银行卡]{费用、ATM与银行卡}

\subsection[生活费]{生活费}
\begin{enumerate}
    \item 下述额外支出含:\textbf{书费等费用}、\textbf{水果蔬菜}、健身运动、零食饮料、各类聚餐团建、游玩、网购(任意物品)等
    \item 一般生活费极限底线约900元(仅吃饭,无其他任何额外支出\footnotemark)
          \footnotetext{\textbf{注意:该标准仅供参考、极不推荐且不含任何突发事件的弹性缓冲空间!仅限浮烟山校区!};参考标准如下(不构成医疗建议):早餐8元、午餐12元、晚餐10元,可能偶有超限。}
    \item 常规生活标准约1200-1500元,基本涵盖额外支出部分,\uref{tuition}{书费}等特殊支出可能仍会使生活费超限
    \item 虞河校区物价稍高,各标准需上浮约200元;其他见习医院不做说明
\end{enumerate}

\subsection[学费]{学费\footnotemark}
\footnotetext{学费缴费系统使用教程\uref{fee_pay}{见此},财务处地址:行政楼1层西侧。}
\subsubsection[一般情况说明]{一般情况说明}
\label{tuition}
\begin{enumerate}
    \item 学费通常分为四部分\footnotemark
          \footnotetext{详见财务处文件\uhref{https://cwch.sdsmu.edu.cn/_upload/article/files/47/c4/47b017924d219d29d49dd079335a/f04e22ac-aa91-4c49-bceb-f53280c93164.pdf}{《学分制收费管理办法》}、\uhref{https://cwch.sdsmu.edu.cn/_upload/article/files/ce/67/093a0208463a8a5bb333e3117fc6/dfb39999-8683-49e7-b8b0-0ed7af8a6150.pdf}{《关于印发山东省高等学校住宿费收费管理办法的通知》}等。}
          \begin{enumerate}
              \item 专业注册学费:不同专业不一,按学年收取
              \item 学分学费:根据各人选修课情况收费,按学期收取
              \item 住宿费:不同校区、楼栋标准不一,按学年收取
              \item 书费:订书方式多样,费用与收费时间均不确定;通常是第一学期交两次钱款(本学期与下学期),之后的每学期提前交下学期书费;请以班级群内通知为准
          \end{enumerate}
    \item 依惯例,新生入学时需要进行缴纳第一学期的专业注册学费与住宿费,请按录取通知书说明以及新生预报到系统\footnotemark 的提示进行
          \footnotetext{预报到系统使用教程\uref{freshman_query}{见此}。}
    \item 学费的收缴工作在开学后按照学校财务处通知进行,通常以班级为单位进行通知,可开具电子发票
    \item 公费医学生与助学贷款学生无须缴纳专业注册学费、学分学费、住宿费3类费用
    \item 如需申请助学贷款、生源地贷款等有特殊情况的同学详询财务处收费管理科(联系方式详见官网)
    \item 选修课学费按照学分进行收费,所以同一个班的同学学费亦可不同,个人学费以\textbf{“山东第二医科大学财务处”公众号(山东第二医科大学校园统一缴费平台)}中的数据为准
    \item 选修课学分当前规定为1分/100元,多选课多交钱,少选课少交钱(希望大家如果看到了自己希望进一步学习的课程不要吝啬那几百块钱,选课机会只有一次,课程不会重开!)
    \item 此外,一定结合培养方案的要求(见\uref{score}{此处})修够学分,\textbf{否则无法毕业}
\end{enumerate}

\subsection[助学项目与困难补助]{助学项目与困难补助}
\subsubsection[助学项目一览]{助学项目一览}
敬请参照山东省教育厅学生资助管理中心\uhref{https://sdxszz.sdei.edu.cn/Show/7784}{《山东省普通高等学校资助政策简介》}或咨询学校教师
\subsubsection[困难补助一览]{困难补助一览}
详见学校下发的《山东第二医科大学新生报到指南》

\subsection[ATM]{ATM\footnotemark}
\footnotetext{根据业务动态调整,详情见各银行官网。}
\begin{enumerate}
    \item 工商银行:教学楼E区,靠近杏林路侧
    \item 邮政银行:中和广场,A103对过
\end{enumerate}

\subsection[银行卡]{银行卡}
学校统一为各位同学办理一张中国工商银行卡,在入学后发放。该卡包含学生个人信息,将用于大学生活中用于奖助学金等其他费用的发放

\textbf{\uuline{严禁出借银行卡,否则可能导致违法犯罪!!!}}
