% 前言
\chapter[指南简介]{指南简介}

\section[许可证及项目信息]{许可证及项目信息}
山东第二医科大学指南 © 2023-2024 by \textcolor{blue}{$LinkChou$} is licensed under CC BY-SA 4.0. To view a copy of this license, visit \uhref{http://creativecommons.org/licenses/by-sa/4.0}{http://creativecommons.org/licenses/by-sa/4.0}.

Gitee仓库地址:\uhref{https://gitee.com/LinkChou/sdsmu_welcome_tex}{https://gitee.com/LinkChou/sdsmu\_welcome\_tex}

Github仓库地址:\uhref{https://github.com/Mikachu2333/sdsmu_welcome_tex}{https://github.com/Mikachu2333/sdsmu\_welcome\_tex}

\section[版权声明]{版权声明}
\label{copyright}
\subsection[严正声明]{严正声明}
本人对且仅对\textbf{“由本人发布的”}、\textbf{“已通过审核的”}、\textbf{“正式版本的”}《山东第二医科大学指南》的言论负有直接责任。

在本项目的衍生项目中,除下文列举的项目以外,由其他维护者(贡献者)添加的内容\textbf{必须经过本人审查后方可通过}。\textbf{未经审查的内容均与本人无关,本人不对其正确性、时效性、思想情况做出任何保证},由其引发的任何问题也不由本人负责,而是由负责该衍生项目的负责人负责。

\subsection[版权声明]{版权声明}
下列作品均由\textbf{周大为(LinkChou)}创作并保留所有权利,转载请标明作者姓名(笔名亦可)与出处。本文其他未做声明的内容均根据鄙人个人生活经验、学校公告及网络公开资料汇总整理并提炼精简,如有错漏敬请指明。

\begin{enumerate}
    \item \textbf{《山东第二医科大学指南》}
    \item \textbf{《山东第二医科大学地图(浮烟山校区)〔矢量版〕》}(鲁作登字-2024-K-00630461)
    \item \textbf{《山东第二医科大学地图(虞河校区)〔矢量版〕》}(鲁作登字-2024-K-00630460)
    \item \textbf{《山东第二医科大学地图(浮烟山校区敏行楼)〔矢量版〕》}(鲁作登字-2024-K-00630462)
\end{enumerate}

\subsection[授权信息等]{授权信息等}
详见正式版的相关内容,此处不做赘述。

其他未尽事宜请联系\uhref{Mailto:LinkChou@yandex.com}{Mailto: LinkChou@yandex.com}。

\section[\textcolor{red}{下载与更新}]{下载与更新}
\subsection{文档}
最新\textbf{正式版}下载地址:\textbf{\textcolor{red}{\uhref{https://docs.qq.com/s/ETcQ-ZFSrSsh6MK9bm773q}{腾讯文档}}} 或 \uhref{https://gitee.com/LinkChou/sdsmu_welcome_tex/releases/latest}{Gitee发行版} 或 \uhref{https://github.com/mikachu2333/sdsmu_welcome_tex/releases/latest}{GitHub发行版}

\textbf{Preview版}\footnotemark:\uhref{https://github.com/Mikachu2333/sdsmu_welcome_tex/actions}{GitHub Actions自动编译} 或 自行使用 $git\ clone$ 后本地编译
\footnotetext{Preview版本仅用于内测,请以正式版内容为准。}

\subsection[地图]{地图}
百度网盘:\uhref{https://pan.baidu.com/s/1cZpGGFIABB50u-3lst44iQ?pwd=46pa}{https://pan.baidu.com/s/1cZpGGFIABB50u-3lst44iQ?pwd=46pa}

阿里云网盘:\uhref{https://www.alipan.com/s/dZMvgXwkxGp}{https://www.alipan.com/s/dZMvgXwkxGp}

\section[前言]{前言}
\subsection[行文核心]{行文核心}
“\textcolor{red}{\textbf{有备无患、举要治繁}}”为本文核心思想,敬请“\textbf{酌情增减}”。“\textbf{乐道\ 济世}”的校训和“\textbf{严谨、求是、勤奋、进取}”的校风是本指南编纂过程中的思想指引与指路明灯。

\subsection[多版本说明]{多版本说明}
本文的原始大纲由“\textbf{山东第二医科大学频道}”提供,后经“\textbf{LinkChou}”对原始主体部分进行多次全面重写与扩充而成,内容更加全面、更新相对频繁。

由“山东第二医科大学频道”(QQ)发布的版本在本文的基础上进行了\textbf{精简、格式美化}并\textbf{添加了各类优惠政策},通常在暑假中发版,每学年更新一次。

\textbf{简而言之,二者各有千秋。}
