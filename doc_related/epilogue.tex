% 后记
\chapter[后记]{后记}
\section[地图绘制]{地图绘制}
刚入学时,我就始终因为不熟悉学校布局,且实验室与教室门牌号复杂,需要提前许多才能按时进入教室而苦恼;后来,在需要去各个办公室提交申请资料时,又发现找不到办公室的地点。

当时,网络上的主流校园整体地图是由“林弄人”在2015年绘制的\uhref{https://www.zcool.com.cn/work/ZMTgxMDQwMjg=.html}{《手绘潍医》},但其在鄙人看来略显抽象,部分建筑的大小、空间关系因美观而妥协的情况使得地图无法真实反应现实,也缺少了许多重要的细节。此外,学校的校园情况、建筑也在按实际需要逐步扩建、更改,更加让本应承担“指路”功能的地图易于“迷路”。

因此,在2022年10月左右,本人四处寻找适宜用作地图的图纸原稿,最终决定以“UG潍医”微信小程序的开发者提供的《潍坊医学院新校区修建性详细规划(调整)》为初稿的参考,通过使用不同颜色的色块标识建筑与路面,根据实际情况增加路名与建筑名称,以学校实际建设情况为据增添新建筑等形成了初稿并发布($Mika$ 为地图的曾用署名)。

而后,随着使用量的增加,各类建议也纷至沓来,地图的细节亦不断完善。此后,在各位校领导以及学校宣传部张主任的指导下,地图的用语更加规范、格式更加正式。2024年7月,借助高德的粗略卫星图再次重绘了浮烟山校区的地图,并在宣传部各领导的帮助下增添了许多细节。此外,我又在21级临床医学院学工办刘主任的帮助下重制了教学楼的内部详细地图。

因虞河校区缺少地图,故此,在2023年寒假期间本人开始着手绘制虞河校区及人民医院见习点的整体轮廓地图。在先前绘制积累的经验的帮助下,虞河校区地图于2024年3月10日正式完成。鄙人谨在此再一次向所有指导、审核与提供帮助的各位领导、教师以及同学致以诚挚的感谢!

在2024年7月,为方便新生寻找实验室具体位置,本人使用 Affinity® Designer 突破性地增加了敏行楼的详细门牌号地图(虽然因为敏行楼上下楼层门牌号不对应、各门牌号排布极其混乱而难以辨认,\st{但有总比没有好})。

\section[指南写作]{指南写作}

在2022年末,山东第二医科大学频道(曾用名:潍坊医学院表白墙QQ频道)希望我基于自己改制的地图与其提供的原始大纲,编撰一份入学指南,我也很愉快地答应了。但是,在整合的过程中,我发现其中的众多内容不符合当今实际、许多语句措辞不当、各文章病句繁多、各文档内排版混乱等问题,我也一并进行了修正。

但是由于各种错误层出不穷,逐个修正费时耗力且事倍功半,我当即决定以供稿内容为基础,自行抽象出共性内容与特性内容,仅将供稿视作文章目录并系统地重写了一份,以便使文章逻辑顺畅、文风统一,也就有了今天的《山东第二医科大学指南》。

在此过程中,学校各部门的领导、老师给予了我很大的帮助,没有各位领导、老师的鼎力相助也就不可能有本指南的诞生,因文章长度所限无法一一致谢还请谅解。

\section[效果优化]{效果优化}
\subsection[文字显示]{文字显示}
自豪地使用\uhref{https://www.maoken.com/freefonts/15311.html}{梦源宋体}(\uhref{https://github.com/adobe-fonts/source-han-serif}{思源宋体}的改版,降低了文字的行高)进行排版,作为宋体糟糕显示效果的替代品。

\subsection[图片压缩]{图片压缩}
在地图的制作与指南的发布过程中,因地图清晰度过高、图像画布过大导致的图片文件过大\footnotemark 的问题始终困扰着我。在反复研究后,我通过非文字区域的马赛克化、复用色彩、更换 $MozJpeg$ 图片算法、图片文字矢量化等多种方法尽力减小图像体积,力求在地图“保真”的前提下缩小图片体积使之易于传播。最终,我将各地图以 $pdf$ 文件的形式嵌入到了本指南中,且所有地图文件总大小不超过2㎆,实现了质的飞跃。
\footnotetext{浮烟山校区原始整体地图编辑并导出后的png文件大小可达200㎆以上。}

在2024年2月,再次更改导出图片的方式,通过 GIMP 将 $xcf$ 文件直接导出为 $pdf$ 图像(内嵌矢量文本)的方式再次减小了文件体积。2024年7月,使用 Affinity® Designer 重绘了浮烟山校区的整体地图,借助矢量图形的特性再次大幅缩小了文件体积同时提高了清晰度。2024年8月以矢量图的形式重绘了浮烟山校区教学楼的地图,增加了特殊入口的标识并缩小了体积。

\subsection[书签、区域和表格的管理与排版]{书签、区域和表格的管理与排版}
一开始,本文采用Microsoft® Word 2021进行排版并保存为 $docx$ 文件。但是,在文稿频繁交换意见、改进与审核的交换过程中出现了——不同版本的程序显示不同,高分屏与普通屏排版不同,Word、LibreOffice Writer与WPS重复打开并保存后格式混乱需要全面重新排版,书签与超链接难以进行统一管理等诸多问题。因此,早在2022年末,我便已开始计划用\LaTeX 语言进行重写,但因种种原因未能付诸行动。

在2023年暑假期间,我着手使用\XeLaTeX 进行全面重构,在“庚午版 2023.7.21”在新生群发布以后,经慎重考虑,鄙人谨决定彻底放弃维护Word版本(原Word版本由“山东第二医科大学频道”接手维护),用全部精力维护LaTeX版本以保证文稿质量。最终,在2023年8月13日,LaTeX初版维护完毕,并不断跟随学校实际更新。

\subsection[编辑]{编辑}
为避免文章引用混乱、格式在不同Office版本上不断改变,本文章全部使用\LaTeX 撰写,使任何电脑都能在同时使用\uhref{https://tug.org/texlive}{Tex Live套件}与\uhref{https://code.visualstudio.com}{VSCode编辑器}\footnotemark 的情况下获得完全相同的编辑体验。(\st{虽然相较Word编辑的入门门槛也高了亿点点……})
\footnotetext{安装与编辑说明详见\uhref{https://gitee.com/LinkChou/sdsmu_welcome_tex/blob/master/README.md}{README.md}。}

最终发布时通过 $latexmk$ 自动调用 \XeTeX 多次编译以获得 $pdf$ 文件,保证无论是纸质版还是电子版都能在相同的格式下进行阅读。

2024年8月1日,为优化显示效果更换为 \LuaTeX 编译。

\section[实际应用]{实际应用}
最后,特别感谢刘书记,在他的支持下,大多数临床医学院2023级新生在入学前通览了本文,避免了众多常见错误的再犯,显著减少了因使用违规电支持器导致的停电次数,极大地降低了学生因不熟悉校园而产生的各类问题的发生率!

\section[版权信息与授权相关]{版权信息与授权相关}
\label{copyright}
\subsection[版权声明]{版权声明}
下列作品均由\textbf{周大为(LinkChou)}创作并保留所有权利,转载请标明作者姓名(笔名亦可)与出处。本文其他未做声明的内容均根据鄙人个人生活经验、学校公告及网络公开资料汇总整理并提炼精简,如有错漏敬请指明。

\begin{enumerate}
    \item \textbf{《山东第二医科大学指南》}
    \item \textbf{《山东第二医科大学地图(浮烟山校区)〔矢量版〕》}(鲁作登字-2024-K-00630461)
    \item \textbf{《山东第二医科大学地图(虞河校区)〔矢量版〕》}(鲁作登字-2024-K-00630460)
    \item \textbf{《山东第二医科大学地图(浮烟山校区敏行楼)〔矢量版〕》}(鲁作登字-2024-K-00630462)
\end{enumerate}

\subsection[授权信息]{授权信息}
\begin{enumerate}
    \item \textbf{山东第二医科大学各学院(含二级学院)、各部门}
          \begin{enumerate}
              \item 使用范围:
                    \begin{enumerate}
                        \item 本稿所有文字内容
                        \item 本稿中出现的所有图像及图像原稿、图像改稿
                        \item 本稿中使用的其他相关文件
                    \end{enumerate}
              \item 其他事项:
                    \begin{enumerate}
                        \item 使用时需标注作者姓名(笔名亦可)
                        \item 增加或删除或修改本稿后产生的衍生作品版权归属对应组织所有
                        \item 其他未提及的事项请参照CC BY-SA 4.0许可证的相关说明
                    \end{enumerate}
          \end{enumerate}
    \item \textbf{山东第二医科大学的学生组织(仅含校级、院级)}
          \begin{enumerate}
              \item 使用范围:
                    \begin{enumerate}
                        \item 本稿所有文字内容
                        \item 本稿中出现的所有图像(含PDF格式、JPG格式与PNG格式)
                    \end{enumerate}
              \item 其他事项:
                    \begin{enumerate}
                        \item 使用时需标注作者姓名(笔名亦可)
                        \item 二次创作时不得改变本稿原义(错误修正除外)
                        \item 正式发布前需经过本人审核
                        \item 增加或删除本稿后产生的衍生作品版权归属对应组织所有
                        \item 其他未提及的事项请参照CC BY-SA 4.0许可证的相关说明
                    \end{enumerate}
          \end{enumerate}
    \item \textbf{山东第二医科大学表白墙QQ频道}
          \begin{enumerate}
              \item 使用范围:
                    \begin{enumerate}
                        \item 本稿所有文字内容
                        \item 本稿中出现的所有图像及图像原稿、图像改稿
                        \item 本稿中使用的其他文件
                    \end{enumerate}
              \item 其他事项:
                    \begin{enumerate}
                        \item 使用时需标注作者姓名(笔名亦可)
                        \item 允许商用及向稿件中加入该组织的相关广告内容
                        \item 二次创作时不得改变本稿原义(错误修正除外)
                        \item 正式发布前需经过本人审核
                        \item 增加或删除或修改本稿后产生的衍生作品版权归属对应组织所有
                        \item 其他未提及的事项请参照CC BY-SA 4.0许可证的相关说明
                    \end{enumerate}
          \end{enumerate}
\end{enumerate}

\subsection[其他]{其他}
其他未尽事宜请联系\uhref{Mailto:LinkChou@yandex.com}{Mailto: LinkChou@yandex.com}。
