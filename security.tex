%chapter8
\chapter[安全方面]{安全方面}

\section[用电安全]{用电安全}
\begin{enumerate}
    \item 使用安全的充电器、充电线、插排
    \item 如遇空调、电灯、电风扇等器具出现短路、跳闸、停电等现象时切勿自行维修,请按照\uline{\ref{repair_report}}教程处理
    \item 切勿在宿舍使用冰箱(断电且宿管检查),\textbf{\uuline{如果各位同学需要冷藏保存药物(如胰岛素等),请\\前往校医院或大服2楼的药店处}}(地点参见\uline{\ref{common_locations}}),具体收费情况请咨询相关人员
    \item 切勿私改电路,如确有需要,应提前向宿管及公寓管理委员会\footnotemark 报备并获得相应许可
          \footnotetext{位于2号公寓南门处。}
    \item 宿舍人走灭灯,无人则关电、切断插座电源
\end{enumerate}

\section[防火安全]{防火安全}
\begin{enumerate}
    \item 切勿使用蚊香以免发生火灾
    \item 切勿在宿舍内烹饪,禁止在宿舍使用各类燃气灶、固体酒精便携灶等易燃易爆危险品
    \item \textbf{禁止在宿舍内吸烟},若实在无法抑制请前往本楼层公共厕所或在宿舍外吸烟完毕后再进宿舍
    \item 请妥善保管打火机、打火石、镁条、火柴等易燃物
\end{enumerate}

\section[出行安全]{出行安全}
\begin{enumerate}
    \item 学校周边基础设施尚在完善建设过程中(\sout{属实是兔葵燕麦、雨井烟垣}),且频繁有货车高速通过。如无特殊情况,\textbf{尽量不要骑公共自行车或者电动车去市里}(泰华城)等,推荐乘坐71路公交车(预计行程30分钟左右)或打车前往
    \item 货车转弯盲区大,极其容易发生安全事故,在等红绿灯时请务必远离“站立禁止区域”
    \item 乘坐出租车(尤其是拼车时)请妥善保管自身财物;如遇失窃请尽快报警,避免正面冲突(谨防持刀伤人)
    \item \textbf{\uuline{严格遵守交通规则,仔细观察周围情况,切忌边看手机边前进,切忌闯红灯}!}
\end{enumerate}

\section[食品安全]{食品安全}
\begin{enumerate}
    \item 如果在餐厅吃坏了肚子或者发现食物质量问题,可前往餐厅一楼东北角的值班室寻求工作人员的帮助
    \item 出现急性腹泻、血便、米泔水样腹泻等情况请尽快前往校医院就诊,切勿拖延
\end{enumerate}

\section[防诈骗及其他注意事项]{防诈骗及其他注意事项}
\begin{enumerate}
    \item \textbf{\uuline{校园贷毁一生,远离高利贷}!}
    \item \textbf{\uuline{杜绝黄赌毒!不要高估自己的意志力}!}
    \item \textbf{刷单就是诈骗!}
    \item 不要贪小便宜乱扫码,信息泄露吃大亏!
    \item \textbf{\uuline{在正式开学、由带班学长学姐拉入年级的官方QQ群之前,所有的主动拉人入群的各类“通}\\\uuline{知群”“官方群”“学生公告群”等均不可信}!}
    \item 如果碰到一些人自称是市场营销专业、经商专业的,需要卖笔卖本子\footnotemark 才能完成期末考试的,千万不要相信!可以直接联系保卫处
          \footnotetext{一支批发0.1$¥$的笔卖10$¥$呢,比百乐斑马这种外国牌子都贵,利润高达10000\%……}
    \item 女生晚上尽量不要独自前往人烟稀少的地方,尤其是西门附近的桃李路等
    \item 谨防诈骗,\textbf{学校永远不会以教务处、学工办的名义,以邮件表格或短信链接的形式通知填写银行卡号和取款密码!}绝对不要相信以“更新银行卡信息”、“填表申请助学金”为由窃取个人资金密码的骗局!如果不确定消息是否属实请致电本班班长,班主任或学工办老师确认
\end{enumerate}
