%生活
\section[衣食购住玩与生活]{衣食购住玩与生活}

\subsection*{特别声明}
\begin{enumerate}
    \item 本文中所有\textbf{“大服”},均为\textbf{“大学生服务中心”}的习惯性缩略称呼;
    \item 本文中所有\textbf{“南街”},均为\textbf{“汇金街”}的习惯性缩略称呼。
\end{enumerate}
\subsection[衣]{衣}
\begin{enumerate}
    \item 大服的2、3层均有服饰商店可自行选购衣物
    \item 推荐网购,也可乘71路等公共汽车前往大型商超购置
    \item 部分院系提供自愿的系服购买服务,详见各院系通知
\end{enumerate}

\subsection[食]{食}
\subsubsection*{注意}
\begin{enumerate}
    \item 因文章篇幅原因,本指南仅列举了同学们提及次数较多的部分食物或店铺,敬请谅解;
    \item 下列提及的店铺(食物)均按照空间顺序排列,与好吃程度无关;
    \item 所用名称为同学习惯性称呼,括号内为特别提醒或补充说明。
    \item 上标“㊐”的店铺夏季约6:00开始供应(冬季约6:30);
    \item 上标“㊰”的店铺营业时间最晚可至22:30,其余均在18:30~20:30左右停业;
    \item 上标“㊒”的店铺因装修等原因尚未开业或长期停业(超过一周);
    \item 奶茶/咖啡店、水果店等单独说明。
\end{enumerate}

\subsubsection[杏林餐厅]{杏林餐厅}

杏林餐厅全部三层均有大量食物,大多物美价廉。
\begin{table}[H]
    \centering
    \begin{tblr}{
        cells = {c,m},
        cell{1}{1} = {r=3}{},
        cell{4}{1} = {r=3}{},
        cell{7}{2} = {c=4}{},
        vlines,
        hline{1,4,7-8} = {-}{},
        hline{2-3,5-6} = {2-5}{},
            }
        1层              & 麦西麦乐                        & 包子水饺$^㊐$   & 牛肉板面 & 兰州拉面     \\
                        & 大米粒儿$^㊐$(油条)                & 自选菜(稍贵)    & 豆腐脑  & 盒饭(便宜量大) \\
                        & 粥$^㊐$(种类多)                  & 馄饨$^㊐$     & 麻辣烫  & 烤夫王      \\
        2层              & 大骨饭                         & 麻汁馄饨       & 水饺   & 东北玉米面    \\
                        & 烤鸭饭(瓦罐汤)                    & 铁板炒饭(量大管饱) & 米线   & 清真窗口     \\
                        & 馋嘴鱼                         & 自选水饺       & 茶拌饭  & 略        \\
        3层\footnotemark & 略(较贵;有包间,部门聚餐可选,包间人数上限为12人) &            &      &
        \footnotetext{除餐厅东南侧楼梯外均可到达。}
    \end{tblr}
\end{table}

\subsubsection[大服]{大服}
大服有大量商家提供多种食物,大部分的价格较食堂稍高。
\begin{table}[H]
    \centering
    \begin{tblr}{
        cells = {c,m},
        cell{1}{1} = {r=3}{},
        cell{1}{2} = {r=2}{},
        cell{4}{1} = {r=2}{},
        cell{4}{2} = {r=2}{},
        vlines,
        hline{1,4,6} = {-}{},
        hline{2,5} = {3-6}{},
                hline{3} = {2-6}{},
            }
        1层  & 内            & 金小麵$^㊐$(锅贴) & 自选菜                    & 陕西面馆      & 馋嘴鱼      \\
            &              & 新疆炒米粉       & 肠粉                     & 肉夹馍$^㊰$   & 冒菜       \\
            & 外            & 烧烤$^㊰$      & 砂锅$^㊐$(火烧\textbar{}豆脑) & 大饼卷一切$^㊰$ & 速食主义$^㊐$ \\
        -1层 & $\backslash$ & 兰李于         & 自选菜                    & 酸菜鱼       & 螺狮粉      \\
            &              & 烤鸡架         & 宽巷面馆                   & 馋嘴鱼       & 略
    \end{tblr}
\end{table}

\newpage
\subsubsection[汇金街]{汇金街}
出学校南门,往东一个路口。有大量的饭店,价格大多较市里相对高昂,部分味道一般。

\begin{table}[H]
    \centering
    \begin{tblr}{
            cells = {c,m},
            hlines,
            vlines,
        }
        满江红  & 暖溢水饺(相对平价) & 木南王府 & 小四川烧烤      \\
        志科全驴 & 炖大鹅        & 生炖羊茬 & 幸福餐厅(平价量大)
    \end{tblr}
\end{table}

\subsubsection[水果店]{水果店}
\begin{table}[H]
    \centering
    \begin{tabular}{|c|c|c|c|c|c|}
        \Xhline{1.2pt}
        习惯称呼    & 地点                     & 种类 & 新鲜 & 价格 \\
        \Xhline{1.2pt}
        餐厅南水果店  & 餐厅正南侧入口                & 较多 & 较好 & 略高 \\
        \hline
        餐厅西水果店  & 餐厅正西侧入口                & 较少 & 一般 & 一般 \\
        \hline
        大服水果店   & 大服西南侧                  & 最多 & 一般 & 最高 \\
        \hline
        中和/大服超市 & 见 \uline{\ref{market}} & 最少 & 一般 & 一般 \\
        \Xhline{1.2pt}
    \end{tabular}
\end{table}

\subsubsection[奶茶/咖啡店]{奶茶/咖啡店}
\begin{table}[H]
    \centering
    \begin{tabular}{|c|c|c|c|c|c|}
        \Xhline{1.2pt}
        \multirow{2}{*}{食堂} & \textbf{蜜雪冰城}       & 臻茶            & 沪上阿姨  %
                            & 阿水大杯茶               & 麦克风                   \\
        \cline{2-6}
                            & 超级奶爸                & 小度            & 冰雪岛   %
                            & \textbf{瑞幸咖啡}$^{㊒}$ & $\backslash$          \\
        \Xhline{1.2pt}
        \multirow{2}{*}{大服} & 1层/-1层              & \textbf{茶百道}  & 益禾堂   %
                            & \textbf{幸运咖}        & 归臻咖啡                  \\
        \cline{2-6}
                            & 2层                  & \textbf{库迪咖啡} & 遇觅烧仙草 %
                            & 手打冰沙                & $\backslash$          \\
        \Xhline{1.2pt}
    \end{tabular}
\end{table}

\subsection[购]{购}
\label{market}
\begin{table}[H]
    \centering
    \begin{tabular}{|c|c|c|}
        \Xhline{1.2pt}
        习惯称呼       & 地点      & 物品                   \\
        \Xhline{1.2pt}
        大服超市$^{㊰}$ & 在大服正中央  & 日用品,零食,饮料,手套,头套,作业本等 \\
        \hline
        中和超市       & 中和广场    & 日用品(少),零食,饮料,作业本等    \\
        \hline
        餐厅超市       & 餐厅西北侧入口 & 餐巾纸、零食、饮料等           \\
        \Xhline{1.2pt}
    \end{tabular}
\end{table}

\subsection[玩]{玩}
\begin{enumerate}
    \item 多数同学常通过步行前往南街,有KTV、电影院等娱乐场所
    \item 也可通过69、71、101路等公交车(北门乘坐)或13、109路等公交车(南门乘坐)前往市区(如泰华、万达、谷德茂等)游玩
    \item 文体中心(位置参见\uline{\ref{map_fuyanshan_holistic}})内有羽毛球馆、篮球馆(两者互斥)、健身房,还有\textbf{游泳馆}等\footnotemark
          \footnotetext{具体收费标准及预约方式见下文\uline{\ref{sports_center_book}},开放时间见此\uline{\ref{sports_center_operating_hours}}。}
\end{enumerate}

\begin{table}[H]
    \centering
    \caption{文体中心开放时间}
    \label{sports_center_operating_hours}
    \begin{tblr}{
        cells = {c,m},
        row{1} = {font=\bfseries},
        cell{1}{1} = {r=2}{},
        cell{1}{2} = {c=2}{},
        cell{3}{1} = {r=2}{},
        cell{3}{3} = {r=5}{},
        cell{5}{1} = {r=3}{1pt},
        cell{8}{1} = {r=4}{},
        cell{8}{2} = {r=2}{},
        cell{10}{2} = {r=2}{},
        vlines,
        hline{1,3,8,12} = {-}{1pt},
        hline{2,10} = {2-3}{},
        hline{4,6-7} = {2}{},
        hline{5} = {1-2}{},
        hline{9,11} = {3}{},
            }
        开放项目 & 营业时间 \footnotemark &             \\
        \footnotetext{仅限校内,校外政策详见公众号或咨询工作人员;请以学校通知为准。}
             & 周一至周四              & 周五、周末及法定节假日 \\
        健体中心 & 11:45~13:45        & 08:00~21:00 \\
             & 18:00~21:00        &             \\
        羽毛球馆 & 08:00~09:30        &             \\
             & 12:00~13:30        &             \\
             & 18:00~21:00        &             \\
        游泳馆  & 12:00~14:00        & 09:00~11:00 \\
             &                    & 12:00~14:00 \\
             & 18:00~20:00        & 15:00~17:00 \\
             &                    & 18:00~20:00
    \end{tblr}
\end{table}

\subsection[住]{住}
\begin{enumerate}
    \item 宾馆:南街提供大量宾馆、客房等
    \item 自习室:南街部分宾馆提供通宵自习服务
    \item 出租房:附近小区有较多房屋出租\footnotemark
          \footnotetext{须在学校办理走读手续后才可在外居住。}
\end{enumerate}

\subsection[自提点]{自提点}
\begin{table}[H]
    \centering
    \begin{tblr}{
        cells = {c,m},
        cell{2}{1} = {r=2}{},
        cell{4}{1} = {r=2}{},
        row{1} = {font=\bfseries},
        vlines,
        hline{1-2,4,6} = {-}{1pt},
        hline{3,5} = {2-3}{},
            }
        类别   & 地点      & 显示名称         \\
        美团优选 & 中和移动营业厅 & 山二医美团优选移动    \\
             & 9号宿舍楼   & 山二医九号楼群内免费送货 \\
        多多买菜 & 中和移动营业厅 & 潍医多多买菜移动营业厅  \\
             & 9号宿舍楼   & 山二医九号楼自提
    \end{tblr}
\end{table}

\subsection[其他生活常用地点]{其他生活常用地点}
\label{common_locations_fuyanshan}

\begin{longtblr}{
    cells = {c,m},
    rowhead = 1, rowfoot=0,
    row{1} = {font=\bfseries},
    cell{2}{1} = {r=3}{},
    cell{5}{1} = {r=16}{},
    cell{21}{1} = {r=6}{},
    vlines,
    hline{1-2,5,21,27} = {-}{1pt},
    hline{3-17,19-20,22-26} = {2-4}{},
        }
    地点   & 习惯称呼                & 位置      & 功能                           \\
    中和广场 & 学生印务                & A106对过  & 打印复印扫描、\textbf{复习资料、二手书}     \\
         & 移动营业厅               & A104对过  & 移动业务办理                       \\
         & 酷跑文印社 \footnotemark & A103对过  & 打印复印扫描                       \\
    大服   & 厕所                  & -1层东北   & 略                            \\
         & 联通营业厅               & 1层超市旁   & 联通业务办理                       \\
         & 药店、牙科               & 2层西北    & 买药、看牙、\textbf{冷藏药品}          \\
         & 理发店(三家)             & 2层      & 烫染剪发                         \\
         & 复印店(两家)             & 2层      & \textbf{打印复印扫描、证件照}、 复习资料    \\
         & 电信营业厅               & 2层      & 电信业务办理                       \\
         & 移动业务咨询处             & 2层东     & 移动业务咨询                       \\
         & 广电营业厅               & 2层北     & 广电业务办理                       \\
         & 干洗店                 & 2层东     & 干洗、实验服购买、配钥匙                 \\
         & 裁缝店                 & 2层东南    & 改衣                           \\
         & 维修店                 & 2层东南    & 手机电脑维修、配件购买                  \\
         & 联想服务中心              & 2层西     & 维修检测、配件购买                    \\
         & \textbf{办公室}        & 2层东北    & 办水卡、充值退卡                     \\
         & 大服健身房 \footnotemark & 3层      & 运动健身、办理会员卡                   \\
    \footnotetext{仅大服北侧楼梯可前往,健身卡收费详情咨询工作人员,与文体中心健身房不同。}
         & 台球厅                 & 3层      & 打台球                          \\
         & 彩购师                 & 3层      & 衣物与饰品购买                      \\

    \footnotetext{清晰度较“学生印务”略高,少量打印时价格略高。}
    其他   & 证件照                 & B207旁   & \textbf{证件照}、特殊复印(80g/120g纸) \\
         & \textbf{证明打印}       & D105旁   & \textbf{学籍证明、成绩证明}等          \\
         & 自助打印                & 餐厅北侧    & 打印                           \\
         & 二手书买卖               & 大服西北角   & \textbf{二手书}(大量)             \\
         & 校医院                 & 仁和山     & \textbf{药品购买与冷藏,证明开具}        \\
         & 邮局                  & 餐厅西北侧入口 & 学校纪念品购买 \footnotemark        \\
    \footnotetext{注:该邮局无信件投递及接收业务。}

\end{longtblr}
